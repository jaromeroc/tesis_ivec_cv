% Options for packages loaded elsewhere
\PassOptionsToPackage{unicode}{hyperref}
\PassOptionsToPackage{hyphens}{url}
\documentclass[
]{book}
\usepackage{xcolor}
\usepackage{amsmath,amssymb}
\setcounter{secnumdepth}{5}
\usepackage{iftex}
\ifPDFTeX
  \usepackage[T1]{fontenc}
  \usepackage[utf8]{inputenc}
  \usepackage{textcomp} % provide euro and other symbols
\else % if luatex or xetex
  \usepackage{unicode-math} % this also loads fontspec
  \defaultfontfeatures{Scale=MatchLowercase}
  \defaultfontfeatures[\rmfamily]{Ligatures=TeX,Scale=1}
\fi
\usepackage{lmodern}
\ifPDFTeX\else
  % xetex/luatex font selection
\fi
% Use upquote if available, for straight quotes in verbatim environments
\IfFileExists{upquote.sty}{\usepackage{upquote}}{}
\IfFileExists{microtype.sty}{% use microtype if available
  \usepackage[]{microtype}
  \UseMicrotypeSet[protrusion]{basicmath} % disable protrusion for tt fonts
}{}
\makeatletter
\@ifundefined{KOMAClassName}{% if non-KOMA class
  \IfFileExists{parskip.sty}{%
    \usepackage{parskip}
  }{% else
    \setlength{\parindent}{0pt}
    \setlength{\parskip}{6pt plus 2pt minus 1pt}}
}{% if KOMA class
  \KOMAoptions{parskip=half}}
\makeatother
\usepackage{color}
\usepackage{fancyvrb}
\newcommand{\VerbBar}{|}
\newcommand{\VERB}{\Verb[commandchars=\\\{\}]}
\DefineVerbatimEnvironment{Highlighting}{Verbatim}{commandchars=\\\{\}}
% Add ',fontsize=\small' for more characters per line
\usepackage{framed}
\definecolor{shadecolor}{RGB}{248,248,248}
\newenvironment{Shaded}{\begin{snugshade}}{\end{snugshade}}
\newcommand{\AlertTok}[1]{\textcolor[rgb]{0.94,0.16,0.16}{#1}}
\newcommand{\AnnotationTok}[1]{\textcolor[rgb]{0.56,0.35,0.01}{\textbf{\textit{#1}}}}
\newcommand{\AttributeTok}[1]{\textcolor[rgb]{0.13,0.29,0.53}{#1}}
\newcommand{\BaseNTok}[1]{\textcolor[rgb]{0.00,0.00,0.81}{#1}}
\newcommand{\BuiltInTok}[1]{#1}
\newcommand{\CharTok}[1]{\textcolor[rgb]{0.31,0.60,0.02}{#1}}
\newcommand{\CommentTok}[1]{\textcolor[rgb]{0.56,0.35,0.01}{\textit{#1}}}
\newcommand{\CommentVarTok}[1]{\textcolor[rgb]{0.56,0.35,0.01}{\textbf{\textit{#1}}}}
\newcommand{\ConstantTok}[1]{\textcolor[rgb]{0.56,0.35,0.01}{#1}}
\newcommand{\ControlFlowTok}[1]{\textcolor[rgb]{0.13,0.29,0.53}{\textbf{#1}}}
\newcommand{\DataTypeTok}[1]{\textcolor[rgb]{0.13,0.29,0.53}{#1}}
\newcommand{\DecValTok}[1]{\textcolor[rgb]{0.00,0.00,0.81}{#1}}
\newcommand{\DocumentationTok}[1]{\textcolor[rgb]{0.56,0.35,0.01}{\textbf{\textit{#1}}}}
\newcommand{\ErrorTok}[1]{\textcolor[rgb]{0.64,0.00,0.00}{\textbf{#1}}}
\newcommand{\ExtensionTok}[1]{#1}
\newcommand{\FloatTok}[1]{\textcolor[rgb]{0.00,0.00,0.81}{#1}}
\newcommand{\FunctionTok}[1]{\textcolor[rgb]{0.13,0.29,0.53}{\textbf{#1}}}
\newcommand{\ImportTok}[1]{#1}
\newcommand{\InformationTok}[1]{\textcolor[rgb]{0.56,0.35,0.01}{\textbf{\textit{#1}}}}
\newcommand{\KeywordTok}[1]{\textcolor[rgb]{0.13,0.29,0.53}{\textbf{#1}}}
\newcommand{\NormalTok}[1]{#1}
\newcommand{\OperatorTok}[1]{\textcolor[rgb]{0.81,0.36,0.00}{\textbf{#1}}}
\newcommand{\OtherTok}[1]{\textcolor[rgb]{0.56,0.35,0.01}{#1}}
\newcommand{\PreprocessorTok}[1]{\textcolor[rgb]{0.56,0.35,0.01}{\textit{#1}}}
\newcommand{\RegionMarkerTok}[1]{#1}
\newcommand{\SpecialCharTok}[1]{\textcolor[rgb]{0.81,0.36,0.00}{\textbf{#1}}}
\newcommand{\SpecialStringTok}[1]{\textcolor[rgb]{0.31,0.60,0.02}{#1}}
\newcommand{\StringTok}[1]{\textcolor[rgb]{0.31,0.60,0.02}{#1}}
\newcommand{\VariableTok}[1]{\textcolor[rgb]{0.00,0.00,0.00}{#1}}
\newcommand{\VerbatimStringTok}[1]{\textcolor[rgb]{0.31,0.60,0.02}{#1}}
\newcommand{\WarningTok}[1]{\textcolor[rgb]{0.56,0.35,0.01}{\textbf{\textit{#1}}}}
\usepackage{longtable,booktabs,array}
\usepackage{calc} % for calculating minipage widths
% Correct order of tables after \paragraph or \subparagraph
\usepackage{etoolbox}
\makeatletter
\patchcmd\longtable{\par}{\if@noskipsec\mbox{}\fi\par}{}{}
\makeatother
% Allow footnotes in longtable head/foot
\IfFileExists{footnotehyper.sty}{\usepackage{footnotehyper}}{\usepackage{footnote}}
\makesavenoteenv{longtable}
\usepackage{graphicx}
\makeatletter
\newsavebox\pandoc@box
\newcommand*\pandocbounded[1]{% scales image to fit in text height/width
  \sbox\pandoc@box{#1}%
  \Gscale@div\@tempa{\textheight}{\dimexpr\ht\pandoc@box+\dp\pandoc@box\relax}%
  \Gscale@div\@tempb{\linewidth}{\wd\pandoc@box}%
  \ifdim\@tempb\p@<\@tempa\p@\let\@tempa\@tempb\fi% select the smaller of both
  \ifdim\@tempa\p@<\p@\scalebox{\@tempa}{\usebox\pandoc@box}%
  \else\usebox{\pandoc@box}%
  \fi%
}
% Set default figure placement to htbp
\def\fps@figure{htbp}
\makeatother
\setlength{\emergencystretch}{3em} % prevent overfull lines
\providecommand{\tightlist}{%
  \setlength{\itemsep}{0pt}\setlength{\parskip}{0pt}}
\usepackage[]{natbib}
\bibliographystyle{apalike}
% =========================================================
% preamble.tex — Configuración IVEC TESIS OS++ v9 (versión estable)
% =========================================================
\usepackage[utf8]{inputenc}
\usepackage[T1]{fontenc}
\usepackage{setspace}
\usepackage{geometry}
\usepackage{xcolor}
\usepackage{graphicx}
\usepackage{booktabs}
\usepackage{threeparttable}
\usepackage{amsmath, amssymb}
\usepackage{float}
\usepackage{titlesec}
\usepackage{fancyhdr}
\usepackage{natbib}
\usepackage{hyperref}
\usepackage{enumitem}
\usepackage{fontspec}

% --- Fuentes portables y modernas (TeX Gyre) ---------------
\setmainfont{TeX Gyre Termes}   % equivalente a Times New Roman
\setsansfont{TeX Gyre Heros}    % equivalente a Arial
\setmonofont{TeX Gyre Cursor}   % equivalente a Courier New

% --- Márgenes y formato de página --------------------------
\geometry{
  a4paper,
  top=2.5cm,
  bottom=2.5cm,
  left=3cm,
  right=2.5cm
}

% --- Espaciado y párrafos ---------------------------------
\setstretch{1.5}
\setlength{\parindent}{15pt}
\setlength{\parskip}{0.5em}

% --- Color institucional ----------------------------------
\definecolor{ivec}{HTML}{096B72}

% --- Encabezados y pies de página --------------------------
\pagestyle{fancy}
\fancyhf{}
\fancyhead[L]{\nouppercase{\leftmark}}
\fancyhead[R]{Tesis IVEC TESIS OS++ v9}
\fancyfoot[C]{\thepage}

% --- Títulos de capítulos y secciones ----------------------
\titleformat{\chapter}[block]
  {\huge\bfseries\color{ivec}}{\thechapter.}{1em}{}
\titleformat{\section}
  {\Large\bfseries\color{ivec}}{\thesection}{1em}{}
\titleformat{\subsection}
  {\large\bfseries\color{gray}}{\thesubsection}{1em}{}

% --- Hipervínculos y metadatos PDF -------------------------
\hypersetup{
  colorlinks  = true,
  linkcolor   = ivec,
  citecolor   = ivec,
  urlcolor    = ivec,
  pdfauthor   = {Juan Antonio Romero Crespo},
  pdftitle    = {Tesis Doctoral - Modelo IVEC TESIS OS++ v9},
  pdfsubject  = {Evaluación editorial y gobernanza ambiental},
  pdfkeywords = {Vulnerabilidad, Gobernanza, Riesgo, IVEC, MOVE, GCT}
}

% --- Citas APA7 con natbib --------------------------------
\bibliographystyle{apalike}
\setlength{\bibsep}{1pt}

% --- Figuras y tablas -------------------------------------
\setlength{\abovecaptionskip}{10pt}
\setlength{\belowcaptionskip}{5pt}
\renewcommand{\figurename}{Figura}
\renewcommand{\tablename}{Tabla}

% --- Comandos personalizados -------------------------------
\newcommand{\IVEC}{\textsc{IVEC}}
\newcommand{\MOVE}{\textsc{MOVE}}
\newcommand{\GCT}{\textsc{GCT}}

% --- Portada personalizada --------------------------------
\makeatletter
\def\maketitle{
\begin{titlepage}
\centering
{\Large \textbf{UNIVERSIDAD / INSTITUCIÓN}}\\[1cm]
{\huge \textbf{\@title}}\\[1cm]
{\Large \textit{Tesis Doctoral en Gobernanza Ambiental y Riesgo}}\\[2cm]
{\large Autor: \textbf{\@author}}\\[0.5cm]
{\large Director: \textbf{Nombre del Director}}\\[0.5cm]
{\large Compilación: \today}\\[2cm]
{\Large Modelo de Evaluación Editorial}\\
{\Large \textbf{IVEC TESIS OS++ v9}}\\[3cm]
\vfill
\includegraphics[width=0.3\textwidth]{figures/logo-doctorado.png}\\[1cm]
\end{titlepage}
}
\makeatother
\usepackage{bookmark}
\IfFileExists{xurl.sty}{\usepackage{xurl}}{} % add URL line breaks if available
\urlstyle{same}
\hypersetup{
  pdftitle={A Minimal Book Example},
  pdfauthor={John Doe},
  hidelinks,
  pdfcreator={LaTeX via pandoc}}

\title{A Minimal Book Example}
\author{John Doe}
\date{2026-01-09}

\begin{document}
\maketitle

{
\setcounter{tocdepth}{1}
\tableofcontents
}
\chapter{About}\label{about}

This is a \emph{sample} book written in \textbf{Markdown}. You can use anything that Pandoc's Markdown supports; for example, a math equation \(a^2 + b^2 = c^2\).

\section{Usage}\label{usage}

Each \textbf{bookdown} chapter is an .Rmd file, and each .Rmd file can contain one (and only one) chapter. A chapter \emph{must} start with a first-level heading: \texttt{\#\ A\ good\ chapter}, and can contain one (and only one) first-level heading.

Use second-level and higher headings within chapters like: \texttt{\#\#\ A\ short\ section} or \texttt{\#\#\#\ An\ even\ shorter\ section}.

The \texttt{index.Rmd} file is required, and is also your first book chapter. It will be the homepage when you render the book.

\section{Render book}\label{render-book}

You can render the HTML version of this example book without changing anything:

\begin{enumerate}
\def\labelenumi{\arabic{enumi}.}
\item
  Find the \textbf{Build} pane in the RStudio IDE, and
\item
  Click on \textbf{Build Book}, then select your output format, or select ``All formats'' if you'd like to use multiple formats from the same book source files.
\end{enumerate}

Or build the book from the R console:

\begin{Shaded}
\begin{Highlighting}[]
\NormalTok{bookdown}\SpecialCharTok{::}\FunctionTok{render\_book}\NormalTok{()}
\end{Highlighting}
\end{Shaded}

To render this example to PDF as a \texttt{bookdown::pdf\_book}, you'll need to install XeLaTeX. You are recommended to install TinyTeX (which includes XeLaTeX): \url{https://yihui.org/tinytex/}.

\section{Preview book}\label{preview-book}

As you work, you may start a local server to live preview this HTML book. This preview will update as you edit the book when you save individual .Rmd files. You can start the server in a work session by using the RStudio add-in ``Preview book'', or from the R console:

\begin{Shaded}
\begin{Highlighting}[]
\NormalTok{bookdown}\SpecialCharTok{::}\FunctionTok{serve\_book}\NormalTok{()}
\end{Highlighting}
\end{Shaded}

\mainmatter
\pagenumbering{arabic}
\setcounter{page}{1}
\cleardoublepage

\chapter{Introducción}\label{introducciuxf3n}

\section{De la amenaza física a la construcción social del riesgo}\label{de-la-amenaza-fuxedsica-a-la-construcciuxf3n-social-del-riesgo}

Los desastres no deben concebirse como fenómenos puramente naturales. Tal como ha insistido la ecología política de los riesgos, y como recogen los marcos internacionales más recientes, los impactos de inundaciones, olas de calor, incendios forestales o crisis económicas son, ante todo, construcciones sociales que revelan y profundizan desigualdades preexistentes \citep{blaikie1994, wisner2004, undrr2015}. Dicho de otro modo, un mismo evento físico puede generar consecuencias muy distintas según el territorio en el que ocurra y, sobre todo, según las condiciones sociales e institucionales de las comunidades afectadas.

La creciente frecuencia e intensidad de estos eventos, vinculada en gran medida al cambio climático, ha puesto de manifiesto la urgencia de comprender por qué sus efectos se distribuyen de manera tan desigual entre personas y grupos sociales \citep{ipcc2022_ar6wgii}. Este interrogante resulta especialmente relevante en contextos mediterráneos, donde la exposición a riesgos hídricos y climáticos interactúa con desigualdades socioespaciales de larga duración \citep{olcina2018_riesgos}. La pregunta que subyace es clara: ¿qué explica que unos territorios puedan anticipar, resistir y recuperarse de un evento extremo, mientras otros queden atrapados en ciclos de pérdida y fragilidad?

Desde esta perspectiva, la noción de \textbf{vulnerabilidad social} se consolida como categoría analítica central. A diferencia de la vulnerabilidad física (asociada a infraestructuras) o la ambiental (relativa a ecosistemas), la vulnerabilidad social remite a factores socioeconómicos, demográficos y culturales de personas, hogares y comunidades, los cuales determinan tanto su grado de exposición como su capacidad de respuesta. Su manifestación es, además, eminentemente territorial: se inscribe en contextos específicos moldeados por dinámicas urbanas, políticas públicas y tramas locales de exclusión o resiliencia.

La utilidad de este concepto radica, por tanto, en que permite desplazar el foco de atención desde la amenaza física hacia las condiciones estructurales y contextuales que determinan la capacidad de anticipación, afrontamiento y recuperación de las comunidades. Comprender los desastres exige mirar más allá de la magnitud de las amenazas e indagar en las trayectorias históricas, en las dinámicas territoriales y en las configuraciones de desigualdad que hacen a ciertas poblaciones más frágiles que otras.

\section{Genealogía de un concepto en disputa: La evolución de la vulnerabilidad}\label{genealoguxeda-de-un-concepto-en-disputa-la-evoluciuxf3n-de-la-vulnerabilidad}

El concepto de vulnerabilidad no ha sido lineal ni unívoco. Por el contrario, ha atravesado un proceso de resignificación constante, condicionado por el diálogo y las tensiones entre diferentes tradiciones disciplinarias.

En sus primeras formulaciones, la vulnerabilidad se entendió de manera restringida: como \emph{potencial de pérdida} o como simple \emph{exposición al peligro}. Estas definiciones, propias de la escuela de los peligros naturales, asociaban la vulnerabilidad casi exclusivamente a la dimensión física, subrayando la cercanía espacial de personas y bienes respecto de las amenazas. Sin embargo, esta mirada pronto resultó insuficiente para explicar por qué comunidades con niveles similares de exposición experimentaban impactos tan distintos.

En respuesta a esta limitación, los estudios críticos sobre pobreza, desarrollo y exclusión social comenzaron a ampliar el concepto hacia dimensiones sociales y económicas. Autores como \citep{chambers1989}, \citep{castel1995} o \citep{adger1999} mostraron que la vulnerabilidad no podía entenderse sin atender a la privación estructural de recursos, derechos y redes de apoyo. De este modo, se consolidó la idea de que la fragilidad frente a amenazas es menos un estado natural que una \textbf{condición social e históricamente producida}.

A comienzos del siglo XXI, la convergencia de estas tradiciones dio un paso más: la vulnerabilidad empezó a concebirse como una \textbf{propiedad emergente de sistemas socioecológicos complejos}. Trabajos como los de \citep{turner2003} y \citep{birkmann2006} destacaron que las interacciones entre factores sociales, ambientales e institucionales generan configuraciones de riesgo específicas. Esta perspectiva sistémica subrayó dos aspectos decisivos: que la vulnerabilidad es \textbf{dinámica} (cambia en el tiempo) y que es \textbf{multiescalar} (se manifiesta de manera diferente en individuos, hogares, comunidades o territorios).

Los marcos internacionales fueron asimilando esta evolución conceptual. En el AR4 del IPCC (2007), la vulnerabilidad se definía como función de la exposición, la sensibilidad y la capacidad adaptativa. Sin embargo, en el AR5 (2014) se reformuló esta definición: la exposición pasó a considerarse un elemento independiente, mientras que la vulnerabilidad quedó asociada al conjunto de condiciones que determinan la susceptibilidad al daño. Esta distinción es clave, pues evita reducir la vulnerabilidad a la mera presencia en zonas de riesgo. En paralelo, el Marco de Sendai para la Redacción del Riesgo de Desastres (2015--2030) reforzó esta nueva mirada, afirmando que los desastres ``no son naturales'' sino resultado de vulnerabilidades acumuladas, y proponiendo como prioridad el fortalecimiento de la resiliencia comunitaria.

En conjunto, estas aportaciones muestran cómo la vulnerabilidad ha pasado de concebirse como un atributo físico a entenderse como un fenómeno complejo, relacional y situado.

\section{Modelos teóricos de referencia y la necesidad de una síntesis operativa}\label{modelos-teuxf3ricos-de-referencia-y-la-necesidad-de-una-suxedntesis-operativa}

La consolidación del concepto de vulnerabilidad no solo se produjo a través de definiciones sucesivas, sino también mediante modelos teóricos que intentaron ofrecer marcos integrados para su comprensión. Estos modelos han sido determinantes en la configuración de indicadores, metodologías y políticas públicas, y constituyen el trasfondo indispensable para situar la propuesta del IVEC. Su análisis revela una progresión desde la crítica estructural hacia la síntesis operativa, un camino que esta tesis busca culminar.

Uno de los marcos más influyentes es el \textbf{modelo de Presión y Liberación (PAR)}, desarrollado por Blaikie et al. \citep{blaikie1994} y sistematizado posteriormente por Wisner et al. \citep{wisner2004}. Este modelo parte de una idea sencilla pero poderosa: los desastres ocurren cuando la presión ejercida por las causas estructurales y dinámicas de la vulnerabilidad se encuentra con una amenaza natural. Al distinguir entre ``causas raíz'' (desigualdades históricas), ``presiones dinámicas'' (procesos intermedios) y ``condiciones inseguras'' (manifestaciones locales), el PAR reveló cómo las trayectorias de exclusión se acumulan hasta estallar en forma de desastre. Su aporte principal fue, por tanto, situar en el centro del análisis los factores socioeconómicos y políticos que generan vulnerabilidad, ofreciendo un esquema procesual para entenderla como resultado de dinámicas de largo plazo.

De forma complementaria, \textbf{Cardona (2001)} propuso un \textbf{enfoque holístico} de la gestión del riesgo que concibe la vulnerabilidad en relación con la capacidad de anticipación, respuesta y recuperación. Esta perspectiva introdujo un matiz crucial: la vulnerabilidad no se entiende únicamente como fragilidad o carencia, sino también en función de las \textbf{capacidades} disponibles para afrontarla. Desde esta óptica, conceptos como resiliencia y adaptación aparecen como moduladores que permiten reducir los impactos de las amenazas, pasando de un diagnóstico de déficits a un análisis de potencialidades.

Más recientemente, el proyecto europeo \textbf{MOVE (Methods for the Improvement of Vulnerability Assessment in Europe)}, coordinado por Birkmann et al. \citep{birkmann2013}, supuso otro paso importante al proponer un marco explícitamente \textbf{multidimensional}. MOVE distingue entre exposición, susceptibilidad y falta de resiliencia, y despliega seis dimensiones analíticas: social, económica, física, institucional, cultural y ambiental. Su carácter heurístico y adaptable lo convirtió en un referente para la elaboración de indicadores comparables en distintos contextos, influyendo en el IPCC (2014) y en múltiples políticas de adaptación en Europa y América Latina.

En paralelo a estos marcos conceptuales, se produjo un \textbf{giro metodológico} hacia la medición comparativa, cuyo principal exponente fue la extraordinaria expansión de índices sintéticos como el \emph{Social Vulnerability Index} (SoVI) \citep{cutter2003sovi}. Estos instrumentos permitieron operacionalizar la vulnerabilidad de manera cuantitativa, facilitando la cartografía comparativa y la identificación de áreas críticas. Sin embargo, también recibieron críticas importantes por el riesgo de simplificar fenómenos complejos y, en ocasiones, confundir vulnerabilidad con pobreza o precariedad \citep{rufat2023_criticalreview}. El desafío, por tanto, sigue siendo equilibrar la comparabilidad estadística con la sensibilidad territorial.

\section{La propuesta del IVEC: Hacia un índice estructural y contextual}\label{la-propuesta-del-ivec-hacia-un-uxedndice-estructural-y-contextual}

En este contexto de debates teóricos y vacíos metodológicos surge el \textbf{Índice de Vulnerabilidad Estructural y Contextual (IVEC)}, concebido como una respuesta a la necesidad de equilibrar el rigor conceptual con la aplicabilidad práctica. El IVEC se aleja de una mera agregación de indicadores para proponerse como un marco analítico que descompone la vulnerabilidad social en componentes interrelacionados, permitiendo una lectura más profunda y matizada de sus causas.

Para ello, el índice se articula en torno a \textbf{tres dimensiones analíticas diferenciadas}, que capturan los distintos niveles en los que la fragilidad social se manifiesta:

\begin{enumerate}
\def\labelenumi{\arabic{enumi}.}
\item
  La \textbf{vulnerabilidad individual (Vi)}, que remite a las características demográficas, económicas y de salud de las personas y los hogares. Variables como la edad avanzada, la presencia de enfermedades crónicas, la situación laboral precaria o la falta de ingresos estables inciden directamente en la capacidad individual de anticipar y resistir los impactos de un evento extremo.
\item
  La \textbf{vulnerabilidad contextual (VC)}, vinculada al entorno inmediato en el que se desenvuelven los hogares. Aquí se incluyen factores como la calidad de la vivienda, la accesibilidad a servicios básicos, la segregación barrial o la conectividad con centros de atención y recursos. Esta dimensión refleja que, incluso con características individuales similares, las condiciones del lugar de residencia pueden amplificar o atenuar los efectos de una amenaza.
\item
  La \textbf{vulnerabilidad estructural (VE)}, que se refiere a los procesos históricos y acumulativos de desigualdad socioeconómica que afectan a un territorio. Engloba fenómenos como la persistencia del desempleo estructural, la precariedad del modelo productivo, los bajos niveles educativos, la falta de inversión en infraestructuras o la debilidad de las instituciones públicas. A diferencia de las dimensiones individual y contextual, la vulnerabilidad estructural expresa tendencias de largo recorrido que configuran de manera profunda las oportunidades de los territorios.
\end{enumerate}

Ahora bien, estas dimensiones no operan de forma aislada ni determinista. Una de las aportaciones centrales del IVEC es la incorporación de un conjunto de \textbf{capacidades compensadoras} que modulan su peso relativo y permiten capturar de manera más precisa la complejidad de las trayectorias sociales. El modelo reconoce que las poblaciones no son sujetos pasivos, sino que despliegan recursos para mitigar su fragilidad. Así, se proponen:

\begin{itemize}
\tightlist
\item
  La \textbf{composición del hogar (CH)}, que actúa sobre la vulnerabilidad individual al proporcionar apoyos internos de cuidado y redistribución de recursos.
\item
  Las \textbf{capacidades sociales (CS)}, que intervienen en la vulnerabilidad contextual, en tanto expresan la fortaleza del tejido comunitario y la disponibilidad de servicios de proximidad.
\item
  Las \textbf{capacidades de adaptación (CAD)}, que se proyectan sobre la vulnerabilidad estructural, representando políticas públicas, infraestructuras y dispositivos de largo plazo que fortalecen la resiliencia social.
\item
  Finalmente, en escenarios de riesgo potencial, se incorpora la \textbf{capacidad de respuesta (CR)}, que reduce la incidencia de la vulnerabilidad contextual al considerar la eficacia de los recursos y mecanismos activados frente a emergencias.
\end{itemize}

La articulación de estas dimensiones y capacidades se traduce en dos expresiones complementarias del índice. Por un lado, el \textbf{IVEC puro}, que capta la vulnerabilidad social en su forma estructural y territorial. Por otro, el \textbf{IVEC-Riesgo}, que incorpora la capacidad de respuesta y se inserta en la fórmula clásica del riesgo, permitiendo estimar de manera más ajustada la fragilidad social efectiva frente a amenazas específicas.

\section{Justificación del caso de estudio: La Comunitat Valenciana como laboratorio territorial}\label{justificaciuxf3n-del-caso-de-estudio-la-comunitat-valenciana-como-laboratorio-territorial}

La elección de la Comunitat Valenciana como ámbito de aplicación del IVEC no responde únicamente a criterios de conveniencia, sino a su condición de \textbf{laboratorio territorial paradigmático}. La región presenta una compleja superposición de características demográficas, económicas, ambientales e institucionales que permiten observar con nitidez la interacción entre la desigualdad social y la exposición a riesgos, justificando plenamente su pertinencia como caso de estudio.

\subsection{Panorama demográfico y territorial}\label{panorama-demogruxe1fico-y-territorial}

Con una población que en 2025 supera los 5,4 millones de habitantes, la Comunitat Valenciana exhibe una estructura territorial policéntrica pero marcada por profundos desequilibrios funcionales. El territorio se organiza en torno a tres grandes áreas metropolitanas ---València, Alacant-Elx y Castelló--- que concentran cerca del 60\% de la población regional. Esta macrorregión costera se apoya en una red de ciudades intermedias que cumplen funciones comarcales. Esta marcada concentración demográfica y económica en la franja litoral contrasta con un extenso interior rural afectado por procesos históricos de despoblamiento y un acusado envejecimiento, donde más del 30\% de sus habitantes supera los 65 años.

\subsection{Transformaciones productivas y fragilidades económicas}\label{transformaciones-productivas-y-fragilidades-econuxf3micas}

Desde mediados del siglo XX, la estructura productiva valenciana ha experimentado un profundo cambio, transitando de una economía tradicionalmente agraria a un modelo diversificado donde los servicios ---particularmente el turismo--- y un tejido industrial de pymes ocupan un lugar central. Ciertas comarcas se especializaron en industrias manufactureras intensivas en mano de obra --calzado, textil, mueble, azulejo-- que fueron durante décadas motores de empleo local. Sin embargo, la globalización y la deslocalización debilitaron severamente la competitividad de muchos de estos distritos, provocando cierres de empresas y generando bolsas de desempleo estructural.

La crisis económica de 2008 intensificó estas debilidades preexistentes. En 2013, la Comunitat Valenciana alcanzó una tasa de paro del 28,3\% y su PIB sufrió una contracción superior a la media nacional. La recuperación posterior fue incompleta, dejando un legado de precarización laboral y dificultad de acceso a la vivienda. Más recientemente, la pandemia de COVID-19 en 2020 impactó con especial dureza en sectores clave como el turismo y la hostelería, evidenciando de nuevo la fragilidad de un modelo económico dependiente de actividades estacionales.

\subsection{Desigualdades sociales y espaciales}\label{desigualdades-sociales-y-espaciales}

Los indicadores sociales actuales reflejan una realidad territorial profundamente desigual. En 2024, la tasa AROPE (Riesgo de Pobreza o Exclusión Social) en la Comunitat Valenciana alcanzó el 29,9\% de la población, afectando a 1,5 millones de personas, una cifra sensiblemente superior a la media estatal del 25,8\%. Esta vulnerabilidad no se distribuye de forma homogénea, sino que adopta configuraciones espaciales muy marcadas. En las áreas urbanas, múltiples estudios han documentado la concentración de desventajas en barrios periféricos de València (Natzaret, Orriols), Alicante (Virgen del Remedio) o Castelló (San Lorenzo), donde convergen déficits de equipamientos, degradación residencial y empleo precario. En paralelo, las comarcas rurales del interior enfrentan un ``reto demográfico'' asociado al despoblamiento y la pérdida de servicios básicos, generando un círculo vicioso que refuerza el abandono económico y social.

\subsection{Riesgos ambientales y marco institucional}\label{riesgos-ambientales-y-marco-institucional}

Finalmente, la dimensión ambiental añade una capa de complejidad fundamental.Aproximadamente el 15\% del territorio autonómico está clasificado como zona de riesgo potencial de inundación según el PATRICOVA \citep{gvapatricova}, afectando a más de medio millón de personas. Esta exposición se ve agravada por una histórica ocupación urbanística de áreas inundables \citep{olcina2018_riesgos} y por los impactos previstos del cambio climático, que auguran una intensificación de olas de calor, tensiones hídricas y un ascenso del nivel del mar \citep{copernicus2023, medecc2020_mar1}. Estos riesgos no operan en el vacío, sino que se amplifican en territorios socialmente frágiles. A nivel institucional, aunque la Generalitat Valenciana ejerce competencias clave en protección civil y servicios sociales, su capacidad de acción se ve a menudo limitada por restricciones financieras y de coordinación intergubernamental, dificultando una gobernanza integrada del riesgo.

\section{Planteamiento del problema de investigación}\label{planteamiento-del-problema-de-investigaciuxf3n}

El recorrido conceptual y territorial realizado hasta ahora permite señalar una paradoja central que constituye el núcleo de esta investigación. Mientras la literatura académica y los marcos internacionales han avanzado hacia una concepción compleja e integrada de la vulnerabilidad, en la práctica institucional persiste una \textbf{profunda disociación} entre el análisis de las amenazas físicas y la consideración de las condiciones sociales que determinan la capacidad de respuesta de las comunidades. Esta brecha no es meramente teórica, sino que tiene consecuencias directas y, a menudo, trágicas.

Esta desconexión resulta especialmente visible en la Comunitat Valenciana. Los principales instrumentos autonómicos de planificación del riesgo ---como el Plan de Acción Territorial frente a Inundaciones (PATRICOVA) o el Plan Especial frente a Incendios Forestales (PEIF)--- ofrecen cartografías muy precisas de los peligros físicos. Sin embargo, en lo que respecta a la dimensión social, apenas incluyen indicadores más allá de algunas variables demográficas agregadas, como la densidad de población. Se trata de un enfoque que, aunque técnicamente riguroso en el plano físico, resulta \textbf{reduccionista en términos sociológicos}: la vulnerabilidad se diluye bajo parámetros demográficos o espaciales sin capturar las desigualdades estructurales y contextuales que explican la fragilidad diferencial de los territorios.

La \textbf{DANA de octubre de 2024} reveló con una crudeza incontestable la fragilidad de este enfoque centrado solo en la amenaza. La magnitud de la catástrofe no se explica únicamente por la intensidad del fenómeno meteorológico. Los análisis posteriores al evento evidenciaron que las variables que más determinaron la elevada mortalidad y la distribución de los daños fueron de naturaleza eminentemente social: la \textbf{precariedad habitacional} en zonas inundables, la \textbf{inseguridad laboral} que impidió a muchos abandonar sus puestos de trabajo, el \textbf{aislamiento} de personas mayores o dependientes, y la \textbf{debilidad de las redes comunitarias} en barrios de reciente conformación. Ninguno de estos elementos, que son el corazón de la vulnerabilidad social, figuraba en los mapas de riesgo ni en los protocolos de alerta temprana de la Generalitat \citep{gva2024informedana}.

Este caso local ilustra una problemática global: la distancia entre los compromisos normativos de integrar la vulnerabilidad en la gestión del riesgo y la realidad operativa, donde dicha integración no se ha materializado plenamente. A nivel científico, y como se desarrollará en el capítulo 2, la vulnerabilidad sigue siendo un concepto polisémico y escasamente operacionalizado en la práctica \citep{rufat2023_criticalreview}. Los enfoques existentes suelen fragmentarse en tradiciones disciplinares, emplear escalas de análisis heterogéneas y apoyarse en indicadores parciales, lo que dificulta la comparabilidad de resultados y su aplicación territorial efectiva.

Esta dispersión se traduce en lo que hemos presenciado: aproximaciones de gestión del riesgo fragmentarias y, en gran medida, \textbf{reactivas}. Predominan diagnósticos centrados en la amenaza física, sin incorporar de forma sistemática las condiciones subyacentes que producen y reproducen la vulnerabilidad. Para superar estas limitaciones, la presente tesis plantea la construcción de un marco analítico integrado que articule los enfoques complementarios de PAR, MOVE y Cardona. Desde esta óptica compuesta, el riesgo deja de concebirse como un mero cálculo probabilístico de daños esperados, para entenderse como una propiedad emergente de sistemas socio-territoriales dinámicos.

\section{Pregunta de investigación, hipótesis y objetivos}\label{pregunta-de-investigaciuxf3n-hipuxf3tesis-y-objetivos}

La cuestión central que articula esta tesis puede formularse de la siguiente manera:

\emph{¿Es posible construir y aplicar un índice sintético de vulnerabilidad social ---territorialmente desagregado, teóricamente fundamentado y empíricamente replicable--- que permita identificar patrones diferenciados de riesgo en la Comunitat Valenciana y contribuya a una gobernanza del riesgo más anticipadora, equitativa y eficaz?}

Para responder a esta pregunta, la investigación se apoya en dos hipótesis complementarias, una de carácter teórico-epistemológico y otra de naturaleza empírica.

\subsection{Hipótesis sobre la configuración del campo científico}\label{hipuxf3tesis-sobre-la-configuraciuxf3n-del-campo-cientuxedfico}

La primera hipótesis aborda la propia estructura del conocimiento sobre la vulnerabilidad. A pesar de la aparente fragmentación conceptual y metodológica que se observa en la literatura, se postula que:

\begin{quote}
\emph{A pesar de la fragmentación inicial del campo de estudio de la vulnerabilidad, la producción científica de las últimas décadas muestra indicios de una progresiva convergencia parcial e hibridación interdisciplinaria.}
\end{quote}

Esta hipótesis será contrastada en el capítulo 2 mediante un \textbf{análisis bibliométrico} de la literatura internacional, con el fin de identificar si, más allá de la diversidad de enfoques, existen patrones de integración conceptual y metodológica que justifiquen la propuesta de un marco sintético como el IVEC.

\subsection{Hipótesis de validación empírica}\label{hipuxf3tesis-de-validaciuxf3n-empuxedrica}

La segunda hipótesis busca validar la capacidad explicativa del índice propuesto en un escenario real de desastre. Para ello, se formula que:

\begin{quote}
\emph{Las unidades territoriales con valores elevados del IVEC coinciden en mayor medida con los territorios más negativamente afectados por la DANA de octubre de 2024 en la Comunitat Valenciana.}
\end{quote}

Es fundamental realizar una matización metodológica en este punto. El cálculo del IVEC se realiza con datos administrativos correspondientes a \textbf{abril de 2025}, mientras que el evento de validación, la DANA, ocurrió en \textbf{octubre de 2024}. Lejos de ser una inconsistencia, esta decisión se fundamenta en la propia naturaleza del índice. El IVEC está diseñado para medir \textbf{vulnerabilidades estructurales y contextuales}, es decir, procesos sociales de cambio lento y gran inercia, como las desigualdades socioeconómicas, las pautas demográficas o la segregación residencial. Estas condiciones no experimentan fluctuaciones significativas en un lapso de seis meses.

Por tanto, se postula que los datos de abril de 2025 constituyen un \textbf{proxy de alta fidelidad} de las condiciones de vulnerabilidad que ya estaban presentes en los territorios antes del desastre. El IVEC no mide el \emph{impacto} inmediato de la DANA, sino la \textbf{condición de fragilidad preexistente} que moduló dicho impacto. Esta aproximación es metodológicamente sólida y coherente con la sociología del riesgo, que entiende los desastres como la materialización de vulnerabilidades acumuladas en el tiempo.

\subsection{Objetivos de la investigación}\label{objetivos-de-la-investigaciuxf3n}

Para responder a la pregunta central y contrastar estas hipótesis, la tesis se marca un itinerario de objetivos que se desarrollarán secuencialmente a lo largo de los capítulos.

En el plano \textbf{teórico y normativo} (Capítulos 2, 3 y 4), los objetivos son:
- Clarificar el concepto de vulnerabilidad social desde una perspectiva interdisciplinar.
- Realizar un análisis bibliométrico para contrastar la hipótesis sobre la configuración del campo científico.
- Revisar la integración de la vulnerabilidad en los marcos normativos de diferentes escalas.

Desde el punto de vista \textbf{metodológico} (Capítulo 5), la tesis persigue los siguientes objetivos:
- Diseñar y fundamentar el IVEC como un índice metodológicamente robusto.
- Definir sus dimensiones, variables e indicadores, y establecer un marco reproducible y comparable.

En el bloque \textbf{empírico y de análisis de resultados} (Capítulo 6), los objetivos son:
- Implementar el IVEC en la malla de 1 km² para toda la Comunitat Valenciana.
- Cartografiar e interpretar los patrones espaciales de vulnerabilidad resultantes.
- Simular escenarios de riesgo para explorar el comportamiento del índice.
- Contrastar la hipótesis de validación empírica a partir de la DANA de 2024.

Finalmente, en la fase de \textbf{discusión y conclusiones} (Capítulos 7 y 8), los objetivos son:
- Integrar los hallazgos empíricos con los marcos conceptuales y normativos.
- Evaluar críticamente los resultados y las limitaciones de la investigación.
- Formular recomendaciones estratégicas para una gobernanza del riesgo más justa y transformadora.

\section{Estructura y síntesis recapituladora de la tesis}\label{estructura-y-suxedntesis-recapituladora-de-la-tesis}

La introducción ha trazado el itinerario que fundamenta la investigación. Se ha mostrado que los desastres son procesos sociales mediados por desigualdades estructurales y territoriales. Se ha revisado la evolución del concepto de vulnerabilidad, los modelos teóricos de referencia y los debates críticos que acompañan su uso. Sobre esa base se ha presentado el IVEC como propuesta original, se ha justificado la elección de la Comunitat Valenciana como caso de estudio y se ha anclado el problema en la evidencia empírica de la DANA de 2024. Finalmente, se ha formulado la pregunta de investigación junto con las hipótesis y los objetivos que guiarán el resto del trabajo.

De este modo, el itinerario de la tesis queda claramente definido, siguiendo una estructura lógica que avanza desde la fundamentación teórica hasta la validación empírica y la discusión final. El trabajo se adentra ahora en su primer gran bloque de fundamentación, dedicado a establecer las bases conceptuales, científicas e institucionales de la investigación. A lo largo de los \textbf{capítulos 2, 3 y 4}, se profundizará en el andamiaje que sostiene esta tesis: se realizará un exhaustivo estado de la cuestión, se desarrollará el marco teórico integrado y se analizará el marco normativo que regula la gestión del riesgo.

Una vez establecidos estos cimientos, el \textbf{capítulo 5} detallará la arquitectura metodológica de la investigación, explicando el diseño, la construcción y el proceso de validación del IVEC. A continuación, el \textbf{capítulo 6} presentará la aplicación empírica del índice en la Comunitat Valenciana, exponiendo los resultados y los patrones espaciales de vulnerabilidad identificados. Finalmente, los \textbf{capítulos 7 y 8} conformarán el bloque de cierre, donde se discutirán los hallazgos en diálogo con la teoría, se sintetizarán las principales conclusiones y se formularán recomendaciones para una gobernanza del riesgo más justa y transformadora. Este recorrido, en su conjunto, busca no solo responder a las preguntas de investigación planteadas, sino también reforzar la coherencia académica y la pertinencia social de la tesis.

\cleardoublepage
\addtocontents{toc}{\protect\vspace{1.2\baselineskip}}
\addtocontents{toc}{\protect\contentsline{chapter}
  {\centering\textbf{\large Capítulo~2}}{}{}}
\thispagestyle{plain}

\chapter{Cartografía conceptual, tensiones y evolución del campo. Estado de la cuestión}\label{cartografuxeda-conceptual-tensiones-y-evoluciuxf3n-del-campo.-estado-de-la-cuestiuxf3n}

\section{Introducción}\label{introducciuxf3n-1}

En las últimas décadas, la noción de vulnerabilidad ha adquirido una centralidad ineludible en las ciencias sociales y en los estudios territoriales. Bajo esta etiqueta se agrupan análisis sobre desigualdad, pobreza, exclusión, riesgo de desastres, cambio climático, salud pública, envejecimiento, precariedad habitacional o crisis de cuidados. Lejos de constituir un concepto unívoco, la vulnerabilidad se ha convertido en un punto de cruce donde confluyen tradiciones teóricas diversas, agendas institucionales heterogéneas y dispositivos técnicos de evaluación cada vez más sofisticados. Esta expansión semántica ha permitido iluminar dimensiones antes invisibilizadas del daño social; pero también ha generado ambigüedades, solapamientos y usos estratégicos que dificultan delimitar con precisión qué se está nombrando, cómo se explica y con qué fines se moviliza el término.

En este contexto, el presente capítulo se propone ordenar críticamente el estado de la cuestión en torno a la vulnerabilidad, no mediante una mera acumulación de definiciones, sino a través de una arquitectura analítica que permita situar las distintas contribuciones en un marco coherente. La premisa de partida es que buena parte de las controversias que atraviesan el campo ---sobre si la vulnerabilidad es universal o diferencial, estructural o individual, social o ecológica, técnica o política--- remiten, en el fondo, a desacuerdos sobre cuatro planos analíticos: qué se entiende que existe cuando se habla de vulnerabilidad (plano ontológico), qué obligaciones y prioridades se le asocian (plano normativo), cómo se considera legítimo conocerla y explicarla (plano epistemológico--teórico) y cómo se traduce ese conocimiento en instrumentos concretos de análisis (plano metodológico).

La primera parte del capítulo adopta, por tanto, una perspectiva meta-analítica: en lugar de privilegiar una definición de vulnerabilidad frente a otras, se reconstruyen los principales modos de existencia que se le atribuyen en la literatura (vulnerabilidad constitutiva, producida socialmente, socioecológica, tecnocrática, disposicional-clínica), los horizontes ético--políticos que orientan su uso (justicia social, derechos, equidad territorial, resiliencia, adaptación), las epistemologías que sostienen su producción de conocimiento (empirista-instrumental, crítica, interpretativa, sistémico--ecológica, integradora) y las familias metodológicas que la operativizan (índices e indicadores, análisis estructurales, enfoques participativos, modelizaciones socioecológicas, metodologías mixtas). Este recorrido no busca clausurar el debate, sino hacer explícita su arquitectura interna y mostrar cómo ciertas tensiones ---entre universalidad y desigualdad, entre técnica y política, entre estructura y experiencia, entre lo social y lo ambiental--- reaparecen de forma recurrente en distintos niveles de análisis.

Sobre esta base conceptual se formula la primera hipótesis de la tesis, de carácter transversal al conjunto del capítulo: la hipótesis de que el campo contemporáneo de estudios sobre vulnerabilidad se encuentra inmerso en un proceso de convergencia transdisciplinar. Esta convergencia no implica la desaparición de las diferencias entre disciplinas o corrientes, sino el despliegue progresivo de lenguajes analíticos compartidos (por ejemplo, en torno al trinomio exposición--sensibilidad--capacidad), de marcos explicativos híbridos (que combinan aportes estructurales, clínicos y socioecológicos) y de repertorios metodológicos que cruzan fronteras disciplinares (índices cuantitativos integrados con lecturas cualitativas, análisis espaciales alimentados por marcos críticos, etc.). La hipótesis sostiene, en suma, que el campo, pese a su origen fragmentario, tiende a organizarse en torno a núcleos conceptuales y prácticos cada vez más interconectados.

Esta proposición no puede sostenerse únicamente mediante argumentos teóricos o impresiones de lectura. Requiere ser sometida a una prueba empírica específica: es necesario examinar cómo se relacionan entre sí las publicaciones, qué redes de coautoría y citación se configuran, qué conceptos co--ocurren en los mismos trabajos, cómo evolucionan los temas a lo largo del tiempo y hasta qué punto disciplinas que antes operaban en paralelo comienzan a solaparse. Por este motivo, el capítulo no se limita a una revisión narrativa, sino que culmina con un análisis bibliométrico y semántico diseñado precisamente para validar o refutar la hipótesis de convergencia transdisciplinar. El análisis bibliométrico permitirá identificar estructuras de colaboración, núcleos de referencia y proximidades disciplinares; el análisis semántico, por su parte, ayudará a cartografiar la evolución de las palabras clave y de los clústeres temáticos, revelando afinidades, desplazamientos y aproximaciones entre campos conceptualmente distantes.

La estructura del capítulo refleja esta lógica de refinamiento progresivo. En la sección siguiente se desarrolla el plano ontológico, donde se examinan los distintos modos en que la vulnerabilidad es concebida en la literatura: como fragilidad constitutiva de la vida humana, como resultado de procesos estructurales de desigualdad, como propiedad emergente de sistemas socioecológicos acoplados, como magnitud tecnificable o como susceptibilidad diferencial inscrita en cuerpos y biografías concretas. Esta clarificación ontológica permite comprender por qué determinadas líneas de investigación ponen el acento en los cuerpos, otras en las estructuras, otras en los territorios y otras en los ensamblajes entre todos ellos.

A continuación, se aborda el plano normativo, en el que se analizan los marcos ético--políticos que han ido configurando el uso público de la vulnerabilidad: desde las agendas de reducción del riesgo de desastres y adaptación climática (Marco de Hyogo, Marco de Sendai) hasta las políticas europeas, estatales y autonómicas en materia de cohesión social, derechos de ciudadanía, servicios sociales y equidad territorial. Este plano muestra que la vulnerabilidad no es solo una categoría descriptiva, sino también un dispositivo normativo que legitima determinadas intervenciones, visibiliza ciertos daños y puede, en ocasiones, neutralizar su potencial crítico si se reduce a etiqueta administrativa.

Sobre estas bases se despliega el plano epistemológico--teórico, que reconstruye las principales formas en que se ha tratado de conocer y explicar la vulnerabilidad: epistemologías empirista--instrumentales centradas en la medición comparativa; enfoques estructurales y relacionales que priorizan la reconstrucción de las causas profundas del riesgo; perspectivas interpretativas y comunitarias que atienden a las experiencias y significados locales; marcos sistémico--socioecológicos que analizan la vulnerabilidad como resultado de interdependencias complejas entre sistemas humanos y naturales; y aproximaciones híbridas que tratan de articular estos registros. Este plano actúa como bisagra entre la ontología y la metodología, mostrando cómo ciertas concepciones de lo que la vulnerabilidad ``es'' orientan, casi inevitablemente, lo que se considera un conocimiento válido sobre ella.

Por último, el plano metodológico revisa las familias de herramientas que han dado forma empírica a estas discusiones: índices e indicadores sintéticos, análisis estructurales de desigualdad, estudios cualitativos y participativos, modelizaciones espaciales y socioecológicas, y enfoques mixtos que combinan varios de estos registros. Esta sección no pretende ofrecer un manual exhaustivo de técnicas, sino destacar cómo cada elección metodológica implica una toma de posición respecto a los planos ontológico, normativo y epistemológico previamente expuestos.

Solo después de completar este recorrido ---ontológico, normativo, epistemológico--teórico y metodológico--- el capítulo está en condiciones de presentar con rigor la primera hipótesis y de justificar la pertinencia del análisis bibliométrico y semántico como estrategia de contrastación empírica. De este modo, el estado de la cuestión no se reduce a un inventario de autores y definiciones, sino que se convierte en un dispositivo de clarificación conceptual y reordenación crítica del campo, que permitirá, en los capítulos posteriores, situar la propuesta del IVEC y los análisis territoriales de la vulnerabilidad en el marco de una convergencia transdisciplinar cuidadosamente evaluada.

\section{El plano ontológico: qué ``es'' la vulnerabilidad}\label{el-plano-ontoluxf3gico-quuxe9-es-la-vulnerabilidad}

Antes de discutir cómo se explica, cómo se conoce o cómo se mide la vulnerabilidad, es necesario aclarar \textbf{qué se entiende que existe} cuando se recurre a este término. El plano ontológico remite precisamente a esta pregunta: no se ocupa aún de la validez de los datos ni de las técnicas empleadas, sino de la naturaleza del objeto al que se refiere el análisis.

En la literatura contemporánea, la vulnerabilidad no aparece como una entidad unívoca, sino como un \textbf{campo de problemas} en el que conviven varios modos de concebir qué es lo vulnerable y dónde se localiza. Las discrepancias sobre si la vulnerabilidad es universal o diferencial, individual o estructural, social o socioecológica, son, en gran medida, discrepancias ontológicas.

En esta tesis se distinguen cinco configuraciones que permiten ordenar ese debate: la \textbf{vulnerabilidad constitutiva}, que remite a la fragilidad inherente a la vida humana; la \textbf{vulnerabilidad disposicional-clínica}, que introduce gradientes de susceptibilidad diferencial en cuerpos y biografías; la \textbf{vulnerabilidad estructural-relacional}, que subraya la producción social e histórica de la exposición al daño; la \textbf{vulnerabilidad tecnocrático-instrumental}, que la traduce en magnitud medible y gobernable; y la \textbf{vulnerabilidad funcional-sistémica}, que la concibe como propiedad emergente de sistemas socioecológicos acoplados. Cada una ilumina ciertos aspectos del fenómeno y, al mismo tiempo, deja otros en segundo plano; sus aportaciones deben entenderse tanto por lo que permiten ver como por aquello que tienden a invisibilizar.

\subsection{Vulnerabilidad constitutiva: fragilidad universal de la vida humana}\label{vulnerabilidad-constitutiva-fragilidad-universal-de-la-vida-humana}

La perspectiva constitutiva parte de una afirmación fuerte: \textbf{toda vida humana es vulnerable}. La vulnerabilidad no se deriva aquí de una situación particular ni de una carencia localizada, sino del hecho mismo de existir como cuerpos finitos, expuestos y dependientes.

Autoras como \textbf{Martha Fineman} y \textbf{Judith Butler} han formulado con especial claridad esta intuición. Fineman propone la figura del \emph{sujeto vulnerable} como alternativa crítica al sujeto liberal autosuficiente: los individuos no son entidades plenamente autónomas, sino seres cuya supervivencia cotidiana descansa sobre redes de cuidado, infraestructuras públicas y provisiones colectivas \citep{fineman2008}. Butler, por su parte, desarrolla una ontología social de los cuerpos en la que la exposición a la herida ---física, simbólica, afectiva--- constituye simultáneamente un foco de riesgo y la base de la interdependencia, el vínculo y la responsabilidad mutua \citep{butler2009frames}.

Desde este enfoque, la vulnerabilidad \textbf{no equivale a debilidad} ni a fallo individual. Es una condición estructural de la existencia encarnada. Su relevancia normativa es inmediata: si toda vida es vulnerable, la obligación de sostenerla deja de ser un asunto caritativo u ocasional para convertirse en principio organizador de las instituciones y las políticas públicas.

La vulnerabilidad constitutiva proporciona así un \textbf{suelo ético universal} desde el que interrogar la distribución desigual de la protección y del daño. Pero esa misma universalización delimita su principal límite: al insistir en lo que todas las vidas comparten, esta ontología corre el riesgo de \textbf{difuminar las diferencias materiales, históricas y de poder} que hacen que algunas personas y territorios estén sistemáticamente más expuestos, dispongan de menos recursos de defensa y sean considerados menos ``protegibles''. De ahí la necesidad de complementar este plano con otros que introducen la dimensión diferencial.

\subsection{Vulnerabilidad disposicional-clínica: susceptibilidad diferencial inscrita en cuerpos y biografías}\label{vulnerabilidad-disposicional-cluxednica-susceptibilidad-diferencial-inscrita-en-cuerpos-y-biografuxedas}

Si la vulnerabilidad constitutiva marca un horizonte compartido, la ontología disposicional-clínica introduce una distinción clave: \textbf{no todos los cuerpos son igualmente susceptibles al daño}, ni todas las trayectorias de vida acumulan la misma carga de fragilidad.

En este plano, la vulnerabilidad se entiende como una \textbf{disposición diferenciada}, una mayor o menor probabilidad de que ciertos eventos ---enfermedad, accidente, aislamiento, pérdida--- se traduzcan en daño efectivo. Dicha disposición se ancla en configuraciones corporales y psicosociales concretas: edad avanzada, multimorbilidad, discapacidad, deterioro cognitivo, trastornos de salud mental, soledad prolongada, entre otras.

Este enfoque se ha desarrollado sobre todo en la literatura de salud pública, gerontología y salud comunitaria, donde se habla de ``grupos de riesgo'' o de ``personas especialmente vulnerables''. Ontológicamente, lo que se afirma no es solo que todos los cuerpos son vulnerables (plano constitutivo), sino que determinados conjuntos de características \textbf{incrementan la susceptibilidad} ante perturbaciones relativamente similares.

La aportación de esta ontología reside en que actúa como \textbf{bisagra} entre la vulnerabilidad universal y las desigualdades estructurales. Traduce la fragilidad compartida en gradientes de susceptibilidad individual, pero al mismo tiempo remite inevitablemente a los contextos donde esa susceptibilidad se actualiza. Un mismo nivel de fragilidad clínica puede tener consecuencias muy distintas según las condiciones de vivienda, la existencia de redes de apoyo, la accesibilidad de los servicios o el tipo de barrio en el que se reside.

Su límite se hace visible cuando se aísla del entorno social y territorial. Si se reifica como ontología puramente biomédica, corre el riesgo de \textbf{individualizar} la vulnerabilidad, desplazando la atención desde las estructuras que la producen o la agravan hacia los cuerpos que la soportan. Por ello, en esta tesis la vulnerabilidad disposicional-clínica se entiende como un pliegue específico del plano constitutivo, pero en diálogo necesario con las perspectivas estructural y sistémica que se describen a continuación.

\subsection{Vulnerabilidad estructural-relacional: producción social de la exposición y del daño}\label{vulnerabilidad-estructural-relacional-producciuxf3n-social-de-la-exposiciuxf3n-y-del-dauxf1o}

La ontología estructural-relacional desplaza el foco desde los cuerpos hacia las \textbf{relaciones sociales, las posiciones estructurales y las trayectorias históricas} que distribuyen de manera desigual la exposición al riesgo y la capacidad de respuesta. Aquí, la vulnerabilidad no es una condición universal ni un rasgo individual, sino un \textbf{resultado históricamente producido} de procesos de desigualdad, precarización y desprotección institucional.

Aportaciones como las de \textbf{Ulrich Beck} sobre la sociedad del riesgo \citep{beck1992}, los análisis de capital y dominación simbólica de \textbf{Pierre Bourdieu} \citep{bourdieu1986} o la noción de ``zonas de desafilación'' de \textbf{Robert Castel} \citep{castel1995} coinciden en señalar que los riesgos contemporáneos, lejos de distribuirse de forma homogénea, tienden a concentrarse allí donde se acumulan desventajas de clase, género, origen étnico o estatus migratorio.

En el ámbito específico de los estudios de desastres, el marco \textbf{Pressure and Release (PAR)} de Blaikie y colaboradores \citep{blaikie1994} traduce esta ontología en cadenas causales. La vulnerabilidad emerge del encadenamiento de \textbf{causas profundas} (estructuras económicas, políticas, ideológicas), \textbf{presiones dinámicas} (crecimiento urbano desregulado, recorte de derechos, degradación ambiental) y \textbf{condiciones inseguras} (barrios expuestos, viviendas precarias, ausencia de servicios). El desastre aparece así como la culminación de procesos previos, no como un evento externo que golpea de manera indiscriminada.

Ontológicamente, la vulnerabilidad se concibe aquí como \textbf{condensación de relaciones de poder} en el espacio y en el tiempo. No es algo que ``les ocurre'' a ciertos grupos por lo que son, sino algo que se \textbf{produce} a través de decisiones políticas, estrategias de acumulación económica, modelos de desarrollo territorial y arreglos institucionales que definen quién queda protegido, quién queda expuesto y quién permanece invisible.

Esta lectura relacional resulta indispensable para recuperar la dimensión crítica del concepto y vincular la vulnerabilidad con procesos de injusticia social, colonialidad del poder, desigualdades de género o segregación urbana. Su principal límite aparece cuando, al privilegiar estructuras y trayectorias largas, se tiende a \textbf{subestimar la agencia de los actores y las dinámicas ecológicas} que también condicionan la vulnerabilidad. Las desigualdades estructurales no se despliegan en un vacío biofísico: se materializan en territorios concretos, atravesados por dinámicas ambientales que interactúan con las relaciones sociales. Este desplazamiento abre el camino hacia la ontología funcional-sistémica.

\subsection{Vulnerabilidad tecnocrático-instrumental: la cifra como forma de existencia pública}\label{vulnerabilidad-tecnocruxe1tico-instrumental-la-cifra-como-forma-de-existencia-puxfablica}

Una cuarta configuración ontológica se ha consolidado de la mano de la expansión de sistemas de indicadores, índices sintéticos y herramientas de diagnóstico territorial. Desde esta perspectiva, la vulnerabilidad \textbf{adquiere existencia pública en la medida en que puede ser medida, comparada y cartografiada}.

Instrumentos como el Social Vulnerability Index (SoVI) de Cutter y colaboradores \citep{cutter2003}, los atlas de vulnerabilidad urbana o los visores estadísticos territoriales traducen realidades sociales heterogéneas en escalas numéricas y mapas de intensidades desiguales. Lo que el índice incorpora pasa a ser visible y gobernable; lo que queda fuera de la batería de indicadores tiende a diluirse en la periferia de la decisión pública.

Ontológicamente, esta mirada desplaza el énfasis desde la vulnerabilidad como proceso hacia la vulnerabilidad como \textbf{magnitud cuantificable}. La vulnerabilidad se ``objetiva'' como posición relativa en un ranking, como valor continuo en una escala o como gradiente espacial en un mapa. La cifra no solo describe: también contribuye a definir qué cuenta como problema y qué queda fuera del campo de visión institucional.

La potencia de este plano es evidente. Permite \textbf{estandarizar} diagnósticos, comparar territorios, priorizar recursos, construir series temporales y justificar intervenciones. Pero su uso acrítico comporta riesgos significativos. Cuando la vulnerabilidad se identifica sin matices con el valor de un índice, se corre el peligro de \textbf{naturalizar decisiones técnicas contingentes} (qué variables se seleccionan, cómo se ponderan, qué umbrales se fijan), de confundir vulnerabilidad con pobreza medida de forma estrecha o de perder de vista las genealogías históricas y relacionales que han producido las situaciones diagnosticadas.

En esta tesis, la ontología tecnocrático-instrumental se reconoce como necesaria ---sin indicadores no hay política territorial viable---, pero se inscribe deliberadamente en diálogo con los planos estructural, socioecológico y normativo que se desarrollan en las secciones siguientes. La medición se concibe como un momento, no como el lugar definitivo donde se agota el sentido de la vulnerabilidad.

\subsection{Vulnerabilidad funcional-sistémica: propiedad emergente de sistemas socioecológicos}\label{vulnerabilidad-funcional-sistuxe9mica-propiedad-emergente-de-sistemas-socioecoluxf3gicos}

Los debates sobre cambio climático, desastres, adaptación y resiliencia han favorecido la consolidación de una ontología que entiende la vulnerabilidad como \textbf{propiedad emergente de sistemas en los que interactúan componentes sociales, económicos, institucionales y ecológicos}.

Autores como Turner \citep{turner2003} o Adger \citep{adger2006}, entre otros, proponen marcos en los que la vulnerabilidad se define como función de la exposición, la sensibilidad y la capacidad de respuesta de sistemas socioecológicos multiescalares. En estos enfoques, los individuos y los grupos siguen siendo relevantes, pero la unidad analítica fundamental pasa a ser el \textbf{ensamblaje socioecológico}: cuencas hidrográficas urbanizadas, sistemas agrícolas dependientes de recursos hídricos finitos, áreas litorales sometidas a presiones turísticas, entornos urbanos atravesados por islas de calor, entre otros.

La vulnerabilidad no reside ya únicamente en ``quién es pobre'' o ``quién está enfermo'', sino en cómo se organizan los usos del suelo, las infraestructuras, las instituciones y los patrones de consumo que, en interacción con dinámicas ambientales, generan configuraciones de riesgo. El énfasis se desplaza hacia los \textbf{procesos}: cambios de uso del suelo, transformaciones urbanas, intensificación agrícola, degradación de ecosistemas, alteración de ciclos hidrológicos.

Esta ontología aporta al menos dos elementos decisivos. En primer lugar, rompe de forma explícita la dicotomía entre ``factores sociales'' y ``amenazas naturales'': el riesgo y la vulnerabilidad se conciben como \textbf{co--producciones} inseparables de lo social y lo ecológico. En segundo lugar, introduce la idea de \textbf{no linealidad y umbral}: los sistemas pueden absorber perturbaciones hasta cierto punto, pero, una vez rebasados determinados límites, pequeñas variaciones pueden desencadenar transformaciones abruptas, degradaciones aceleradas o colapsos parciales.

Su posible limitación aparece cuando, al privilegiar el lenguaje sistémico y la modelización, se diluye la dimensión política de la vulnerabilidad en categorías de ``ajuste'', ``optimización'' o ``gestión adaptativa''. En ese caso, se corre el riesgo de perder capacidad para nombrar relaciones de poder, responsabilidades históricas o injusticias estructurales.

En el marco de esta tesis, el enfoque funcional-sistémico se toma como una pieza necesaria, pero no suficiente. Su aporte se hará especialmente visible cuando, en capítulos posteriores, se articule con las ontologías constitutivas, disposicionales y estructurales en la construcción del marco teórico operativo del IVEC y en el análisis de la vulnerabilidad territorial en la Comunitat Valenciana.

Genial, vamos con el \textbf{plano normativo}, manteniendo el tono y el nivel que llevamos y sin abusar de párrafos clonados en longitud.

Lo dejo en formato fácilmente pegable en tu Rmd (ajustas numeración si hace falta).

\section{El plano normativo: qué debe protegerse y con qué criterios}\label{el-plano-normativo-quuxe9-debe-protegerse-y-con-quuxe9-criterios}

Si el plano ontológico se pregunta \textbf{qué es} la vulnerabilidad, el plano normativo se interroga por \textbf{lo que debe hacerse} frente a ella: qué vidas, qué territorios y qué riesgos se consideran prioritarios, qué niveles de daño se juzgan aceptables, qué obligaciones se atribuyen al Estado y qué responsabilidades a los individuos, las familias o las comunidades. No se trata únicamente de un añadido ético a posteriori; las opciones normativas \textbf{preconfiguran el propio objeto de estudio}, porque definen qué situaciones se nombran como vulnerabilidad, cuáles se invisibilizan y qué formas de respuesta se consideran legítimas.

En la literatura y en las políticas contemporáneas pueden identificarse varias orientaciones normativas recurrentes. En este capítulo se destacan cuatro, estrechamente vinculadas a las ontologías previamente descritas: la vulnerabilidad como base de derechos universales y deber de cuidado; la vulnerabilidad como criterio de focalización y prioridad; la vulnerabilidad como cuestión de justicia estructural; y la vulnerabilidad como problema de sostenibilidad y justicia socioecológica. Estas orientaciones no aparecen de forma aislada, pero su distinción analítica permite clarificar los supuestos que orientan el diseño de instrumentos como el IVEC.

\subsection{Vulnerabilidad y derechos universales: la obligación de sostener vidas vulnerables}\label{vulnerabilidad-y-derechos-universales-la-obligaciuxf3n-de-sostener-vidas-vulnerables}

La perspectiva que más directamente dialoga con la ontología constitutiva concibe la vulnerabilidad como \textbf{fundamento de derechos universales} y de un deber generalizado de cuidado. Si todos los seres humanos son vulnerables por el mero hecho de existir como cuerpos finitos y dependientes, entonces la protección frente al daño no puede quedar condicionada a la pertenencia a un grupo específico ni a la demostración de una ``carencia extrema'', sino que debe organizarse como \textbf{garantía básica}.

Fineman formula esta idea en términos de \emph{Estado responsivo}: un Estado que asume la vulnerabilidad universal como punto de partida y orienta sus instituciones a sostener, de manera continua, las interdependencias que hacen posible la vida social \citep{fineman2008}. Butler, desde otro registro, insiste en que la vulnerabilidad compartida obliga a repensar los marcos de duelo y de reconocimiento: no todas las vidas son lloradas ni defendidas del mismo modo, y la tarea normativa consiste en ampliar el círculo de aquellas que cuentan como dignas de protección \citep{butler2009frames}.

En el plano jurídico-político, esta orientación se traduce en dispositivos de protección social \textbf{de base universal}: sistemas públicos de salud y de educación, redes de servicios sociales que reconocen derechos subjetivos de atención, prestaciones económicas desvinculadas de la estricta lógica contributiva. En el contexto valenciano, la Ley de Servicios Sociales Inclusivos \citep{ley3-2019ServiciosSocialesInclusivos} se inscribe en esta tradición al reconocer la vulnerabilidad como condición inherente a la vida en sociedad y consagrar un catálogo de derechos subjetivos de acceso a servicios y prestaciones sin necesidad de acreditar situaciones excepcionales.

El valor de esta perspectiva radica en que desplaza la vulnerabilidad del campo de la excepcionalidad al de la \textbf{normalidad social}. No obstante, su propio énfasis universalista puede dificultar la identificación de diferencias internas: sin una articulación con enfoques que introduzcan la dimensión estructural y territorial, corre el riesgo de formular derechos abstractos que no se traducen en cambios efectivos allí donde los daños son más graves y persistentes.

\subsection{Vulnerabilidad como criterio de focalización: priorizar a ``grupos vulnerables''}\label{vulnerabilidad-como-criterio-de-focalizaciuxf3n-priorizar-a-grupos-vulnerables}

En un registro distinto, pero a menudo entrelazado con el anterior, se sitúan las aproximaciones normativas que conciben la vulnerabilidad como \textbf{criterio de prioridad}. Aquí no se discute la importancia de un suelo de derechos, pero se introduce la idea de que determinados colectivos ---``personas mayores que viven solas'', ``hogares con menores en contextos de privación severa'', ``población migrante sin red de apoyo''--- deben ser atendidos con prioridad, bien porque su susceptibilidad al daño es mayor, bien porque disponen de menos recursos para afrontarlo.

Esta lógica de focalización se apoya, de manera explícita o implícita, en la ontología disposicional-clínica y en la tecnocrático-instrumental. El lenguaje de los ``grupos vulnerables'' se traduce con frecuencia en \textbf{criterios de elegibilidad} para programas específicos, en baremos de acceso a prestaciones o en mapas que señalan zonas prioritarias de intervención. Desde el punto de vista normativo, el argumento es doble: por un lado, la focalización sería una forma de justicia correctiva, al dirigir más recursos hacia quienes enfrentan mayores riesgos; por otro, se la presenta como una exigencia de eficiencia en contextos de recursos limitados.

Esta orientación tiene una ventaja evidente: evita la ficción de la homogeneidad y reconoce que, dentro de un marco universal de derechos, existen grupos y territorios que requieren \textbf{atención reforzada}. Sin embargo, también genera tensiones importantes. La categoría de ``grupo vulnerable'' puede derivar en \textbf{estigmatización}, fijar identidades deficitarias o invisibilizar la capacidad de agencia de las personas; la focalización puede transformarse en una suerte de ``gestión de reservas de vulnerabilidad'' que convive con recortes generalizados en los sistemas universales.

En el marco de esta tesis, esta tensión es central. El IVEC incorpora la necesidad de identificar territorios y combinaciones de factores que justifican priorización, pero lo hace explícitamente desde un horizonte universalista: se trata de intensificar la protección allí donde la vulnerabilidad se acumula, no de restringir los derechos solo a quienes cumplen determinados criterios de daño.

\subsection{Vulnerabilidad y justicia estructural: redistribución, reconocimiento y poder}\label{vulnerabilidad-y-justicia-estructural-redistribuciuxf3n-reconocimiento-y-poder}

Una tercera orientación normativa, estrechamente vinculada a la ontología estructural-relacional, entiende la vulnerabilidad ante todo como \textbf{problema de justicia}. Aquí el foco ya no se sitúa en si el Estado debe o no proteger (se da por supuesto que sí), ni en qué grupos han de ser priorizados, sino en \textbf{qué estructuras producen la vulnerabilidad} y qué transformaciones serían necesarias para desactivarlas.

La teoría de las capacidades de Sen y Nussbaum, por ejemplo, plantea que la injusticia se mide por la privación de capacidades efectivas para llevar vidas que las personas tienen razones para valorar \citep{sen1999, nussbaum2011}. Desde esta perspectiva, las situaciones que en el plano operativo se describen como ``vulnerabilidad'' remiten, en realidad, a fallos profundos en la distribución de recursos, en el acceso a derechos y en el reconocimiento de ciertas formas de vida. Fraser añade a esta lectura la necesidad de pensar la justicia como combinación de \textbf{redistribución}, \textbf{reconocimiento} y \textbf{participación}: no basta con cambiar la distribución de bienes, es preciso transformar los marcos interpretativos que devalúan a ciertos grupos y los procedimientos que los excluyen de la toma de decisiones \citep{fraser2013fortunes}.

En los estudios de vulnerabilidad social, esta orientación se traduce en una exigencia clara: las políticas no pueden limitarse a \textbf{gestionar consecuencias} (por ejemplo, compensando pérdidas tras un desastre o paliando la pobreza mediante prestaciones mínimas), sino que han de interrogar y modificar los procesos que producen sistemáticamente esas exposiciones: modelos de urbanización, regímenes laborales, sistemas fiscales, dispositivos de control migratorio, etc.

Normativamente, esta perspectiva refuerza la dimensión \textbf{crítica} de la vulnerabilidad. Cuestiona concepciones que la naturalizan o la presentan como efecto inevitable de ``choques externos'' y la reubica en el terreno de las responsabilidades políticas. Sin embargo, si no se articula con las otras orientaciones, puede quedar en un nivel muy general de denuncia, difícil de traducir en instrumentos concretos de diagnóstico y priorización.

Esta tesis asume una posición explícita en esta línea: el IVEC se concibe no solo como herramienta para \textbf{localizar} vulnerabilidad, sino como dispositivo para \textbf{leerla estructuralmente}, de manera que sus resultados orienten la acción institucional hacia las causas y no solo hacia los síntomas.

\subsection{Vulnerabilidad, sostenibilidad y justicia socioecológica}\label{vulnerabilidad-sostenibilidad-y-justicia-socioecoluxf3gica}

La expansión de marcos internacionales como el Marco de Acción de Hyogo, el Marco de Sendai, la Agenda 2030 o el Acuerdo de París ha situado la vulnerabilidad en el centro de las discusiones sobre \textbf{sostenibilidad}, \textbf{reducción del riesgo de desastres} y \textbf{adaptación climática} \citep{un2005hyogo, un2015sendai, un2015agenda2030, unfccc2015paris}. En estos contextos, la vulnerabilidad adquiere una dimensión normativa específica: se la interpreta como expresión de desigualdades profundas en la forma en que diferentes sociedades participan en la generación de riesgos globales y, al mismo tiempo, soportan sus consecuencias.

La noción de \textbf{justicia climática} sintetiza bien esta preocupación. Subraya que quienes menos han contribuido históricamente al cambio climático ---poblaciones de bajos ingresos, comunidades rurales, pueblos indígenas--- suelen ser los más expuestos a sus impactos, con menores capacidades de adaptación. La vulnerabilidad, en este registro, no es solo un problema de protección interna, sino una cuestión de \textbf{responsabilidad distribuida} entre regiones, generaciones y grupos sociales.

En paralelo, los enfoques de sistemas socioecológicos ponen el acento en la necesidad de salvaguardar al mismo tiempo \textbf{condiciones de vida dignas} y \textbf{funciones ecosistémicas}. Normativamente, esto se traduce en principios como la precaución, la responsabilidad intergeneracional o el derecho a un medio ambiente saludable. Las decisiones sobre ordenación del territorio, infraestructuras, usos del suelo o gestión del agua dejan de ser asuntos meramente técnicos para convertirse en elecciones con implicaciones directas en la distribución de la vulnerabilidad.

En el caso valenciano, esta perspectiva es especialmente pertinente en ámbitos como el litoral, las áreas inundables o los sistemas agrarios periurbanos. Las opciones de desarrollo turístico, la urbanización de zonas de riesgo o las transformaciones del regadío no solo alteran ecosistemas, sino que reconfiguran de forma duradera quién queda más o menos expuesto ante eventos extremos. El plano normativo socioecológico exige, por tanto, considerar la vulnerabilidad como criterio central en las decisiones de planificación, y no como un resultado colateral que se gestiona a posteriori.

\subsection{Posicionamiento normativo de la tesis}\label{posicionamiento-normativo-de-la-tesis}

Las orientaciones normativas descritas no se excluyen entre sí, pero ponen el acento en dimensiones distintas. La vulnerabilidad puede pensarse como condición universal que justifica un \textbf{suelo de derechos} y un deber general de cuidado; como criterio de \textbf{priorización} que orienta recursos hacia grupos y territorios concretos; como expresión de \textbf{injusticias estructurales} que requieren transformaciones profundas; y como indicador de \textbf{desajustes socioecológicos} que interpelan las políticas de sostenibilidad y adaptación.

El posicionamiento adoptado en esta tesis es deliberadamente \textbf{combinado}. Parte de una concepción universalista de la protección social, inspirada en la vulnerabilidad constitutiva, pero la articula con una lectura estructural y territorial del daño, próxima a las perspectivas de justicia social. Reconoce la utilidad de identificar perfiles y áreas especialmente vulnerables, pero considera que esa priorización solo es legítima si se apoya en sistemas sólidos de derechos universales y si evita fijar de manera estigmatizante a los llamados ``grupos vulnerables''. Integra, finalmente, la dimensión socioecológica como requisito para que la evaluación de la vulnerabilidad territorial no ignore las dinámicas ambientales que la condicionan.

Este marco normativo no se limita a enunciar principios abstractos. Informa de manera directa el diseño del IVEC y la lectura que se hará de sus resultados: orienta la selección de dimensiones, justifica la atención preferente a las desigualdades estructurales y territoriales, y condiciona la interpretación de los indicadores como instrumentos al servicio de la justicia social y territorial, y no como fines en sí mismos.

En las secciones siguientes, esta base normativa se pondrá en diálogo con los planos epistemológico-teórico y metodológico, de modo que la construcción del dispositivo de análisis no pueda desprenderse de las preguntas de fondo que lo motivan: \textbf{qué vidas se quieren proteger, frente a qué riesgos, en qué territorios y con qué tipo de respuestas institucionales}.

\section{Plano epistemológico y teórico: cómo se conoce y se explica la vulnerabilidad}\label{plano-epistemoluxf3gico-y-teuxf3rico-cuxf3mo-se-conoce-y-se-explica-la-vulnerabilidad}

Tras aclarar \textbf{qué es} la vulnerabilidad (plano ontológico) y \textbf{qué debe hacerse} cuando se detecta (plano normativo), el siguiente paso consiste en preguntarse \textbf{cómo se construye conocimiento sobre ella y con qué marcos se interpreta}. El plano epistemológico se refiere a los criterios que definen qué cuenta como evidencia válida, qué relación se establece entre teoría y datos, qué se considera explicación adecuada y qué formas de ignorancia se toleran o se consideran problemáticas. El plano teórico, por su parte, organiza ese conocimiento en modelos, esquemas causales y narrativas explicativas sobre por qué determinadas personas, hogares o territorios resultan vulnerables.

En la práctica, ambos planos son difícilmente separables. Las formas de conocer la vulnerabilidad llevan incorporadas ciertas ideas sobre cómo funciona el mundo social y ambiental, y los marcos teóricos que se adoptan orientan a su vez qué datos se buscan, cómo se recogen y cómo se interpretan. Por este motivo, en este capítulo se opta por abordar de manera conjunta el \textbf{plano epistemológico y el plano teórico}, presentando las grandes orientaciones que, a la vez, definen modos de conocer y proponen explicaciones características sobre la vulnerabilidad social.

Esta elección responde también a una razón pragmática: buena parte de los modelos que ya se han mencionado en el plano ontológico ---desde el enfoque \emph{Pressure and Release} hasta las propuestas de Cardona, MOVE o los marcos socioecológicos--- no son sólo ontologías implícitas sobre qué ``es'' la vulnerabilidad, sino también \textbf{artefactos epistemológico--teóricos}: especifican qué dimensiones deben observarse, cómo se relacionan entre sí y qué tipo de causalidad se asume. Separar artificialmente ambos planos generaría duplicidades y fragmentaría un debate que, en la literatura, aparece entrelazado.

En términos muy generales, el campo puede organizarse en torno a tres grandes orientaciones epistemológico--teóricas, que dialogan estrechamente con los planos ontológicos ya descritos:

\begin{itemize}
\item
  Una \textbf{orientación empirista--instrumental}, asociada a ontologías tecnocrático--instrumentales y disposicionales, que concibe la vulnerabilidad como algo observable, mensurable y formalizable en modelos de riesgo. Aquí dominan las tradiciones epidemiológicas, clínicas, demográficas y de análisis de indicadores, así como los marcos que operacionalizan la vulnerabilidad en términos de variables y funciones (por ejemplo, ciertos usos de la fórmula riesgo = amenaza × vulnerabilidad).
\item
  Una \textbf{orientación estructural--crítica}, vinculada a las ontologías estructural--relacionales, que entiende la vulnerabilidad como resultado de mecanismos sociales, históricos y territoriales de producción de desigualdad. En este registro se inscriben los enfoques que, como PAR o los estudios de geografía crítica del riesgo, combinan análisis histórico, sociología del poder y lectura territorial para explicar por qué determinados grupos y lugares resultan sistemáticamente más expuestos.
\item
  Una \textbf{orientación compleja--sistémica}, asociada a las ontologías funcional--sistémicas, que concibe la vulnerabilidad como propiedad emergente de sistemas socioecológicos acoplados. Aquí se sitúan los marcos de sistemas socioecológicos, los enfoques de resiliencia y adaptación, y las propuestas que ponen el acento en la no linealidad, los umbrales y la co--producción de procesos sociales y ambientales.
\end{itemize}

Junto a estas tres orientaciones principales, la literatura incorpora contribuciones \textbf{interpretativas y fenomenológicas} ---centradas en las experiencias vividas de la vulnerabilidad y en los significados que los actores atribuyen al riesgo---, que no constituyen una cuarta familia separada, pero sí matizan y problematizan las otras tres. En muchos casos, estos trabajos se articulan con perspectivas estructurales o críticas, aportando densidad cualitativa a diagnósticos construidos a partir de datos cuantitativos o modelos sistémicos.

El objetivo de esta sección no es elaborar una tipología exhaustiva, sino \textbf{identificar los modos dominantes de conocer y explicar la vulnerabilidad social}, mostrar cómo se enlazan con las ontologías y compromisos normativos revisados en las secciones anteriores, y situar en ese mapa el posicionamiento de esta tesis. En particular, el desarrollo del IVEC y el análisis bibliométrico y semántico del campo se enmarcan en una apuesta por combinar la capacidad descriptiva y comparativa de la orientación empirista--instrumental con la lectura estructural--crítica de las desigualdades y con una sensibilidad hacia la complejidad socioecológica.

En los apartados siguientes se profundizará en cada una de estas orientaciones, atendiendo a tres cuestiones en cada caso: qué entiende por conocimiento válido sobre la vulnerabilidad, qué modelo explicativo la estructura y qué potencial y límites presenta para analizar la vulnerabilidad social y territorial en la Comunitat Valenciana.

\subsection{Orientación empirista--instrumental: vulnerabilidad como riesgo observable y medible}\label{orientaciuxf3n-empiristainstrumental-vulnerabilidad-como-riesgo-observable-y-medible}

La orientación empirista--instrumental parte de una premisa sencilla pero poderosa: la vulnerabilidad puede y debe ser \textbf{observada, medida y formalizada}. Conocer implica, en este marco, convertir situaciones complejas en variables, construir indicadores comparables, estimar relaciones estadísticas entre exposición y daño, y traducir la incertidumbre en probabilidades manejables. La vulnerabilidad se aproxima así al lenguaje del riesgo epidemiológico, de la demografía, de la econometría aplicada o de la ingeniería de desastres, donde la explicación se formula en términos de asociaciones cuantificables y funciones bien definidas.

Epistemológicamente, esta orientación se apoya en variantes del empirismo y del positivismo: el mundo social y territorial se concibe como una realidad accesible a través de indicadores observables; las hipótesis se someten a contraste mediante datos; la validez del conocimiento se sustenta en la replicabilidad, la robustez estadística y la capacidad de predicción. Teóricamente, suele trabajar con esquemas causales relativamente lineales, donde la vulnerabilidad aparece como función de ciertas características de los individuos, los hogares o los territorios (edad, nivel educativo, tipo de empleo, densidad residencial, etc.) y se inserta en modelos de riesgo del tipo ``probabilidad de que ocurra X dado que se cumplen Y condiciones''.

Esta manera de conocer ha sido particularmente influyente en dos grandes dominios. En el campo de la salud, las aproximaciones epidemiológicas y de salud pública construyen \textbf{poblaciones de riesgo} a partir de encuestas, registros clínicos o sistemas de vigilancia, identificando factores que aumentan la probabilidad de enfermar o de sufrir eventos adversos. En el ámbito socioeconómico y territorial, los análisis de pobreza, exclusión y vulnerabilidad utilizan censos, encuestas de hogares y registros administrativos para estimar la distribución de recursos, oportunidades y privaciones. En ambos casos, la vulnerabilidad se vuelve ``legible'' mediante baterías de indicadores que permiten comparar grupos, barrios o municipios, y detectar concentraciones de fragilidad.

Una de las fortalezas de esta orientación reside en su capacidad para \textbf{producir descripciones sistemáticas y comparables}. El paso de categorías vagas (``personas en situación precaria'', ``barrios desfavorecidos'') a medidas explícitas obliga a precisar qué dimensiones se consideran relevantes, con qué umbrales y con qué peso relativo. Ello facilita el diálogo con la planificación y con la gestión de políticas: los diagnósticos cuantificados permiten priorizar intervenciones, asignar recursos, justificar decisiones y hacer seguimiento de tendencias en el tiempo. En el contexto de esta tesis, esta capacidad descriptiva y comparativa resulta imprescindible tanto para el análisis bibliométrico y semántico como para la construcción del IVEC, que se apoya en datos cuantitativos para representar de forma sintética configuraciones de vulnerabilidad territorial.

Sin embargo, el modo de explicación que acompaña a esta epistemología tiene límites claros. Al privilegiar asociaciones entre variables observadas, tiende a \textbf{debilitar la atención a mecanismos sociales profundos}, a genealogías históricas y a relaciones de poder que no se capturan fácilmente en indicadores. La vulnerabilidad aparece entonces como resultado de combinaciones de características individuales o territoriales, sin que se explicite con suficiente claridad cómo se produjeron esas configuraciones ni qué estructuras las sostienen. La linealidad de muchos modelos ---que buscan aislar efectos netos de cada factor--- contrasta con la naturaleza acumulativa, no lineal y frecuentemente contingente de las trayectorias de vulnerabilidad.

Además, la orientación empirista--instrumental corre el riesgo de \textbf{naturalizar sus propias decisiones técnicas}. La selección de variables, la forma de agregarlas, la elección de umbrales o la escala de análisis no son operaciones neutras, sino elecciones normativas y teóricas que definen qué cuenta como vulnerabilidad y qué queda fuera. Cuando estos supuestos permanecen implícitos, el resultado puede presentarse como descripción objetiva, invisibilizando que otras representaciones del mismo territorio, con otras variables y otras relaciones, producirían mapas y diagnósticos diferentes.

Estas limitaciones no invalidan la orientación, pero sí señalan la necesidad de situarla en diálogo con perspectivas estructurales y críticas. En la medida en que se reconozca su carácter \textbf{instrumental} ---esto es, su función como forma de hacer visibles ciertos patrones para la acción pública---, la epistemología empirista puede funcionar como base de un conocimiento más amplio, siempre que sus resultados se vuelvan a interrogar a la luz de preguntas sobre causalidad social, justicia y poder. La explicación deja entonces de reducirse a ``las zonas con tales características presentan mayor vulnerabilidad'' para abrirse a cuestiones como ``qué procesos produjeron esas características'', ``qué actores se benefician de su reproducción'' o ``qué transformaciones estructurales serían necesarias para modificarlas''.

En el diseño del IVEC, esta tesis adopta precisamente esta posición intermedia. Asume la utilidad de la orientación empirista--instrumental para \textbf{construir indicadores robustos} y para ofrecer una cartografía comparativa de la vulnerabilidad social y territorial en la Comunitat Valenciana, pero evita confundir el índice con la realidad que representa. El dispositivo cuantitativo se concibe como una \textbf{infraestructura de conocimiento}: una forma de ordenar la información que debe ser interpretada en diálogo con los planos ontológico y normativo y contrastada con lecturas históricas, estructurales y socioecológicas. Solo en ese cruce ---y no en la cifra por sí misma--- puede aspirar a convertirse en una herramienta explicativa y orientadora de la acción pública.

\subsection{Orientación estructural--crítica: vulnerabilidad como resultado de mecanismos sociales e históricos}\label{orientaciuxf3n-estructuralcruxedtica-vulnerabilidad-como-resultado-de-mecanismos-sociales-e-histuxf3ricos}

Frente a la confianza de la orientación empirista--instrumental en la observación sistemática y la correlación entre variables, la orientación estructural--crítica parte de una inquietud diferente: \textbf{no basta con describir quién aparece como vulnerable}, es necesario reconstruir los \textbf{mecanismos sociales, históricos y territoriales} que han producido esa vulnerabilidad. Conocer implica, aquí, ir más allá de los patrones observables y preguntar por las estructuras que los sostienen, por las relaciones de poder que los reproducen y por los conflictos que los atraviesan.

Epistemológicamente, esta orientación se nutre de tradiciones como el \textbf{realismo crítico}, la sociología histórica y la geografía crítica. La propuesta de Bhaskar de distinguir entre eventos observables, mecanismos generativos y estructuras subyacentes resulta especialmente pertinente \citep{bhaskar1978}: los datos empíricos ---tasas de pobreza, daños por desastres, indicadores de salud o mapas de exclusión--- son tomados como \textbf{puntos de partida}, no como explicación suficiente. El trabajo epistemológico consiste en una operación de retroducción: a partir de patrones recurrentes de daño y desprotección, se infieren y contrastan los mecanismos que los generan, desde la configuración de los mercados de trabajo hasta las políticas urbanas, los regímenes de propiedad del suelo o las formas de discriminación institucional.

Sobre este suelo se organizan marcos teóricos que han sido centrales para el estudio de la vulnerabilidad. La \textbf{sociedad del riesgo} de Beck \citep{beck1992} muestra cómo los riesgos modernos ---tecnológicos, ambientales, financieros--- se producen dentro de las mismas estructuras que prometen controlarlos, y cómo su distribución refleja y amplifica divisiones de clase, género u origen. La teoría de los capitales y los campos de Bourdieu permite leer la vulnerabilidad como una posición desigual en espacios sociales estructurados, donde el acceso diferenciado a recursos económicos, culturales y simbólicos condiciona la capacidad de anticipar, evitar o absorber el daño \citep{bourdieu1986, bourdieu1994}. El marco \emph{Pressure and Release} (PAR) de Blaikie y colaboradores traduce estas intuiciones a un lenguaje específicamente centrado en riesgos y desastres, articulando causas profundas, presiones dinámicas y condiciones inseguras como momentos de una misma cadena causal \citep{blaikie1994}.

En esta orientación, la explicación de la vulnerabilidad adopta la forma de \textbf{narrativas causales densas}. No se limita a señalar que ciertos barrios presentan peores indicadores, sino que reconstruye procesos de desinversión pública, desplazamiento residencial, precarización laboral o cambios en la regulación urbanística que han ido acumulando desventajas en esos espacios. La vulnerabilidad aparece entonces como efecto de trayectorias históricas de desigualdad, de decisiones políticas concretas y de conflictos distributivos, más que como resultado neutro de ``factores'' descontextualizados.

Ello se traduce en un repertorio metodológico que combina, con distinta intensidad, análisis cuantitativo y cualitativo. Los registros estadísticos, los censos o las bases de datos georreferenciadas se utilizan para identificar \textbf{patrones de concentración del daño o de la desprotección}; sobre estos patrones se superponen análisis documentales, reconstrucciones históricas, entrevistas, etnografía urbana o estudios de caso que permiten entender cómo han operado determinados mecanismos en contextos específicos. La evidencia válida no se limita a la cifra, sino que incluye testimonios, archivos, legislación, planes urbanísticos, memorias institucionales o relatos de conflicto social.

La dimensión crítica de esta orientación no reside únicamente en el tipo de datos que utiliza, sino en la forma de interrogar la realidad. La vulnerabilidad es leída como \textbf{síntoma de injusticia}, como indicador de fallos en la estructura de oportunidades, en el reconocimiento institucional o en la distribución de la protección. De ahí la estrecha afinidad entre esta epistemología y los marcos normativos de justicia social y capacidades revisados en la sección anterior: conocer la vulnerabilidad implica también señalar responsabilidades, evidenciar asimetrías de poder y abrir preguntas sobre qué transformaciones serían necesarias para que determinados daños dejaran de ser previsibles.

Esta perspectiva presenta, sin embargo, sus propios límites. Al privilegiar la reconstrucción de mecanismos profundos y de genealogías largas, corre el riesgo de \textbf{subestimar la dimensión situada y experiencial} de la vulnerabilidad, así como la capacidad de agencia de los actores que habitan contextos estructuralmente desfavorables. Del mismo modo, puede mostrar cierta reticencia a utilizar herramientas cuantitativas e instrumentos de medición comparativa, precisamente allí donde estos pueden aportar claridad sobre el alcance y la distribución de las desigualdades. Cuando esto ocurre, la explicación corre el peligro de quedar encapsulada en narrativas potentes pero difícilmente operativas para la planificación o la evaluación de políticas.

En el marco de esta tesis, la orientación estructural--crítica funciona como \textbf{contrapunto necesario} a la empirista--instrumental. Orienta la selección de variables del IVEC hacia aquellas que poseen un claro anclaje en procesos de desigualdad ---trayectorias de precariedad laboral, estructuras de edad, formas de tenencia, segregación residencial--- y exige que la lectura de los resultados no se agote en la comparación de valores, sino que se inserte en historias territoriales concretas. Al mismo tiempo, invita a interpretar el análisis bibliométrico y semántico del campo no sólo como un mapa neutral de términos y coautorías, sino como una huella de cómo ciertas agendas, instituciones y regiones han conseguido definir qué cuenta como vulnerabilidad y qué aproximaciones quedan en la periferia del canon académico.

En suma, la orientación estructural--crítica aporta una forma de conocer la vulnerabilidad que toma en serio su carácter \textbf{históricamente producido y políticamente cargado}. Su contribución principal es recordar que no hay mapa de vulnerabilidad que pueda considerarse completo si no se pregunta quién lo ha hecho posible, con qué categorías, desde qué posición y sobre qué historias de desigualdad se proyecta. La tesis se apoya en esta orientación para no perder de vista que toda medición, y el propio IVEC, deben leerse a la luz de estos procesos más amplios, si no quieren convertirse en meras descripciones despolitizadas de estados de cosas injustos.

\subsection{Orientación compleja--sistémica: vulnerabilidad como propiedad emergente de sistemas socioecológicos}\label{orientaciuxf3n-complejasistuxe9mica-vulnerabilidad-como-propiedad-emergente-de-sistemas-socioecoluxf3gicos}

La orientación compleja--sistémica se distancia tanto de la lógica aditiva de la orientación empirista--instrumental como del énfasis en mecanismos sociales profundos propio de la orientación estructural--crítica. Su punto de partida es otro: la vulnerabilidad no se localiza únicamente en individuos, hogares o grupos sociales, sino en \textbf{configuraciones dinámicas} donde interactúan, de manera inseparable, procesos ecológicos, tecnológicos, económicos, institucionales y culturales. Conocer la vulnerabilidad implica, por tanto, comprender el \textbf{comportamiento de sistemas socioecológicos acoplados}, sus patrones de retroalimentación y sus umbrales de transformación.

Epistemológicamente, esta orientación se nutre de la teoría de sistemas, de los estudios sobre complejidad y de los enfoques de resiliencia ecológica y socioecológica. Autores como Turner, Adger o Folke han formulado algunas de las propuestas más influyentes en este campo \citep{turner2003, adger2006, folke2006}. En estos trabajos, la vulnerabilidad se define como función de la exposición, la sensibilidad y la capacidad de respuesta, pero esta tríada se inscribe en una visión más amplia: los sistemas están compuestos por múltiples componentes interdependientes, operan en varias escalas espaciales y temporales, y su respuesta a las perturbaciones no es lineal ni proporcional.

Teóricamente, el foco se desplaza desde las listas de factores hacia las \textbf{relaciones entre componentes}. Las preguntas dejan de formularse exclusivamente en términos de ``quién es vulnerable'' para plantearse como ``qué configuraciones de usos del suelo, infraestructuras, instituciones y prácticas de vida producen ciertos patrones de vulnerabilidad'' o ``cómo interactúan decisiones locales y dinámicas globales para generar nuevos riesgos''. La explicación adopta la forma de \textbf{relatos sistémicos}: cambios en un subsistema (por ejemplo, la expansión de un determinado tipo de cultivo, la construcción de una infraestructura o la modificación de una política de suelo) desencadenan reacciones en cadena que alteran la exposición, la sensibilidad o la capacidad de respuesta de otros componentes del sistema.

Este modo de conocer se traduce en una preferencia por metodologías que captan interacciones y dinámicas: modelos de simulación, análisis de escenarios, estudios de redes, análisis de trayectorias históricas de sistemas territoriales, combinados cada vez más con aproximaciones participativas que incorporan conocimiento local sobre cómo se perciben y gestionan los riesgos. La validez del conocimiento no se mide solo por la precisión en la estimación de parámetros, sino por la \textbf{capacidad del modelo para representar comportamientos plausibles}, identificar puntos de fragilidad, explorar futuros posibles y orientar decisiones bajo condiciones de incertidumbre.

Una contribución central de esta orientación es poner en primer plano la idea de \textbf{no linealidad y umbral}. Los sistemas socioecológicos pueden absorber perturbaciones durante largos periodos sin cambios ostensibles, pero al cruzar ciertos límites experimentan transiciones rápidas hacia estados cualitativamente distintos: inundaciones que se agravan de forma súbita tras años de urbanización difusa, incendios que se hacen más severos al combinarse el abandono agrícola con olas de calor extremas, redes de cuidados que colapsan tras encadenar crisis económicas y sanitarias. La vulnerabilidad se entiende, en consecuencia, como una condición que puede permanecer larvada ---aparentemente contenida--- hasta que determinadas combinaciones de factores precipitan un cambio abrupto.

Esta perspectiva permite integrar de manera más explícita la dimensión ambiental y climática en el análisis de la vulnerabilidad social. Las contribuciones de Adger sobre vulnerabilidad y adaptación al cambio climático \citep{adger2006}, por ejemplo, muestran cómo los impactos del calentamiento global no pueden explicarse únicamente por la exposición física a eventos extremos, sino por la manera en que los sistemas sociales dependen de servicios ecosistémicos, infraestructuras y mercados que se reorganizan bajo presión climática. La atención se dirige hacia \textbf{acoplamientos críticos}: sistemas alimentarios dependientes de importaciones, ciudades que dependen de redes de transporte vulnerables, regiones donde las políticas agrarias, el ciclo del agua y la estructura de la propiedad del suelo generan patrones de riesgo cada vez más difíciles de gestionar.

Ahora bien, la orientación compleja--sistémica no está exenta de ambivalencias. El énfasis en sistemas, resiliencia y adaptación puede derivar, si no se cuida, en un lenguaje que \textbf{neutraliza el conflicto y la responsabilidad}. Cuando los problemas se describen en términos de ``ajustes del sistema'' o ``capacidad adaptativa'', resulta fácil perder de vista quién decide qué adaptaciones son aceptables, quién soporta sus costes y quién se beneficia de mantener ciertas configuraciones productivas o territoriales. La vulnerabilidad corre entonces el riesgo de aparecer como resultado de dinámicas impersonales, sin sujetos ni relaciones de poder claramente identificables.

Desde el punto de vista epistemológico, además, la sofisticación de algunos modelos puede generar una ilusión de exhaustividad: la sensación de que, al representar muchos componentes y relaciones, el sistema ha sido plenamente comprendido. Sin embargo, toda modelización implica \textbf{selección y simplificación}. Lo que queda fuera ---por ejemplo, formas de discriminación institucional, economías informales o prácticas de resistencia comunitaria--- puede seguir siendo decisivo para entender quién resulta vulnerable en la práctica.

En esta tesis, la orientación compleja--sistémica se incorpora de manera selectiva. Es fundamental para pensar la vulnerabilidad territorial en la Comunitat Valenciana en relación con procesos ambientales y climáticos ---especialmente en lo relativo a riesgos hidrometeorológicos, presiones sobre recursos naturales y transformaciones del modelo territorial---, y proporciona un lenguaje útil para hablar de \textbf{interdependencias y umbrales}. Sin embargo, no se adopta de forma aislada ni autosuficiente. Sus herramientas y marcos se leen en diálogo con la orientación estructural--crítica, que devuelve al análisis la cuestión de la justicia y del poder, y con la orientación empirista--instrumental, que ofrece descripciones comparables de desigualdades y patrones territoriales.

\subsection{Orientación fenomenológica e interpretativa: vulnerabilidad como experiencia vivida y significado situado}\label{orientaciuxf3n-fenomenoluxf3gica-e-interpretativa-vulnerabilidad-como-experiencia-vivida-y-significado-situado}

Tal y como se anticipaba en la introducción de este plano, junto a las grandes orientaciones empirista--instrumental, estructural--crítica y compleja--sistémica se ha desarrollado un conjunto de trabajos que abordan la vulnerabilidad desde una clave \textbf{fenomenológica e interpretativa}. Su punto de partida no es tanto la estructura social, el sistema socioecológico o el patrón estadístico, sino la pregunta por \textbf{cómo se vive, se siente y se narra la vulnerabilidad} en la experiencia cotidiana de los sujetos y los colectivos.

Epistemológicamente, esta orientación se enraíza en tradiciones fenomenológicas y hermenéuticas, así como en la antropología interpretativa y en la sociología cualitativa. El énfasis se sitúa en los \textbf{mundos de vida}: la vulnerabilidad no se da simplemente como ``hecho'' externo, sino como trama de significados, afectos y prácticas con las que las personas se orientan en contextos de incertidumbre y daño potencial. El riesgo, la exposición o la inseguridad no son solo magnitudes objetivas, sino también formas de ser percibidas, nombradas y elaboradas simbólicamente \citep{lupton1999risk, douglas1992risk}.

Teóricamente, esta orientación entiende la vulnerabilidad como \textbf{experiencia encarnada}: se manifiesta en el cuerpo que siente miedo o fatiga, en la temporalidad suspendida de quien espera una resolución administrativa o un diagnóstico médico, en la pérdida de familiaridad con el entorno tras una inundación o un desahucio, en la percepción de ausencia de reconocimiento por parte de las instituciones. La explicación no se formula únicamente en términos de causas externas, sino que busca comprender cómo los sujetos \textbf{dotan de sentido} a su propia exposición al daño, cómo articulan narrativamente las rupturas biográficas y cómo negocian, aceptan, resisten o resignifican las categorías con las que son clasificados como vulnerables.

Metodológicamente, esta forma de conocer se apoya en \textbf{entrevistas en profundidad, historias de vida, etnografías, observación participante y análisis narrativo}. La validez del conocimiento no se mide por la representatividad estadística, sino por la capacidad de producir descripciones densas que capturen la textura de la experiencia: las ambivalencias, los silencios, las contradicciones, los matices afectivos. La vulnerabilidad aparece entonces como un proceso que se despliega en el tiempo, donde se entrelazan dimensiones materiales y simbólicas: la pérdida de empleo, por ejemplo, no es solo un descenso de ingresos, sino también una transformación del estatus, de las redes de apoyo y de la propia imagen de sí; la etiqueta institucional de ``persona vulnerable'' puede abrir derechos, pero también estigmatizar y reducir la identidad a un diagnóstico administrativo \citep{das2007life, farmer2004pathologies}.

Esta orientación resulta especialmente fecunda para poner en cuestión la aparente neutralidad de las categorías técnicas. Al escuchar cómo los sujetos experimentan ser situados en un mapa de vulnerabilidad o ser objeto de intervenciones específicas, se evidencian fricciones, resistencias y resignificaciones: hay quienes no se reconocen en la etiqueta que se les atribuye; hay quienes perciben las ayudas como humillantes; hay quienes reivindican su vulnerabilidad como base para reclamar derechos, y quienes la rechazan por asociarla a dependencia o incapacidad. La vulnerabilidad deja entonces de ser sólo un diagnóstico externo y se convierte también en \textbf{objeto de disputa simbólica}.

La vinculación con los planos ontológicos y normativos ya desarrollados es directa. La fenomenología de la vulnerabilidad da espesor a la \textbf{vulnerabilidad constitutiva} al mostrar cómo la fragilidad universal se inscribe en biografías concretas; matiza la lectura \textbf{disposicional--clínica} al mostrar que no hay susceptibilidad corporal que no esté mediada por significados, temores y expectativas; y complejiza la perspectiva \textbf{estructural--relacional} al recordar que las estructuras se viven desde posiciones situadas, con grados distintos de conciencia, resignación, ira o esperanza. Desde el punto de vista normativo, esta orientación refuerza la dimensión de \textbf{reconocimiento}: no basta con redistribuir recursos o modificar estructuras si las personas afectadas no son tratadas como sujetos de voz legítima sobre su propia vulnerabilidad.

En relación con las otras orientaciones epistemológico--teóricas, la aportación principal de la perspectiva fenomenológica e interpretativa es \textbf{descentrar la mirada exclusivamente externa}. Frente al riesgo de que la orientación empirista--instrumental reduzca la vulnerabilidad a un índice, o de que la orientación compleja--sistémica la diluya en dinámicas sistémicas, los estudios interpretativos recuerdan que toda configuración de vulnerabilidad se encarna en vidas que se preguntan qué está ocurriendo, por qué les ocurre a ellas y qué margen de acción tienen. Frente al énfasis en estructuras de la orientación crítica, subrayan que incluso en contextos fuertemente condicionados existe agencia, elaboración simbólica y construcción de sentido.

Esta tesis no se sitúa en la tradición fenomenológica como enfoque dominante, pero sí incorpora algunas de sus intuiciones como \textbf{exigencia transversal}: la necesidad de que cualquier representación de la vulnerabilidad ---sea estadística, estructural o sistémica--- permanezca abierta a ser contrastada con las experiencias de quienes la viven; la importancia de evitar que el IVEC, o cualquier instrumento similar, se convierta en una clasificación muda sobre los territorios sin incorporar la voz de los actores locales; y la conveniencia de que futuras fases de investigación y de evaluación cualitativa se apoyen en metodologías interpretativas para \textbf{verificar, matizar o cuestionar} los diagnósticos derivados del análisis cuantitativo y de los marcos teóricos estructurales y sistémicos.

Con esta cuarta orientación, el plano epistemológico--teórico queda completo: la vulnerabilidad puede conocerse como riesgo observable y medible, como resultado de mecanismos sociales e históricos, como propiedad emergente de sistemas socioecológicos y como experiencia vivida y narrada. En la sección de cierre de este plano ---que podemos redactar a continuación si quieres--- se explicitará cómo se articula, de manera concreta, la posición de la tesis frente a estas cuatro miradas en el diseño del IVEC y en el análisis bibliométrico y semántico que cierra el capítulo.

\section{Plano metodológico: cómo se operacionaliza la vulnerabilidad en la investigación aplicada}\label{plano-metodoluxf3gico-cuxf3mo-se-operacionaliza-la-vulnerabilidad-en-la-investigaciuxf3n-aplicada}

Los planos ontológico, normativo y epistemológico--teórico delimitan qué se entiende por vulnerabilidad, qué debe protegerse y qué cuenta como conocimiento válido y explicación adecuada. El plano metodológico desplaza ahora la atención hacia una cuestión distinta, aunque estrechamente vinculada: \textbf{cómo se traduce ese marco conceptual en estrategias concretas de investigación y de operacionalización}. Dicho de otro modo, qué dispositivos empíricos se utilizan para identificar situaciones de vulnerabilidad, qué escalas se privilegian, qué fuentes de datos se movilizan y cómo se articulan, en la práctica, las distintas formas de saber.

En el campo de la vulnerabilidad social, las elecciones metodológicas lejos de ser neutras prolongan, a menudo de forma implícita, las ontologías y epistemologías previamente descritas. Las aproximaciones empiristas e instrumentales tienden a cristalizarse en \textbf{diseños cuantitativos basados en indicadores, índices y modelos de riesgo}; las perspectivas estructurales y críticas se apoyan en \textbf{estudios de caso histórico--comparativos}, cartografías de desigualdad y análisis documentales; los enfoques complejos y socioecológicos recurren con frecuencia a \textbf{modelos sistémicos, análisis de escenarios y herramientas acopladas de SIG y simulación}; por último, las orientaciones fenomenológicas e interpretativas se expresan metodológicamente en \textbf{etnografías, historias de vida y dispositivos cualitativos de escucha prolongada}. A estas familias se superponen, en las últimas décadas, metodologías participativas y de co--producción de conocimiento que atraviesan las anteriores y buscan reequilibrar las relaciones entre investigación académica y actores locales.

El objetivo de esta sección no es proponer un inventario exhaustivo de técnicas, sino \textbf{identificar las principales configuraciones metodológicas} que han ganado centralidad en el campo, mostrar cómo dialogan (o entran en tensión) con las orientaciones epistemológico--teóricas ya analizadas y situar, en ese mapa, las decisiones metodológicas de esta tesis. La presentación sigue, por tanto, un hilo paralelo: parte de los instrumentos de medición cuantitativa y modelización, se desplaza hacia las metodologías histórico--estructurales y sistémicas, incorpora las aportaciones de los enfoques cualitativos e interpretativos y culmina en las experiencias de combinación y co--producción que buscan integrar estas tradiciones.

\subsection{Medición cuantitativa e índices compuestos}\label{mediciuxf3n-cuantitativa-e-uxedndices-compuestos}

Las metodologías de \textbf{medición cuantitativa} constituyen, probablemente, el dispositivo más visible y estandarizado en los estudios de vulnerabilidad. Se apoyan en la lógica empirista--instrumental descrita anteriormente y se concretan en procedimientos de selección de variables, normalización, ponderación y agregación que permiten construir indicadores sintéticos y mapas comparativos. Uno de los referentes paradigmáticos es el \emph{Social Vulnerability Index} (SoVI), que integra mediante análisis de componentes principales un conjunto amplio de variables socioeconómicas, demográficas y residenciales para identificar patrones territoriales de vulnerabilidad frente a desastres \citep{cutter2003}. Trabajos posteriores han adaptado esta lógica a distintos contextos nacionales y regionales, introduciendo variaciones en la elección de indicadores y en las técnicas de reducción dimensional.

En paralelo, se han desarrollado índices de vulnerabilidad en ámbitos sectoriales específicos ---salud, pobreza energética, inseguridad alimentaria--- y dispositivos de diagnóstico territorial basados en censos y registros administrativos. En el contexto español, el \emph{Atlas de Distribución de la Renta y Desigualdad} y el \emph{Atlas de Áreas Urbanas Vulnerables} han mostrado la capacidad de estas metodologías para \textbf{identificar concentraciones espaciales de desventaja} y orientar políticas urbanas y sociales. En la Comunitat Valenciana, herramientas como el visor VEUS han permitido descender a escalas muy finas (sección censal), combinando dimensiones socioeconómicas, residenciales y de género para ofrecer una imagen detallada de la desigualdad territorial contemporánea.

Metodológicamente, estos dispositivos comparten rasgos comunes: dependen de \textbf{fuentes estadísticas masivas} (censos de población y vivienda, registros de empleo, padrones, encuestas de salud o de condiciones de vida), utilizan procedimientos sistemáticos de tratamiento (estandarización, análisis factorial o de componentes principales, técnicas de clusterización) y producen resultados espacialmente explícitos, fácilmente integrables en sistemas de información geográfica. Su principal fortaleza reside en la capacidad para ofrecer una \textbf{cartografía comparativa y replicable} de la vulnerabilidad, imprescindible para priorizar intervenciones, justificar asignaciones de recursos y evaluar, con series temporales, la evolución de las desigualdades.

Sin embargo, estas mismas características exponen sus límites. La selección de variables suele estar condicionada por la \textbf{disponibilidad de datos} más que por la pertinencia teórica; las decisiones sobre escalas y ponderaciones, aunque a veces justificadas técnicamente, implican opciones normativas que no siempre se hacen explícitas; la agregación en índices sintéticos tiende a \textbf{ocultar tensiones internas} entre dimensiones (por ejemplo, territorios con buena situación económica pero fuerte precariedad residencial, o viceversa). Además, cuando estos instrumentos se desconectan de la reflexión ontológica y normativa, la vulnerabilidad corre el riesgo de ser tratada como etiqueta numérica, desvinculada de las trayectorias históricas y de las experiencias que la sostienen.

La presente tesis asume tanto el potencial como las limitaciones de este enfoque. El IVEC se inscribe en esta familia metodológica en la medida en que construye un índice compuesto a partir de fuentes cuantitativas oficiales, pero lo hace articulando de forma explícita las decisiones de selección y agregación con los planos ontológico, normativo y epistemológico previamente desarrollados. La medición no se concibe como sustituto de la explicación, sino como \textbf{infraestructura cuantitativa} que debe permanecer abierta a revisiones conceptuales y contrastes cualitativos.

\subsection{Estudios de caso histórico--estructurales y análisis comparativo}\label{estudios-de-caso-histuxf3ricoestructurales-y-anuxe1lisis-comparativo}

Las metodologías asociadas a la orientación \textbf{estructural--crítica} se apoyan en diseños de investigación que privilegian el estudio de procesos en profundidad, la reconstrucción histórica y el análisis comparativo situado. Frente a la lógica de la medición extensiva, este enfoque pone el acento en \textbf{cómo se producen y sedimentan las configuraciones de vulnerabilidad} en territorios y colectivos concretos.

En el ámbito de los desastres, los análisis inspirados en el marco \emph{Pressure and Release} han desarrollado estudios de caso que reconstruyen, a lo largo del tiempo, las cadenas causales que vinculan causas profundas (modelos de desarrollo, regímenes de propiedad, estructuras estatales) con presiones dinámicas (migraciones forzadas, reformas económicas, transformaciones urbanas) y condiciones inseguras a escala local \citep{blaikie1994}. En geografía urbana y social, múltiples investigaciones han mostrado cómo procesos de desindustrialización, políticas de vivienda, dinámicas de gentrificación o decisiones infraestructurales reconfiguran la distribución espacial de la vulnerabilidad, produciendo barrios estigmatizados, periferias desconectadas o enclaves de pobreza concentrada.

Metodológicamente, estos estudios combinan \textbf{análisis documental}, explotación selectiva de estadísticas e indicadores, trabajo de campo cualitativo (entrevistas, grupos de discusión, observación participante) y, en ocasiones, técnicas de análisis espacial. Su objetivo no es representar exhaustivamente un sistema, sino \textbf{reconstruir mecanismos}: mostrar cómo una secuencia de decisiones políticas, un marco regulatorio o una configuración económica generan situaciones que hacen a ciertos grupos desproporcionadamente expuestos al daño, ya sea en forma de desastres súbitos o de procesos lentos de deterioro (pérdida de salud, precarización residencial, erosión de redes de apoyo).

El análisis comparativo añade una capa adicional, al contrastar casos con trayectorias distintas para identificar variaciones en los mecanismos de producción de vulnerabilidad y en las respuestas institucionales. Este tipo de diseños permite, por ejemplo, explicar por qué regiones con niveles similares de renta presentan configuraciones de vulnerabilidad muy distintas, o por qué determinadas políticas de bienestar logran amortiguar mejor los impactos de crisis económicas o ambientales.

Desde la perspectiva de esta tesis, estas metodologías ofrecen dos aportes clave. En primer lugar, proporcionan \textbf{marcos narrativos robustos} que permiten interpretar los patrones cuantitativos que emergen de índices e indicadores territoriales: los mapas dejan de ser meras distribuciones de valores para convertirse en huellas de procesos históricos localizables. En segundo lugar, recuerdan que la vulnerabilidad no es un estado sino un \textbf{proceso}, y que su comprensión exige atender a las secuencias de eventos y decisiones que la configuran. Aunque el diseño empírico del IVEC no se basa en estudios de caso extensivos, el diálogo con esta tradición metodológica resulta indispensable para interpretar sus resultados y para orientar futuras líneas de investigación cualitativa y comparativa en la Comunitat Valenciana.

\subsection{Metodologías sistémicas y análisis socioecológico}\label{metodologuxedas-sistuxe9micas-y-anuxe1lisis-socioecoluxf3gico}

Las metodologías asociadas a la orientación \textbf{compleja--sistémica} buscan captar la vulnerabilidad como comportamiento emergente de sistemas socioecológicos interdependientes. Aquí, el énfasis metodológico se sitúa en las \textbf{interacciones entre componentes} (humanos y no humanos), en la dinámica temporal y en la exploración de escenarios alternativos.

En términos concretos, esta familia incluye desde modelos dinámicos de simulación (sistemas de ecuaciones diferenciales, modelos basados en agentes) hasta análisis de redes de dependencia (por ejemplo, entre infraestructuras críticas o cadenas de suministro), pasando por metodologías de evaluación integrada que combinan información biofísica, socioeconómica e institucional. En el ámbito del cambio climático y la gestión del riesgo, este tipo de enfoques se ha utilizado para estimar impactos potenciales bajo distintos escenarios de emisiones, para analizar la robustez de sistemas urbanos o rurales frente a perturbaciones y para identificar \textbf{puntos de fragilidad} donde pequeñas intervenciones pueden producir efectos desproporcionados \citep{turner2003, adger2006}.

A estas herramientas se suman dispositivos metodológicos híbridos que combinan modelización y participación: talleres de escenarios, mapeos colectivos de riesgos, ejercicios de planificación colaborativa que incorporan conocimiento experto y local. En estos contextos, la vulnerabilidad se explora no solo como resultado de dinámicas pasadas, sino como \textbf{configuración en disputa}, sobre la que distintos actores proyectan expectativas, temores y estrategias de adaptación.

La principal fortaleza de estas metodologías reside en su capacidad para \textbf{representar interdependencias y no linealidades}, algo difícilmente abordable con enfoques puramente descriptivos o lineales. Permiten, por ejemplo, evaluar cómo la modificación de un uso del suelo afecta a la regulación hídrica, cómo dicha alteración interactúa con infraestructuras existentes y cómo todo ello repercute en la exposición de determinadas poblaciones a inundaciones o sequías. También posibilitan explorar el impacto combinado de políticas sectoriales que, consideradas aisladamente, podrían parecer neutras o incluso beneficiosas, pero que generan vulnerabilidades nuevas cuando se superponen.

Al mismo tiempo, estas metodologías implican desafíos importantes. La necesidad de simplificación para hacer gestionables los modelos abre espacios de \textbf{incertidumbre y decisión}: qué componentes incluir, qué relaciones considerar, qué procesos asumir como exógenos. Existe el riesgo de que la sofisticación técnica o la complejidad matemática generen una \textbf{opacidad epistemológica}, dificultando la participación de actores no expertos y reforzando asimetrías en la capacidad de definir qué escenarios se consideran plausibles o deseables. Además, cuando el foco se desplaza demasiado hacia el comportamiento del sistema, pueden diluirse las preguntas sobre justicia y responsabilidad en formulaciones genéricas sobre ``resiliencia'' o ``capacidad adaptativa''.

La tesis no desarrolla modelizaciones sistémicas complejas, pero sí se hace eco de algunas de sus intuiciones: la importancia de pensar la vulnerabilidad territorial en la Comunitat Valenciana en relación con dinámicas ambientales (especialmente hídricas y climáticas), la necesidad de considerar interacciones entre dimensiones (sociales, infraestructurales, ecológicas) al construir indicadores y la conveniencia de entender los resultados del IVEC no como fotografía estática, sino como \textbf{estado de un sistema en transformación}, susceptible de evolucionar bajo distintos escenarios de política pública y de cambio ambiental.

\subsection{Metodologías cualitativas e interpretativas}\label{metodologuxedas-cualitativas-e-interpretativas}

Las metodologías cualitativas e interpretativas prolongan, en el plano empírico, la orientación fenomenológica ya descrita. Su preocupación central es \textbf{captar la experiencia vivida de la vulnerabilidad}, los significados que los actores atribuyen al riesgo, las estrategias que despliegan para enfrentarlo y las formas en que las categorías institucionales de vulnerabilidad son apropiadas, negociadas o resistidas.

Estas metodologías despliegan un repertorio conocido ---entrevistas en profundidad, historias de vida, grupos de discusión, observación participante, etnografía de largo recorrido---, pero lo hacen con preguntas específicas: cómo se experimenta la inseguridad residencial en un barrio determinado; qué implica, biográfica y emocionalmente, ser catalogado como ``dependiente'' o ``en riesgo'' por un dispositivo administrativo; cómo se reconfiguran las redes de apoyo en contextos de crisis económica o de desastre ambiental; de qué manera influyen el género, la edad, el origen o la trayectoria migratoria en la vivencia de la vulnerabilidad y en el acceso a recursos de protección \citep{das2007life, farmer2004pathologies}.

La aportación metodológica de estos enfoques es doble. Por un lado, permiten \textbf{acceder a dimensiones que escapan a los indicadores cuantitativos}: sentimientos de desamparo o de estigmatización, percepciones de injusticia, experiencias de violencia institucional, formas de agencia y de resistencia que no se captan en registros administrativos. Por otro, funcionan como mecanismo de \textbf{validación y problematización} de diagnósticos construidos desde otras epistemologías: los mapas de vulnerabilidad pueden ser contrastados con las narrativas de quienes habitan los territorios señalados; las categorías técnicas pueden ser evaluadas a la luz de cómo son vividas por las personas a las que se aplican.

Estas metodologías tienen, sin embargo, limitaciones propias: su alcance espacial y poblacional suele ser acotado; los resultados son difícilmente generalizables en el sentido estadístico; la implicación del investigador en el trabajo de campo introduce formas de reflexividad y de co--producción que deben ser cuidadosamente explicitadas. No obstante, cuando se articulan con herramientas cuantitativas y estructurales, su capacidad de \textbf{densificar la comprensión} de la vulnerabilidad y de evitar su deshumanización resulta difícilmente sustituible.

Aunque la presente tesis no incorpora un trabajo cualitativo sistemático, sí reconoce la necesidad de que los resultados del IVEC y del análisis bibliométrico--semántico se abran, en fases posteriores, a procesos de contraste cualitativo y de deliberación con actores locales. La vulnerabilidad, tal como es medida y modelizada, debe permanecer disponible para ser interrogada por quienes la experimentan.

\subsection{Co--producción, participación e integración metodológica}\label{coproducciuxf3n-participaciuxf3n-e-integraciuxf3n-metodoluxf3gica}

En las últimas décadas han cobrado fuerza metodologías que tratan de \textbf{superar las fronteras rígidas entre expertos y legos}, así como las divisiones entre métodos cuantitativos, cualitativos y sistémicos. Bajo etiquetas diversas ---investigación--acción participativa, co--producción de conocimiento, planificación colaborativa--- se han desarrollado dispositivos que buscan incorporar a actores locales, movimientos sociales, administraciones y comunidades científicas en procesos conjuntos de diagnóstico y diseño de respuestas frente a la vulnerabilidad.

En estos contextos, la metodología no se concibe únicamente como herramienta de observación, sino como \textbf{espacio de negociación y de aprendizaje mutuo}. Los indicadores estadísticos se discuten con vecinos y organizaciones; los mapas de riesgo se coproducen con quienes viven en los territorios; los escenarios de futuro se construyen en talleres donde se contrastan saberes expertos y experiencias situadas. El resultado no es sólo un incremento de la calidad de la información, sino también una redistribución, aunque sea parcial, de la capacidad de definir qué se considera vulnerabilidad, qué problemas se priorizan y qué criterios se utilizan para evaluar las políticas.

La integración metodológica no se reduce, por tanto, a combinar técnicas en un diseño mixto, sino que implica una reflexión sobre \textbf{quién participa en la producción de conocimiento} y con qué capacidad de influencia. Este tipo de experiencias ha sido especialmente relevante en contextos de planificación urbana, gestión del agua, adaptación al cambio climático y diseño de políticas sociales territoriales, donde los diagnósticos técnicos han sido históricamente cuestionados por comunidades afectadas que no se reconocían en ellos.

Para la tesis, estas metodologías funcionan como horizonte más que como práctica plenamente desplegada. El diseño empírico del trabajo ---centrado en el análisis bibliométrico y semántico, por un lado, y en la construcción del IVEC, por otro--- no permite desarrollar un proceso participativo amplio, pero sí incorporar algunas de sus exigencias: transparencia en la construcción de indicadores, posibilidad de que actores institucionales y locales cuestionen y reinterpreten los resultados, y apertura a futuras fases de investigación--acción donde el IVEC pueda ser utilizado como \textbf{dispositivo de diálogo} más que como diagnóstico cerrado.

\subsection{Posicionamiento metodológico de la tesis}\label{posicionamiento-metodoluxf3gico-de-la-tesis}

En conjunto, el panorama metodológico de la vulnerabilidad social muestra un campo plural y tensionado. Las metodologías cuantitativas de medición y modelización ofrecen potencia descriptiva y comparativa, pero corren el riesgo de simplificar y despolitizar; los estudios histórico--estructurales aportan profundidad explicativa, aunque a menudo carecen de instrumentos para la comparación sistemática; las aproximaciones sistémicas permiten pensar interdependencias y umbrales, pero pueden diluir la responsabilidad y reforzar la opacidad técnica; los enfoques cualitativos e interpretativos densifican la comprensión de la experiencia, al precio de una menor extensividad y de una dependencia fuerte del trabajo de campo prolongado; las metodologías participativas y de co--producción abren el juego a otros actores, pero enfrentan limitaciones de escala, de recursos y de institucionalización.

La tesis se sitúa metodológicamente en una posición \textbf{articuladora y reflexiva}. En el plano empírico, combina dos estrategias principales: un \textbf{análisis bibliométrico y semántico} del campo de estudios sobre vulnerabilidad social, que permite evaluar la primera hipótesis relativa a la convergencia transdisciplinar, y la \textbf{construcción de un índice de vulnerabilidad estructural y contextual (IVEC)} para la Comunitat Valenciana, que operativiza el modelo conceptual desarrollado en los planos ontológico, normativo y epistemológico--teórico. Ambos dispositivos se apoyan en metodologías cuantitativas, pero incorporan como referencia continua las exigencias planteadas por las tradiciones estructurales, sistémicas e interpretativas.

Metodológicamente, ello se traduce en varias decisiones: selección de variables y de dimensiones del IVEC anclada en debates teóricos y no solo en disponibilidad de datos; explicitación de los criterios de normalización, ponderación y agregación; lectura crítica de los resultados a la luz de procesos históricos y configuraciones territoriales, y apertura a que futuras fases de investigación cualitativa y participativa contrasten, matizen o cuestionen los diagnósticos cuantitativos obtenidos. El análisis bibliométrico y semántico, por su parte, se concibe como una forma de \textbf{reflexividad disciplinar}: un modo de someter el propio campo de estudios a examen, identificando qué marcos epistemológicos y metodológicos han dominado, qué vacíos persisten y qué espacios de convergencia o de tensión se vislumbran.

Con este plano metodológico se completa el dispositivo conceptual que soporta el capítulo 2. Sobre esta base se formulan, en la sección siguiente, las hipótesis de la investigación y se despliega el análisis bibliométrico y semántico que permitirá evaluar, de manera empírica, la hipótesis de convergencia transdisciplinar en los estudios de vulnerabilidad social.

\cleardoublepage

\chapter{Marco teórico: enfoques y herramientas para el análisis de la vulnerabilidad social}\label{marco-teuxf3rico-enfoques-y-herramientas-para-el-anuxe1lisis-de-la-vulnerabilidad-social}

\section{Introducción}\label{introducciuxf3n-2}

El análisis de la vulnerabilidad social ha generado una pluralidad de enfoques que, aunque diversos en su origen disciplinar, convergen en la necesidad de superar las interpretaciones lineales y naturalistas del riesgo.
Esta multiplicidad de perspectivas refleja no solo una evolución epistemológica del campo, sino también una reconfiguración en las formas de abordar la relación entre estructuras sociales, dinámicas territoriales y respuestas institucionales.

El presente capítulo se propone construir un andamiaje teórico-metodológico sólido para sustentar el modelo IVEC (Índice de Vulnerabilidad Estructural y Contextual), combinando tres marcos conceptuales clave: el modelo PAR (Pressure and Release), el enfoque multidimensional MOVE y el enfoque funcional de Cardona.
Cada uno de ellos aporta elementos fundamentales para una lectura integrada y crítica de la vulnerabilidad social: causalidad estructural, articulación multidimensional y operacionalización institucional.

Lejos de plantear una síntesis homogénea, se adoptará una lógica de articulación por capas: epistemológica, conceptual, operativa y crítica.
Esta estrategia permite no solo comparar enfoques, sino también identificar sus puntos de complementariedad y tensión, base imprescindible para el diseño de herramientas analíticas y políticamente relevantes.
En coherencia con los objetivos de la investigación, se prestará especial atención a las implicaciones de cada marco en términos de justicia social, gobernanza del riesgo y capacidad transformadora.

\subsection{El modelo PAR: causalidad estructural y condiciones inseguras}\label{el-modelo-par-causalidad-estructural-y-condiciones-inseguras}

El modelo PAR (Pressure and Release), desarrollado por \citet{blaikie1994}, constituye una de las propuestas más influyentes en la crítica a la visión tecnocrática y despolitizada de los desastres.
Su planteamiento central es que el riesgo no es simplemente el resultado de amenazas físicas sobre poblaciones expuestas, sino la culminación de un proceso estructural en el que las condiciones de vida precarias y las desigualdades históricas se combinan con eventos extremos para producir desastres.
Desde esta perspectiva, el desastre no se entiende como una ``anomalía natural'', sino como una expresión condensada de injusticias sociales y territoriales.

A nivel conceptual, el modelo descompone el proceso de producción del riesgo en tres niveles: (1) causas raíz, (2) dinámicas de presión y (3) condiciones inseguras.
Las causas raíz incluyen elementos como el sistema económico dominante, la marginación política o la desposesión territorial.
Las dinámicas de presión aluden a procesos que traducen estas estructuras en contextos específicos ---como la urbanización informal, la precarización laboral o la deforestación--- mientras que las condiciones inseguras remiten a los factores inmediatos que determinan la vulnerabilidad física y social (materialidad de las viviendas, acceso a servicios, densidad institucional, etc.).

En términos operativos, el modelo PAR ha sido decisivo para promover metodologías participativas orientadas a desentrañar la génesis social del riesgo y facilitar su transformación.
Metodologías como el mapeo de riesgos con actores locales, los análisis históricos de vulnerabilidad o los diagnósticos participativos de capacidades emergen como aplicaciones concretas de este marco.

Desde una lectura crítica, sin embargo, el modelo PAR ha sido cuestionado por dos motivos principales.
Primero, por su dificultad para integrar dimensiones subjetivas y simbólicas de la vulnerabilidad ---como las percepciones de riesgo o los sistemas de creencias--- que son clave en contextos culturalmente diversos.
Segundo, por su escasa operatividad en contextos institucionales donde se requiere traducir diagnósticos complejos en instrumentos de planificación pública.
Estas limitaciones justifican su articulación con otros marcos, como el enfoque funcional de Cardona o el modelo multidimensional MOVE, que aportan criterios para la evaluación comparada y la construcción de indicadores aplicables a políticas públicas.

No obstante, el aporte del modelo PAR permanece central para la construcción del IVEC: su capacidad para visibilizar las condiciones estructurales de la vulnerabilidad y situarlas en marcos históricos, políticos y territoriales lo convierte en una referencia ineludible para cualquier análisis orientado a la justicia social.

\subsection{El marco MOVE: multidimensionalidad, integración y contexto}\label{el-marco-move-multidimensionalidad-integraciuxf3n-y-contexto}

El marco MOVE (\emph{Methods for the Improvement of Vulnerability Assessment in Europe}), surgido en el contexto del proyecto europeo del mismo nombre \citep{birkmann2013}, constituye uno de los esfuerzos más sistemáticos por articular una lectura multidimensional, contextual y operativa de la vulnerabilidad.
A diferencia del modelo PAR, que parte de una lógica causal estructural, el enfoque MOVE propone una matriz conceptual basada en dimensiones, componentes y factores que hacen posible la evaluación empírica de la vulnerabilidad en territorios específicos.

En su estructura básica, el modelo MOVE distingue entre tres componentes principales: exposición, susceptibilidad y falta de resiliencia.
Esta tríada funcional permite descomponer la vulnerabilidad en factores analíticamente diferenciables, cada uno de los cuales puede ser evaluado mediante indicadores específicos.
A nivel dimensional, el modelo identifica seis dimensiones clave: física, social, económica, institucional, cultural y ecológica.
La combinación de estos ejes analíticos posibilita un abordaje holístico, sensible al contexto y útil para múltiples escalas de intervención.

Desde un punto de vista epistemológico, el modelo MOVE se alimenta de tradiciones diversas: la economía política aporta la lectura estructural, la socio-ecología introduce la coevolución sociedad-naturaleza, la gestión del riesgo proporciona los marcos institucionales y normativos, y la adaptación climática añade la dimensión temporal y anticipatoria del análisis.
Esta convergencia teórica se traduce en una propuesta metodológica integradora, orientada tanto a la diagnosis como a la planificación.

En términos operativos, el modelo ha sido aplicado en estudios de caso sobre riesgos climáticos en diversas regiones europeas y extrapolado a contextos latinoamericanos.
Su capacidad para incorporar indicadores mixtos (cuantitativos y cualitativos) lo convierte en una herramienta versátil para estudios académicos, diagnósticos institucionales y formulación de políticas.
En el caso de la Comunitat Valenciana, su estructura multidimensional resulta especialmente pertinente para capturar la heterogeneidad territorial, institucional y socioeconómica de la región.

No obstante, el modelo MOVE no está exento de críticas.
Su énfasis en la operacionalización puede conducir a una tecnificación excesiva del análisis, con riesgo de despolitizar los procesos que producen vulnerabilidad.
Además, aunque incorpora dimensiones institucionales y culturales, estas suelen ser tratadas de forma superficial o proxy, lo que limita su potencial interpretativo.
De ahí la necesidad de articularlo con marcos como el PAR, que recuperan la densidad histórica y estructural del fenómeno, o el enfoque funcional de Cardona, que refuerza la dimensión institucional.

La propuesta IVEC retoma el modelo MOVE como base para la dimensión contextual de la vulnerabilidad, adaptando sus componentes a partir de una lectura crítica que privilegia la articulación territorial y el vínculo con las capacidades institucionales.

\section{El enfoque funcional de Cardona: operacionalización del riesgo y capacidades}\label{el-enfoque-funcional-de-cardona-operacionalizaciuxf3n-del-riesgo-y-capacidades}

El enfoque funcional propuesto por Cardona \citep{cardona2001, cardona2003, cardona2014} representa una contribución clave en el campo de la evaluación del riesgo, especialmente en el ámbito latinoamericano.
Su originalidad reside en haber construido un puente entre la conceptualización teórica de la vulnerabilidad y su traducción en herramientas e indicadores aplicables a la planificación y la gestión pública.
Frente a modelos centrados en la descripción o la causalidad, el enfoque de Cardona se sitúa claramente en la dimensión operativa del análisis del riesgo.

El punto de partida de su propuesta es una comprensión funcional de la vulnerabilidad: no como una condición estática, sino como un atributo relacional, contingente a la interacción entre amenaza, exposición y capacidad de respuesta.
Esta perspectiva permite descomponer el riesgo en cuatro componentes ---amenaza, exposición, vulnerabilidad y capacidad---, cada uno de los cuales puede ser evaluado mediante indicadores específicos.
Esta lógica ha sido la base del Índice de Riesgo Holístico (IRH), desarrollado por Cardona y ampliamente utilizado en América Latina por organismos como CEPAL y BID.

Epistemológicamente, su enfoque incorpora elementos de las ciencias naturales, aplicadas y sociales, proponiendo una clasificación funcional de las disciplinas en función de su papel dentro del ciclo del riesgo.
Así, las ciencias naturales se encargan de caracterizar las amenazas; las ciencias aplicadas, de proponer soluciones técnicas; y las ciencias sociales, de identificar factores de vulnerabilidad y capacidades institucionales.
Esta matriz disciplinar permite integrar distintos saberes en un enfoque pragmático y orientado a la acción pública.

Desde una perspectiva normativa, Cardona enfatiza el rol de las capacidades como moduladores del riesgo: no basta con conocer la exposición o la peligrosidad, sino que es necesario evaluar las capacidades institucionales, comunitarias y normativas que determinan la respuesta ante los eventos adversos.
Esta dimensión ha sido clave para el desarrollo de índices como el Índice de Capacidades Institucionales (ICI) o la propuesta de escenarios de riesgo combinados mediante lógica bayesiana.

Críticamente, su enfoque ha sido cuestionado por su énfasis cuantitativo, que puede invisibilizar dimensiones subjetivas o cualitativas del riesgo, como las percepciones, las memorias históricas o las formas de resistencia.
Además, la integración de capacidades institucionales, aunque conceptualmente potente, requiere de datos que no siempre están disponibles o actualizados, lo que limita su aplicabilidad en ciertos contextos.

Pese a estas limitaciones, el enfoque funcional de Cardona ofrece un marco indispensable para vincular el análisis de vulnerabilidad con la toma de decisiones institucionales.
En esta tesis, sus aportes son recuperados en dos sentidos: como base para la conceptualización de la dimensión de capacidades del modelo IVEC, y como criterio operativo para la construcción de indicadores aplicables a la planificación territorial en la Comunitat Valenciana.

\section{Articulación crítica de enfoques: hacia una comprensión integrada de la vulnerabilidad}\label{articulaciuxf3n-cruxedtica-de-enfoques-hacia-una-comprensiuxf3n-integrada-de-la-vulnerabilidad}

Tras haber examinado los tres enfoques clave ---PAR, MOVE y el modelo funcional de Cardona---, es posible avanzar hacia una articulación conceptual y metodológica que permita integrar sus aportes, superando las limitaciones propias de cada uno.
Esta sección tiene como objetivo construir un marco analítico coherente y útil para la formulación del modelo IVEC, combinando profundidad estructural, operatividad técnica y sensibilidad territorial.

Desde una perspectiva epistemológica, el modelo PAR introduce una lógica causal que permite rastrear las raíces históricas y políticas de la vulnerabilidad.
En cambio, MOVE parte de una matriz dimensional que permite desagregar el fenómeno en componentes y dimensiones, articulando exposición, susceptibilidad y resiliencia.
Por su parte, Cardona opera en un registro funcional, orientado a la medición y gestión del riesgo mediante indicadores, lo que facilita su implementación en políticas públicas.

Estas tres lógicas ---causal, dimensional y funcional--- no son excluyentes, sino que pueden combinarse en un marco integrado.
En este sentido, proponemos entender la vulnerabilidad como un campo de tensiones estructurales (PAR), condiciones medibles y localizadas (MOVE), y capacidades institucionales y sociales (Cardona).
Esta tríada permite abordar el fenómeno desde múltiples escalas y con diferentes niveles de abstracción.

En términos operativos, esta integración implica una lectura estratificada de los indicadores.
Los indicadores estructurales, inspirados en el PAR, permiten captar las desigualdades históricas y las causas raíz.
Los indicadores contextuales, derivados de MOVE, permiten mapear las condiciones específicas de exposición y susceptibilidad.
Finalmente, los indicadores de capacidades, tomados del enfoque de Cardona, permiten evaluar el margen de maniobra institucional y social ante eventos adversos.

Este enfoque integrado también permite responder a una crítica habitual en los estudios de vulnerabilidad: la fragmentación excesiva y la falta de comparabilidad entre enfoques.
Al articular causalidad, medición y gobernanza, se construye un marco que permite tanto el diagnóstico riguroso como la formulación de estrategias de intervención.

Además, esta integración responde a una exigencia normativa emergente: los marcos internacionales (Sendai, IPCC, Agenda 2030) reclaman herramientas capaces de conectar vulnerabilidad estructural, riesgo climático y políticas de resiliencia.
El modelo IVEC, basado en esta articulación, busca responder a esta exigencia, desde una perspectiva situada y territorialmente sensible.

\section{Convergencia metodológica hacia el modelo IVEC}\label{convergencia-metodoluxf3gica-hacia-el-modelo-ivec}

La revisión crítica de los modelos PAR, MOVE y Cardona ha permitido identificar un conjunto de principios teóricos y herramientas metodológicas que, lejos de ser incompatibles, pueden ser articulados de manera complementaria.
Esta convergencia metodológica constituye la base del Índice de Vulnerabilidad Estructural y Contextual (IVEC), una propuesta integrada que combina profundidad estructural, operatividad técnica y relevancia política.

El modelo PAR aporta una perspectiva estructural y crítica, al situar la vulnerabilidad en el marco de relaciones de poder históricas y procesos sociales de larga duración.
Su lógica causal permite construir narrativas explicativas sobre el origen del riesgo, evitando reduccionismos naturalistas o tecnocráticos.
Esta perspectiva resulta clave para incorporar en el IVEC una dimensión de vulnerabilidad estructural, que dé cuenta de las desigualdades persistentes que configuran condiciones de inseguridad.

El marco MOVE contribuye con una arquitectura conceptual y operativa capaz de desagregar la vulnerabilidad en dimensiones analíticas: exposición, susceptibilidad y falta de resiliencia.
Este andamiaje es fundamental para construir indicadores multidimensionales que, respetando la complejidad del fenómeno, permitan su operacionalización en análisis empíricos y herramientas de gestión.

El enfoque funcional de Cardona, finalmente, introduce una lógica pragmática orientada a la intervención.
Su propuesta de construir índices a partir de dimensiones medibles y su énfasis en la evaluación de capacidades institucionales y sociales, ofrece un puente entre el diagnóstico y la acción.
En el contexto valenciano, esta aproximación resulta especialmente pertinente, dada la necesidad de vincular el conocimiento sobre vulnerabilidad con estrategias concretas de gobernanza del riesgo.

Esta estructura permitirá, en los capítulos siguientes, avanzar en el diseño metodológico y la validación empírica del IVEC.
Además, sienta las bases para una lectura integrada del riesgo social, que no se limite a describir patrones de vulnerabilidad, sino que oriente decisiones de política pública informadas, justas y eficaces.

\begin{figure}

{\centering \includegraphics[width=1\linewidth]{figures/IVEC_marco_teorico_soft} 

}

\caption[Arquitectura epistemotécnica del modelo IVEC]{Arquitectura epistemotécnica del modelo IVEC y su extensión hacia el riesgo  \newline \textit{Nota.} Las capacidades reactivas (CR) se activan en situaciones de alerta o en escenarios prospectivos. \newline \textit{Fuente:} Elaboración propia.}\label{fig:ivec-flujo}
\end{figure}

\section{Conclusión del capítulo}\label{conclusiuxf3n-del-capuxedtulo}

El presente capítulo ha propuesto una lectura articulada de tres enfoques teóricos y metodológicos que, desde perspectivas distintas, han contribuido decisivamente a la comprensión contemporánea de la vulnerabilidad y el riesgo.
A través del modelo PAR, el marco MOVE y el enfoque funcional de Cardona, se ha recorrido un itinerario que abarca desde la causalidad estructural hasta la operatividad institucional, pasando por una conceptualización multidimensional y contextualizada de la vulnerabilidad social.

La articulación entre estos enfoques no pretende diluir sus diferencias ni proponer una síntesis simplista, sino reconocer que sus respectivas fortalezas ---la profundidad analítica del PAR, la sistematicidad del MOVE y la aplicabilidad del enfoque de Cardona--- permiten construir un marco más robusto para el análisis del riesgo en contextos complejos como el de la Comunitat Valenciana.

Además, la doble clasificación propuesta ---epistemológica (MOVE) y funcional (Cardona)--- ha permitido visibilizar las contribuciones disciplinares desde una mirada integrada, capaz de conjugar la producción de conocimiento con la acción transformadora.
Esta articulación resulta clave para avanzar hacia un modelo de gobernanza del riesgo que no se limite a la gestión técnica de amenazas, sino que incorpore la justicia social, la equidad territorial y la participación institucional como principios orientadores.

De este modo, el capítulo sienta las bases conceptuales y metodológicas para la construcción del Índice de Vulnerabilidad Estructural y Contextual (IVEC), que se abordará en los capítulos siguientes.
Esta propuesta no solo pretende describir patrones de desigualdad y fragilidad social, sino también orientar intervenciones públicas más eficaces, inclusivas y sostenibles.
En definitiva, el IVEC busca contribuir a una gobernanza del riesgo que reconozca la vulnerabilidad no como una condición inherente a ciertos grupos, sino como el resultado de procesos estructurales que pueden y deben ser transformados.

\cleardoublepage

\chapter{Marco normativo e institucional: gobernanza del riesgo y capacidades para la resiliencia}\label{marco-normativo-e-institucional-gobernanza-del-riesgo-y-capacidades-para-la-resiliencia}

\section{Introducción}\label{introducciuxf3n-3}

Este capítulo cumple una función doblemente estratégica en la arquitectura de la tesis: por un lado, prolonga la línea argumental desarrollada en los capítulos teóricos anteriores (especialmente en relación con los marcos PAR, MOVE y Cardona), y por otro, prepara el terreno metodológico para la operacionalización de la vulnerabilidad en clave institucional. Se trata, por tanto, de un capítulo híbrido ---teórico y operativo--- que examina los marcos normativos y las políticas públicas implicadas en la gestión del riesgo, no solo como dispositivos técnicos, sino como expresiones de modelos de gobernanza. A partir del estudio de la legislación internacional, nacional y autonómica, se analiza críticamente cómo se institucionaliza la categoría de vulnerabilidad y qué capacidades se promueven (o se omiten) desde la acción pública. En consecuencia, este capítulo persigue los siguientes objetivos:

\begin{itemize}
\tightlist
\item
  \textbf{Identificar} los principales marcos internacionales y europeos que orientan la gobernanza del riesgo, valorando su recepción en contextos normativos descentralizados como el español.
\item
  \textbf{Caracterizar} la arquitectura normativa estatal, autonómica y local vinculada a la gestión del riesgo, el territorio y las políticas sociales, prestando atención a sus solapamientos, vacíos y desconexiones.
\item
  \textbf{Revisar críticamente} el uso institucional de la categoría ``vulnerable'', distinguiendo entre enfoques asistenciales, estigmatizantes y estructurales.
\item
  \textbf{Establecer} una tipología de capacidades institucionales relevantes para la resiliencia, a partir del cruce entre marcos normativos y políticas públicas sectoriales.
\item
  \textbf{Fundamentar} conceptualmente la inclusión de dimensiones institucionales y normativas en el diseño del índice IVEC, especialmente en lo referido a la medición de la vulnerabilidad estructural y contextual.
\end{itemize}

Estos objetivos articulan una lectura crítica y propositiva del entramado normativo que enmarca las respuestas públicas frente al riesgo, situando a la gobernanza del riesgo como un campo de disputa entre racionalidades técnicas, lógicas de reconocimiento y horizontes de justicia social.

\section{La gobernanza internacional del riesgo: principios, tensiones y traducciones}\label{la-gobernanza-internacional-del-riesgo-principios-tensiones-y-traducciones}

\subsection{Del riesgo natural a la resiliencia sistémica: evolución de marcos globales}\label{del-riesgo-natural-a-la-resiliencia-sistuxe9mica-evoluciuxf3n-de-marcos-globales}

\begin{itemize}
\tightlist
\item
  Marco de Hyogo (2005--2015)
\item
  Marco de Sendai (2015--2030)
\item
  Agenda 2030 y ODS
\item
  Acuerdo de París
\item
  Informes IPCC
\end{itemize}

\subsection{Conceptualización de la vulnerabilidad en los marcos internacionales}\label{conceptualizaciuxf3n-de-la-vulnerabilidad-en-los-marcos-internacionales}

\begin{itemize}
\tightlist
\item
  De la vulnerabilidad física a la social
\item
  Reconocimiento del carácter estructural/contextual
\item
  Definiciones operativas y límites
\end{itemize}

\subsection{Riesgos de una institucionalización funcionalista}\label{riesgos-de-una-institucionalizaciuxf3n-funcionalista}

\begin{itemize}
\tightlist
\item
  Críticas desde Fassin, Watts, Martínez
\item
  La vulnerabilidad como etiqueta de gobernabilidad
\item
  La ambivalencia del enfoque de resiliencia
\end{itemize}

\section{La arquitectura normativa en España: competencias y desconexiones}\label{la-arquitectura-normativa-en-espauxf1a-competencias-y-desconexiones}

\subsection{Nivel estatal}\label{nivel-estatal}

\begin{itemize}
\tightlist
\item
  Protección Civil y emergencias (Ley 17/2015)
\item
  Ley de Cambio Climático y Transición Energética
\item
  Sistema Nacional de Servicios Sociales
\end{itemize}

\subsection{Nivel autonómico (Comunitat Valenciana)}\label{nivel-autonuxf3mico-comunitat-valenciana}

\begin{itemize}
\tightlist
\item
  Ley Valenciana de Servicios Sociales Inclusivos
\item
  Estrategias de adaptación y reducción del riesgo
\item
  Planes territoriales y sectoriales
\end{itemize}

\subsection{Nivel local y supramunicipal}\label{nivel-local-y-supramunicipal}

\begin{itemize}
\tightlist
\item
  PATRICOVA y planes urbanísticos
\item
  Mancomunidades, consorcios y servicios sociales de base
\item
  Limitaciones de escala y recursos
\end{itemize}

\subsection{Solapamientos y vacíos normativos}\label{solapamientos-y-vacuxedos-normativos}

\begin{itemize}
\tightlist
\item
  Desconexión entre políticas sociales y de riesgo
\item
  Falta de integración territorial e institucional
\end{itemize}

\section{Clasificación institucional de la vulnerabilidad: entre el ser y el estar}\label{clasificaciuxf3n-institucional-de-la-vulnerabilidad-entre-el-ser-y-el-estar}

\subsection{El lenguaje institucional: definiciones y criterios}\label{el-lenguaje-institucional-definiciones-y-criterios}

\begin{itemize}
\tightlist
\item
  Grupos vulnerables / en situación de vulnerabilidad
\item
  Instrumentos normativos y administrativos
\end{itemize}

\subsection{Crítica a los enfoques asistencialistas}\label{cruxedtica-a-los-enfoques-asistencialistas}

\begin{itemize}
\tightlist
\item
  De la protección al control
\item
  Invisibilización de causas estructurales
\end{itemize}

\subsection{Hacia una categoría relacional y situada}\label{hacia-una-categoruxeda-relacional-y-situada}

\begin{itemize}
\tightlist
\item
  Aportes desde la sociología crítica y la epidemiología social
\item
  Implicaciones para la acción pública
\end{itemize}

\section{Capacidades institucionales para la resiliencia: tipología y bases normativas}\label{capacidades-institucionales-para-la-resiliencia-tipologuxeda-y-bases-normativas}

\subsection{Anticipación: planificación, evaluación y alerta}\label{anticipaciuxf3n-planificaciuxf3n-evaluaciuxf3n-y-alerta}

\begin{itemize}
\tightlist
\item
  Normas que refuerzan la vigilancia y el diagnóstico
\item
  Evaluación de riesgos y escenarios prospectivos
\end{itemize}

\subsection{Respuesta: coordinación, flexibilidad y proximidad}\label{respuesta-coordinaciuxf3n-flexibilidad-y-proximidad}

\begin{itemize}
\tightlist
\item
  Dispositivos operativos, redes interinstitucionales
\item
  Protocolos, recursos y mecanismos de activación
\end{itemize}

\subsection{Recuperación: reconstrucción, aprendizaje e inclusión}\label{recuperaciuxf3n-reconstrucciuxf3n-aprendizaje-e-inclusiuxf3n}

\begin{itemize}
\tightlist
\item
  Estrategias post-emergencia
\item
  Participación social y memoria del riesgo
\end{itemize}

\section{Aportes al modelo IVEC: institucionalización de la vulnerabilidad y dimensiones operativas}\label{aportes-al-modelo-ivec-institucionalizaciuxf3n-de-la-vulnerabilidad-y-dimensiones-operativas}

\subsection{Fundamentación normativa de la vulnerabilidad estructural}\label{fundamentaciuxf3n-normativa-de-la-vulnerabilidad-estructural}

\begin{itemize}
\tightlist
\item
  Políticas que (no) abordan causas profundas
\item
  Relevancia de los marcos regulatorios en la construcción del riesgo
\end{itemize}

\subsection{Marco institucional para la vulnerabilidad contextual}\label{marco-institucional-para-la-vulnerabilidad-contextual}

\begin{itemize}
\tightlist
\item
  Dimensiones legales y administrativas que configuran la exposición diferencial
\end{itemize}

\subsection{Capacidades institucionales como expresión de la resiliencia pública}\label{capacidades-institucionales-como-expresiuxf3n-de-la-resiliencia-puxfablica}

\begin{itemize}
\tightlist
\item
  De la medición a la acción: diseño de indicadores IVEC
\item
  Posibilidades y límites de intervención institucional
\end{itemize}

\section{Conclusión}\label{conclusiuxf3n}

\begin{itemize}
\tightlist
\item
  Recapitulación crítica del marco institucional vigente
\item
  Tensiones entre normatividad y justicia social
\item
  Conexión con el diseño metodológico del capítulo 5
\end{itemize}

\cleardoublepage

\chapter{Metodología}\label{metodologuxeda}

\section{Introducción}\label{introducciuxf3n-4}

El estudio de la vulnerabilidad social y el riesgo requiere hoy, más que nunca, un abordaje que combine el rigor técnico con una sensibilidad analítica hacia las desigualdades sociales.
La presente investigación asume como punto de partida la potencia que ofrecen los enfoques cuantitativos y las herramientas técnico-administrativas desarrolladas en los últimos años, tanto en el ámbito europeo como en el estatal, en la gestión autonómica y local.
Sin embargo, se reconoce que los indicadores, por sí mismos, no son neutros: su valor reside en la capacidad de interpretarlos a la luz de los procesos sociales que configuran el territorio y las trayectorias diferenciales de las comunidades.

El capítulo metodológico se articula, en este sentido, como un espacio de encuentro entre la tradición cuantitativista ---con su capacidad de producir conocimiento comparable, replicable y operativo--- y la necesidad de incorporar una perspectiva reflexiva que sitúe la medición en el contexto de las desigualdades socioeconómicas que atraviesan la Comunitat Valenciana.

El desarrollo del Índice de Vulnerabilidad Estructural y Contextual (IVEC) se apoya en la solvencia de los marcos europeos (MOVE) y en la sofisticación técnica de los sistemas de información poblacional, como el Análisis Poblacional Segmentado Integrado y Geográfico (APSIG).
Se reconoce el mérito de estas metodologías por su aporte a la estandarización, la comparabilidad y la planificación pública basada en evidencia.
El diseño metodológico se beneficia de la robustez de técnicas estadísticas (como el Análisis de Componentes Principales -- PCA y k-means), de la transparencia de fuentes abiertas y del rigor documental propio de la planificación sanitaria y territorial.

Mi contribución consiste en expandir las posibilidades de la lógica técnico-cuantitativa, no solo para medir la vulnerabilidad, sino también para interpretarla críticamente en relación con las desigualdades sociales.
El IVEC es concebido aquí como una herramienta que, al tiempo que mantiene la precisión analítica y la reproducibilidad exigida por los estándares internacionales, permite identificar y analizar desigualdades socioeconómicas que las estadísticas convencionales suelen agregar y, en consecuencia, simplificar.
Es precisamente en la lectura e interpretación reflexiva de estos indicadores donde se sitúa el valor añadido de este trabajo.

En suma, este capítulo metodológico no renuncia al rigor técnico ni a la operatividad, sino que los pone al servicio de una lectura más afinada de la vulnerabilidad social, alineando la medición con el objetivo último de fundamentar políticas de prevención de riesgos y de cohesión social que respondan a los retos reales del siglo XXI.
De este modo, la metodología no solo cumple funciones analíticas, sino que orienta, en última instancia, la formulación de políticas públicas más justas y situadas.

\section{5. 2. El modelo integral y transdisciplinar para la estimación de la vulnerabilidad social}\label{el-modelo-integral-y-transdisciplinar-para-la-estimaciuxf3n-de-la-vulnerabilidad-social}

Antes de presentar en detalle las características del modelo IVEC, resulta pertinente enunciar cuáles serían los atributos deseables de un modelo integral y transdisciplinar para la estimación de la vulnerabilidad social y territorial.
La literatura especializada coincide en que, pese a los avances teóricos y metodológicos alcanzados, la construcción de un marco verdaderamente integrador y abierto a la transdisciplinariedad sigue siendo un reto pendiente en el campo de los estudios de riesgo y vulnerabilidad \citep{rufat2019-slowburn, birkmann2013}.

En este contexto, conviene distinguir entre enfoques multidisciplinares, interdisciplinares y transdisciplinares.
Según \citet{choipak2006-multi}, la multidisciplina yuxtapone conocimientos sin integrarlos; la interdisciplinariedad busca coordinar vínculos entre disciplinas, pero mantiene sus fronteras; y la transdisciplinariedad trasciende esos límites para generar marcos analíticos nuevos que integran ciencias naturales, sociales y de la salud en diálogo con las humanidades.
\citet{klein2010-inter} resume esta diferencia señalando que la multidisciplina es aditiva, la interdisciplinariedad interactiva y la transdisciplinariedad holística.
En línea con \citet{nicolescu2002-trans} y \citet{morin1992-meto}, esta última implica atravesar y recombinar saberes, abriendo la posibilidad de co-producción de conocimiento con actores sociales.

Este horizonte transdisciplinar constituye la única vía adecuada para abordar fenómenos complejos como la vulnerabilidad social y territorial, configurados en la intersección de dimensiones heterogéneas ---económicas, ambientales, sanitarias, institucionales y culturales--- que no pueden comprenderse desde una sola disciplina ni mediante una simple suma de ellas {[}\citet{alvargonzalez2011-multi}; jahn2012-trans{]}.

Sobre esta base, resulta necesario definir un conjunto de criterios normativos que orienten el diseño de cualquier índice de vulnerabilidad con vocación integral y transdisciplinar.
Estos criterios garantizan la solidez conceptual, la pertinencia social y la utilidad práctica de la herramienta, y sirven como punto de referencia frente al cual se evaluará el modelo IVEC.
A continuación, se desarrollan de forma sistemática en doce subsecciones específicas.

\subsection{Fundamentación teórica y conceptual sólida}\label{fundamentaciuxf3n-teuxf3rica-y-conceptual-suxf3lida}

La construcción de un índice con vocación transdisciplinar comienza por garantizar un marco conceptual compartido.
Para ello es necesario que todos los actores implicados ---ingenieros, sociólogos, urbanistas, personal de emergencia o representantes vecinales--- hablen un mismo idioma.
Esto requiere consensuar definiciones precisas de conceptos básicos como vulnerabilidad, exposición, riesgo y resiliencia.
Sin este paso, cada disciplina tendería a interpretar los términos desde sus propias tradiciones, generando ambigüedades y dificultando la comparación de resultados.

Una vez establecido este glosario común, el índice debe reflejar de manera coherente las distintas escalas en las que se manifiesta la vulnerabilidad: la estructura económica y política de un territorio, el contexto social de cada entorno, las características individuales de las personas y las dinámicas de hogares y comunidades.
Mantener estas capas conectadas es fundamental para evitar simplificaciones, ya que los factores que explican el riesgo rara vez se concentran en un solo nivel.

Esta claridad conceptual, además, debe guiar la selección de variables.
No se trata de aplicar una lista genérica a cualquier situación, sino de elegir indicadores que realmente traduzcan las amenazas específicas de interés: por ejemplo, la proximidad a cauces y la altura sobre el nivel del suelo si el riesgo es la inundación, o el grado de aislamiento térmico de las viviendas cuando se trata de olas de calor.
De este modo, el índice se mantiene fiel a su propósito y evita medir aspectos irrelevantes o redundantes.

\subsection{Co-definición del problema y orientación a la solución}\label{co-definiciuxf3n-del-problema-y-orientaciuxf3n-a-la-soluciuxf3n}

El valor de un índice de vulnerabilidad depende de su capacidad para responder a preguntas que sean relevantes para quienes gestionan los riesgos y para quienes los sufren.
Por ello, la formulación del problema no debe quedar restringida al ámbito académico, sino surgir de un proceso participativo en el que se involucren investigadores, profesionales de la administración y actores sociales.

Este diálogo permite acordar qué fenómenos se consideran prioritarios y por qué, asegurando que el índice mida lo que realmente importa y no únicamente lo que resulta más sencillo de calcular.De este modo, la herramienta se convierte en un instrumento útil para la política pública, alineado tanto con los objetivos de reducción del riesgo como con los principios de equidad social y justicia territorial.

\subsection{Integración de saberes heterogéneos y datos mixtos}\label{integraciuxf3n-de-saberes-heteroguxe9neos-y-datos-mixtos}

El enfoque transdisciplinar requiere reconocer la validez de fuentes de conocimiento muy diversas.
Los registros administrativos, las bases estadísticas y los sensores ofrecen datos cuantitativos precisos sobre población, territorio o infraestructura.
A la vez, los talleres comunitarios, las entrevistas o las narrativas locales aportan una comprensión cualitativa del modo en que se experimenta la vulnerabilidad.
Ninguno de estos enfoques, por sí solo, captura la complejidad del fenómeno.

El reto metodológico consiste en integrar ambas perspectivas, traduciendo los resultados a un lenguaje común sin perder sus particularidades.
Así, se obtiene un diagnóstico más completo, que refleja tanto los patrones cuantificables como la textura social que los explica.
Este esfuerzo de integración no solo enriquece el análisis, sino que también otorga legitimidad social al índice, al reconocer la experiencia de quienes viven el riesgo en primera persona.

\subsection{Sensibilidad multiescalar}\label{sensibilidad-multiescalar}

Un índice de vulnerabilidad gana relevancia práctica cuando sus resultados pueden interpretarse en distintos niveles de análisis: desde la escala micro ---la persona o el hogar--- hasta agrupaciones meso y macro como el municipio, la comarca o la comunidad autónoma.
Esta flexibilidad permite identificar focos de riesgo muy localizados y, al mismo tiempo, ofrecer insumos comparables para orientar políticas de alcance más amplio.

Sin embargo, el cambio de escala plantea un riesgo metodológico clásico: el sesgo ecológico.
Este error se produce cuando se confunden los promedios de un territorio con la realidad de cada uno de sus habitantes.
Así, por ejemplo, una sección censal con renta media baja no implica que todos sus residentes sean económicamente precarios.
Para evitarlo, es fundamental utilizar técnicas que vinculen la información individual con la agregada ---como los modelos multinivel o la desagregación estadística--- y, siempre que sea posible, trabajar con microdatos georreferenciados.

De esta manera, el índice respeta las particularidades de cada escala y asegura que las conclusiones obtenidas en un nivel no distorsionen lo que ocurre en otro.

\subsection{5.2.5 Inclusión explícita de capacidades y resiliencia en marcos de riesgo y desastres}\label{inclusiuxf3n-expluxedcita-de-capacidades-y-resiliencia-en-marcos-de-riesgo-y-desastres}

Medir la vulnerabilidad únicamente a partir de las carencias socioeconómicas, las características demográficas o de la exposición a peligros ofrece una imagen incompleta.
Para comprender cómo un territorio enfrenta los desastres es imprescindible considerar también sus capacidades y resiliencia.

Por un lado, se encuentran las capacidades adaptativas, como la existencia de planes de prevención, sistemas de alerta temprana o normativas que reducen el riesgo antes de que ocurra el evento.
Por otro, las capacidades reactivas, es decir, los recursos que facilitan la respuesta inmediata y la recuperación posterior: servicios de emergencia bien dotados, fondos de contingencia o protocolos de reconstrucción.

Estos indicadores deben abarcar tanto los recursos formales ---infraestructuras, políticas, presupuestos--- como aquellos menos visibles pero igualmente decisivos, como el capital social y las redes comunitarias de apoyo.
Solo al integrar ambas dimensiones se obtiene una visión más completa, que no solo señala quién es más vulnerable, sino también quién dispone de más herramientas para afrontar y superar la crisis.

\subsection{5.2.6 Parsimonia y robustez estadística}\label{parsimonia-y-robustez-estaduxedstica}

Un índice de vulnerabilidad solo es útil si resulta manejable y comprensible tanto para especialistas como para decisores públicos.
Para ello, conviene aplicar el principio de parsimonia: seleccionar únicamente las variables estrictamente necesarias para explicar el fenómeno, evitando duplicidades o correlaciones excesivas que generen ruido más que información \citep{hinkel2011-indicators}.
En este sentido, la parsimonia no implica simplificación arbitraria, sino una búsqueda deliberada del modelo más claro y eficiente que conserve capacidad explicativa.

La robustez estadística, por su parte, exige que la combinación de indicadores se sustente en métodos contrastados.
Herramientas como el análisis de componentes principales (PCA) o los análisis factoriales permiten identificar qué dimensiones concentran mayor varianza y, por tanto, justifican un mayor peso en el cálculo del índice \citep{greco2019-composite}.
Complementariamente, se pueden incorporar consultas a expertos o procesos participativos que, desde el conocimiento del contexto, orienten la ponderación de variables.

Finalmente, un índice parsimonioso y robusto ha de ser sometido a pruebas de sensibilidad.
Se trata de recalcularlo bajo variaciones controladas ---por ejemplo, modificando ligeramente los pesos o introduciendo escenarios alternativos de datos--- para observar si las posiciones relativas de los territorios cambian de manera drástica.
La estabilidad bajo estas pruebas refuerza la confianza en que el índice refleja patrones reales y no resultados espurios.

\subsection{5.2.7 Validación y verificación empírica}\label{validaciuxf3n-y-verificaciuxf3n-empuxedrica}

Ningún índice puede considerarse fiable si no se contrasta con la realidad observable.
Por ello, resulta imprescindible validar sus resultados frente a eventos ocurridos en el pasado.
Una primera estrategia consiste en comparar si los territorios que obtienen puntuaciones elevadas en el índice coinciden con aquellos que sufrieron mayores impactos en desastres recientes \citep{rufat2019-validation}.
Este contraste con datos empíricos constituye un test de capacidad predictiva.

Además, la validación se enriquece mediante la triangulación con otras fuentes: la comparación con índices consolidados a nivel europeo o internacional, la incorporación de diagnósticos locales y los testimonios cualitativos de las comunidades afectadas \citep{birkmann2013}.
Cada contraste aporta matices adicionales y permite detectar tanto redundancias como innovaciones que el índice introduce respecto a enfoques previos.

La verificación empírica también incluye publicar métricas de ajuste ---como correlaciones o coeficientes de determinación--- que transparenten la relación entre los indicadores empleados y los patrones observados.
Someter el índice a pruebas de robustez con datos atípicos o incompletos, además, ayuda a anticipar sus limitaciones y a diseñar mecanismos de mejora continua.

En síntesis, la validación empírica no es un procedimiento accesorio, sino un criterio central que asegura que el índice no se limite a un ejercicio estadístico, sino que se convierta en una herramienta útil para la gestión del riesgo y la formulación de políticas públicas.

\subsection{5.2.8 Actualización y comparabilidad temporal}\label{actualizaciuxf3n-y-comparabilidad-temporal}

Un índice de vulnerabilidad solo mantiene su pertinencia si es capaz de adaptarse a los cambios sociales, económicos y territoriales que afectan a la población.
De lo contrario, corre el riesgo de convertirse en una fotografía desactualizada que pierde valor para la gestión del riesgo.
Por ello, es imprescindible definir desde el inicio un calendario de actualización periódica que asegure la incorporación de datos recientes y, al mismo tiempo, preserve la comparabilidad con versiones anteriores \citep{birkmann2013}.

El principio de comparabilidad temporal exige que cada nueva edición del índice documente con claridad los cambios metodológicos, de fuentes o de variables.
Ello puede lograrse mediante un sistema de control de versiones que permita rastrear la evolución del índice y garantizar la trazabilidad de los resultados.
La transparencia en este proceso resulta clave para que los usuarios ---investigadores, administraciones o comunidades locales--- puedan interpretar adecuadamente las variaciones en los valores de vulnerabilidad a lo largo del tiempo.

Asimismo, conviene considerar la integración de variables dinámicas que capturan procesos en constante transformación, como la movilidad diaria de la población, la evolución demográfica o las tendencias del mercado laboral.
Incorporar este tipo de información no solo aporta mayor realismo al índice, sino que permite seguir la ``película'' de cómo la vulnerabilidad evoluciona con la sociedad y el territorio, más allá de la simple ``fotografía'' del momento en que se construyó.

De este modo, la actualización y la comparabilidad temporal se consolidan como garantías de que el índice permanece útil, coherente y sensible a las transformaciones sociales que condicionan los escenarios de riesgo.

\subsection{5.2.9 Transparencia metodológica y trazabilidad}\label{transparencia-metodoluxf3gica-y-trazabilidad}

La credibilidad de un índice depende en gran medida de la claridad con que se documenta su construcción.
La transparencia metodológica exige detallar cada decisión tomada: desde la selección de fuentes y variables hasta los criterios para asignar pesos o las técnicas estadísticas empleadas .
Siempre que la normativa lo permita, poner a disposición el conjunto de datos, el código de procesamiento y la documentación técnica garantiza que otros investigadores puedan reproducir los resultados y verificar la validez del índice de forma independiente.

Además, la trazabilidad implica reconocer explícitamente las limitaciones de las fuentes, los supuestos metodológicos y los márgenes de error asociados a cada etapa del proceso.
Este ejercicio no solo fomenta la confianza, sino que también ayuda a prevenir interpretaciones erróneas o aplicaciones fuera de contexto.
Una herramienta transparente y trazable fortalece tanto su legitimidad científica como su utilidad práctica en la formulación de políticas públicas.

\subsection{5.2.10 Participación, justicia espacial y legitimidad social}\label{participaciuxf3n-justicia-espacial-y-legitimidad-social}

Un índice de vulnerabilidad solo puede aspirar a tener legitimidad social si incorpora las voces de los actores locales en todas las fases clave de su elaboración \citep{lang2012-trans}.
Ello supone involucrar a comunidades, técnicos municipales, entidades sociales y responsables políticos en la elección de variables, en la interpretación de resultados y en la priorización de medidas.
Esta participación no debe limitarse a una consulta puntual, sino constituir un proceso continuado de co-producción de conocimiento.

Asimismo, la justicia espacial requiere que el índice permita visualizar cómo se distribuyen los riesgos y las capacidades según ejes de desigualdad como el género, la edad, la procedencia étnica o la situación administrativa de la población migrante.
De este modo, el índice no solo identifica territorios vulnerables, sino también colectivos específicos que concentran desventajas estructurales.

La legitimidad social se consolida cuando los resultados se presentan en formatos accesibles ---mapas interactivos, talleres participativos, infografías claras--- que faciliten la comprensión de diagnósticos complejos.
Esto asegura que el índice se convierta en una herramienta no solo para los expertos, sino también para las comunidades que viven los riesgos en primera persona.

\subsection{5.2.11 Gestión de la incertidumbre y reflexividad}\label{gestiuxf3n-de-la-incertidumbre-y-reflexividad}

Todo índice de vulnerabilidad trabaja con datos imperfectos: registros incompletos, errores de muestreo o simplificaciones inevitables.
Reconocer esta incertidumbre en lugar de ocultarla constituye una condición de rigor científico y honestidad metodológica \citep{hinkel2011-indicators}.
Una práctica recomendable es acompañar cada valor del índice con intervalos de confianza o márgenes de error obtenidos mediante técnicas estadísticas como el remuestreo bootstrap.

Además, la incertidumbre puede representarse visualmente, por ejemplo, mediante tonos menos saturados en mapas o franjas de variación en gráficos temporales, para que los usuarios distingan entre estimaciones robustas y aquellas que requieren cautela interpretativa.

La reflexividad complementa esta gestión al reconocer que la construcción del índice no es un proceso neutral, sino el resultado de decisiones valorativas y contextuales.
Por ello, se requieren mecanismos de revisión periódica que permitan incorporar aprendizajes, corregir sesgos y ajustar la herramienta a medida que surgen nuevos datos o cambian las condiciones sociales y territoriales.

\subsection{5.2.12 Utilidad práctica e institucionalización}\label{utilidad-pruxe1ctica-e-institucionalizaciuxf3n}

Un índice de vulnerabilidad solo adquiere sentido si se convierte en un insumo efectivo para la planificación y la gestión del riesgo.
Esto implica traducir los resultados en productos concretos ---informes periódicos, paneles interactivos o sistemas de monitoreo en línea--- que puedan ser usados directamente por administraciones públicas, organizaciones comunitarias y agentes de protección civil \citep{birkmann2013}.

La institucionalización requiere garantizar recursos estables y estructuras de gobernanza que aseguren la continuidad de la herramienta.
Ello supone disponer de un presupuesto dedicado, equipos técnicos permanentes y mecanismos de coordinación interinstitucional.
Asimismo, resulta crucial integrar perfiles diversos: especialistas en ciencias sociales y ambientales, expertos en análisis de datos y facilitadores de procesos participativos.

Bajo estas condiciones de sostenibilidad técnica, organizativa y política, el índice podrá conservar su utilidad a largo plazo, consolidándose como una referencia confiable para orientar decisiones y fundamentar políticas públicas de prevención y cohesión social.

\subsection{5.2.13 Síntesis y transición hacia la comparación de modelos}\label{suxedntesis-y-transiciuxf3n-hacia-la-comparaciuxf3n-de-modelos}

Los doce criterios desarrollados en esta sección constituyen el horizonte normativo que debería guiar cualquier índice de vulnerabilidad con vocación integral y transdisciplinar.
Se trata de principios que combinan exigencias conceptuales, metodológicas y prácticas: desde la fundamentación teórica y la integración de saberes heterogéneos, hasta la parsimonia estadística, la validación empírica, la gestión de la incertidumbre y la institucionalización del instrumento.

Cumplir simultáneamente con todos ellos es un desafío considerable.
La mayoría de los marcos existentes en el campo de la vulnerabilidad social y territorial han logrado avances parciales, pero también muestran limitaciones en términos de integración, participación, adaptabilidad o sostenibilidad.

Con este marco de criterios normativos como horizonte de referencia, la sección siguiente (5.3) examina en qué medida los modelos de vulnerabilidad seleccionados como marco teórico ---PAR, Cardona y MOVE--- se aproximan a dicho ideal, y qué lecciones ofrecen para la formulación del IVEC.

\section{5.3. Fundamentos teóricos: análisis comparativo de modelos de vulnerabilidad}\label{fundamentos-teuxf3ricos-anuxe1lisis-comparativo-de-modelos-de-vulnerabilidad}

El diseño del IVEC no parte de cero, sino que se construye sobre una base de marcos conceptuales y metodológicos ampliamente reconocidos en el campo de los estudios de riesgo y vulnerabilidad.
Revisar estos modelos permite identificar aportes fundamentales, así como vacíos persistentes en integración, adaptabilidad y aplicabilidad práctica.

Los hallazgos de \citet{rufat2023-criticalreview} refuerzan la pertinencia de este ejercicio: al contrastar distintos índices de vulnerabilidad social con datos empíricos de eventos reales, muestran que ningún modelo es universalmente válido y que la mayoría presenta limitaciones tanto en su capacidad explicativa como en su utilidad para la gestión del riesgo.

En este contexto, el análisis se centra en tres propuestas especialmente influyentes: el modelo PAR , el enfoque holístico de \citet{cardona2001} y el marco MOVE.
Cada uno de ellos aporta elementos clave para comprender la vulnerabilidad, aunque ninguno logra por sí mismo satisfacer plenamente los requisitos de un modelo integral y transdisciplinar.

\subsection{5.3.1 El modelo PAR}\label{el-modelo-par}

El modelo Pressure and Release (PAR), desarrollado en At Risk, constituye un referente fundamental en los estudios de vulnerabilidad.
Su principal aporte es demostrar que los desastres no son fenómenos ``naturales'', sino el resultado de procesos históricos, estructurales y sociales que generan condiciones de riesgo.

Mediante la metáfora de la ``presión y liberación'', el PAR conecta las causas profundas (como la desigualdad en el acceso a recursos o el poder político), con presiones dinámicas (urbanización acelerada, degradación ambiental) y condiciones inseguras que convierten un evento peligroso en desastre.
El modelo se complementa con el enfoque Access, que analiza cómo los hogares gestionan recursos y toman decisiones frente al riesgo.

En relación con los criterios normativos de la sección 5.2, el PAR destaca por su fundamentación teórica y conceptual sólida y por su énfasis en la participación comunitaria.
Sin embargo, presenta debilidades importantes en parsimonia y robustez estadística, ya que resulta difícil de operacionalizar con indicadores cuantitativos estandarizados, y carece de protocolos de validación empírica.

\subsection{5.3.2 El modelo holístico de Cardona}\label{el-modelo-holuxedstico-de-cardona}

El enfoque holístico propuesto por Cardona parte de una visión sistémica de la vulnerabilidad y el riesgo, integrando dimensiones sociales, físicas, económicas e institucionales.
A diferencia del PAR, centrado en procesos estructurales, el modelo de Cardona ofrece un marco de vulnerabilidad multidimensional, representable mediante indicadores agregados que facilitan la planificación pública.

Su principal fortaleza radica en la inclusión de capacidades de respuesta y adaptación como parte de la medición de la vulnerabilidad, lo que lo aproxima a los criterios de resiliencia señalados en la sección 5.2.
Asimismo, su carácter sintético y su orientación a escalas macro lo hacen útil en procesos de gestión pública y planificación estratégica.

No obstante, su énfasis en el nivel institucional conlleva limitaciones.
El modelo tiende a homogeneizar la diversidad local, mostrando menor sensibilidad hacia las microdinámicas de hogares e individuos.
En términos de los criterios normativos, resulta sólido en integración de dimensiones y en utilidad práctica, pero débil en participación social y justicia espacial.

\subsection{5.3.3 El marco MOVE}\label{el-marco-move}

El marco europeo MOVE (Methods for the Improvement of Vulnerability Assessment in Europe) fue diseñado para consolidar una metodología común de evaluación de la vulnerabilidad en contextos europeos.
Su principal innovación consiste en estructurarse en módulos temáticos (exposición, susceptibilidad, falta de resiliencia, etc.) y ofrecer un repertorio de indicadores comparables a nivel regional y municipal.

Una de sus aportaciones clave es la incorporación explícita de la resiliencia dentro del análisis de vulnerabilidad, alineándose con los principios de 5.2 sobre la inclusión de capacidades.
Asimismo, el MOVE ha logrado avances en parsimonia, robustez estadística y comparabilidad espacial, especialmente en contextos europeos con buena disponibilidad de datos.

Sin embargo, enfrenta críticas importantes: su enfoque modular tiende a fragmentar el análisis en compartimentos sectoriales, con predominio de indicadores cuantitativos.
Esto dificulta integrar saberes cualitativos y dinámicas sociales complejas.
Además, su transferibilidad fuera de Europa es limitada, lo que reduce su adaptabilidad contextual.

\subsection{5.3.4 Síntesis comparativa}\label{suxedntesis-comparativa}

Los tres modelos analizados han sido fundamentales para el desarrollo de los estudios de vulnerabilidad, pero presentan avances parciales frente al horizonte normativo definido en la sección.

El PAR aporta una mirada crítica, histórica y estructural, pero carece de mecanismos de validación y de estandarización cuantitativa.
El modelo holístico de Cardona integra múltiples dimensiones y capacidades, pero tiende a sobrerrepresentar la escala macro y a descuidar la diversidad local.
El MOVE asegura parsimonia, comparabilidad y solidez estadística, pero incorpora de manera restringida los procesos participativos y los saberes cualitativos.

La revisión comparativa confirma lo señalado por \citet{rufat2023-criticalreview}: ningún modelo existente satisface plenamente los criterios normativos definidos en la sección 5.2.

En conjunto, estos marcos representan hitos fundamentales, pero también dejan vacíos persistentes en integración transdisciplinar, adaptabilidad contextual y validación empírica.
La propuesta del IVEC se concibe precisamente como una respuesta a estos desafíos: recoger las fortalezas de cada modelo, superar sus limitaciones y avanzar hacia un índice flexible, multifactorial y sensible a las desigualdades sociales y territoriales de la Comunitat Valenciana.

Con esta base, la sección siguiente (5.4) presenta el diseño del Índice de Vulnerabilidad Estructural y Contextual (IVEC), detallando su lógica conceptual, su formulación metodológica y las innovaciones que introduce respecto a los modelos previamente revisados.

\section{5.4 Diseño del Índice de Vulnerabilidad Estructural y Contextual (IVEC)}\label{diseuxf1o-del-uxedndice-de-vulnerabilidad-estructural-y-contextual-ivec}

La revisión comparativa realizada en la sección anterior ha mostrado que los modelos PAR, Cardona y MOVE representan aportaciones fundamentales en la evolución de los estudios de vulnerabilidad, pero que ninguno logra integrar de manera plena los criterios de un enfoque integral y transdisciplinar.
Sobre esta base, el Índice de Vulnerabilidad Estructural y Contextual (IVEC) se concibe como un esfuerzo por consolidar dichas fortalezas y, al mismo tiempo, superar las limitaciones detectadas.

El IVEC no pretende sustituir a los modelos previos, sino dialogar con ellos: recoge la mirada crítica y estructural del PAR, incorpora la perspectiva sistémica y de capacidades del modelo holístico de Cardona, y se beneficia de la parsimonia y comparabilidad estadística del marco MOVE.
Sin embargo, introduce innovaciones clave que buscan responder a los vacíos señalados: la integración de saberes heterogéneos, la validación empírica en contextos locales, la incorporación explícita de la justicia espacial y la institucionalización de la herramienta como instrumento de política pública.

En lo que sigue se detallan los fundamentos conceptuales, la formulación metodológica y las etapas de implementación del IVEC, con el objetivo de mostrar cómo este índice se configura como una propuesta flexible, robusta y socialmente legítima para la estimación de la vulnerabilidad en la Comunitat Valenciana.

Nota: alto = cumple ampliamente; medio = cumplimiento parcial; bajo = cumplimiento limitado.

Fuente: \emph{Elaboración propia a partir de la revisión realizada por Rufat (2009) y requisitos de los marcos transdisciplinares de Hinkel (2011), Lang (2012) y Jahn (2012).}

\section{5.4 Desarrollo metodológico del IVEC}\label{desarrollo-metodoluxf3gico-del-ivec}

Esta sección describe, paso a paso, la construcción del Índice de Vulnerabilidad Estructural y Contextual (IVEC) y se divide en seis subsecciones.
El primero delimita la unidad de análisis y la escala espacial; el segundo explica la selección y operacionalización de variables; el tercero presenta las fuentes de datos y sus principales carácterísticas; el cuarto expone los procedimientos de normalización, ponderación y agregación; el quinto detalla la fórmula final del índice y las decisiones sobre compensaciones; y el sexto resume las pruebas de validación preliminar, la interpretación de resultados y las limitaciones detectadas.
El objetivo es ofrecer un recorrido transparente y reproducible que vincule los fundamentos teóricos con la implementación técnica.

El diseño metodológico del IVEC se estructura siguiendo la lógica interna del propio modelo.
El recorrido comienza con la unidad de análisis, la del individuo y su hogar, donde se consideran tanto las condiciones socioeconómicas y características demográficas, como las relaciones intrahogar y las capacidades de afrontamiento.
Desde ahí, el análisis se amplía hacia el entorno y la escala territoriales, integrando las características estructurales y contextuales que condicionan la vida cotidiana.

Una vez situada la población en su territorio, se aborda su grado de exposición a las amenazas, es decir, la probabilidad de que determinados eventos peligrosos puedan afectarla en función de su localización y de las dinámicas socioambientales del área.
A continuación, se incorporan las dimensiones de resiliencia, entendidas como las capacidades institucionales y comunitarias de anticipación, respuesta y recuperación frente a los impactos adversos.

Finalmente, este recorrido culmina en la validación del índice, que se plantea en tres planos complementarios.
En primer lugar, se verifica su coherencia respecto a los fundamentos teóricos que lo sustentan, en diálogo con los principales modelos de vulnerabilidad y gobernanza del riesgo.
En segundo lugar, se aplican pruebas estadísticas orientadas a comprobar la consistencia interna, la comparabilidad de los resultados y la estabilidad del índice.
Y, en tercer lugar, se contrasta su capacidad explicativa frente a la realidad empírica, cotejando los valores obtenidos con los efectos producidos por eventos recientes, como la DANA de 2024, lo que permite evaluar hasta qué punto el IVEC anticipa adecuadamente los patrones de impacto territorial y social.

\subsection{5.4.1 Unidad de análisis y escala espacial}\label{unidad-de-anuxe1lisis-y-escala-espacial}

La medición de la vulnerabilidad requiere un nivel de resolución que capture, de manera simultánea, las diferencias micro ---individuos y hogares--- y las desigualdades que estructuran el territorio en niveles meso y macro.
En coherencia con los marcos conceptuales revisados ---MOVE, PAR y Cardona---, esta investigación parte de la premisa de que la vulnerabilidad no es una condición estática ni un atributo individual, sino un proceso multidimensional, relacional y dinámico que emerge de la interacción entre biografías personales, estructuras sociales e inserción territorial \citep{birkmann2013, jamshed2023}.

En el IVEC, la unidad mínima de observación es el individuo, vinculado a su unidad de convivencia (UCO) y georreferenciado en una malla regular de celdas de 1 km² que cubre la totalidad de la Comunitat Valenciana.
Esta decisión metodológica se justifica en tres dimensiones principales:

\begin{enumerate}
\def\labelenumi{\arabic{enumi}.}
\item
  Evitar sesgos de agregación.
  La observación a nivel individual impide que los promedios poblacionales oculten desigualdades internas, un problema recurrente en la literatura sobre índices de vulnerabilidad \citep{cutter2003, rufat2019-validation}.
  El IVEC visibiliza así coexistencias contrastadas ---como áreas de alta vulnerabilidad en municipios con indicadores medios aparentemente favorables--- que quedarían invisibles en escalas agregadas.
\item
  Integración directa con la peligrosidad física.
  La resolución espacial de 1 km² facilita la superposición con las capas de peligrosidad del territorio (inundaciones, incendios forestales, olas de calor), lo que permite analizar vulnerabilidad y riesgo en un mismo soporte cartográfico y con criterios homogéneos.
  Esta integración responde a la exigencia del Marco de Sendai de ``conocer el riesgo en todas sus dimensiones'' \citep{undrr2015}.
\item
  Puente entre escalas domésticas y territoriales.
  La vinculación entre UCO y grid territorial posibilita escalar los resultados hacia ámbitos meso (comarcas, áreas de salud, demarcaciones de riesgo) y macro (Comunitat Valenciana, comparaciones estatales y europeas).
  Esta flexibilidad es coherente con la crítica al \emph{modulus scalaris} formulada en MOVE, que subraya la necesidad de marcos comparativos sin perder especificidad local \citep{birkmann2013}.
\end{enumerate}

Más allá de estos criterios técnicos, la definición de la unidad mínima de análisis constituye también el mecanismo que permite materializar los principios expuestos en la sección 5.2 para un índice integral y transdisciplinar.
En efecto, asegura la integralidad, al integrar dimensiones estructurales, contextuales y de capacidad; garantiza la transdisciplinariedad, al tender puentes entre ciencias sociales, ambientales y de la salud; refuerza la reproducibilidad y trazabilidad, al basarse en registros administrativos estandarizados como SIP; posibilita la comparabilidad multiescalar, al permitir desagregación y agregación flexible en diferentes escalas; y asegura la operatividad para las políticas públicas, al facilitar su incorporación en planes sectoriales como PATRICOVA o PEIF, y alinearse con el Marco de Sendai.
La relación entre estos principios y los criterios metodológicos adoptados se sintetiza en la siguiente tabla.

\begin{table}[!h]
\centering
\caption{\label{tab:transdisciplinares}Principios transdisciplinares y su aplicación en el IVEC}
\centering
\fontsize{8}{10}\selectfont
\begin{threeparttable}
\begin{tabular}[t]{ll}
\toprule
Principio & Aplicación\\
\midrule
Integralidad & Individuo/UCO integra dimensiones estructurales, contextuales y de capacidad.\\
Transdisciplinariedad & Resolución 1 km² permite diálogo entre ciencias sociales, ambientales y de salud.\\
Reproducibilidad y trazabilidad & Uso de SIP garantiza replicabilidad y actualización periódica (mensual).\\
Comparabilidad multiescalar & Malla regular de 1 km² permite análisis micro, meso y macro comparables.\\
Operatividad para políticas públicas & Compatibilidad con planes sectoriales (PATRICOVA, PEIF) y Marco de Sendai.\\
\bottomrule
\end{tabular}
\begin{tablenotes}
\item Fuente: Elaboración propia a partir de la aplicación de los principios de Rufat (2019)
\end{tablenotes}
\end{threeparttable}
\end{table}

En síntesis, la definición de la unidad mínima de análisis en el IVEC ---el individuo vinculado a su UCO y georreferenciado en celdas de 1 km²--- constituye la base sobre la cual se construyen los desarrollos posteriores.
Esta decisión metodológica permite no solo garantizar la consistencia interna del índice, sino también abrir el camino hacia la definición de sus dimensiones e indicadores específicos, que se abordan en la sección siguiente.

\subsection{5.4.2 Dimensiones de análisis e indicadores}\label{dimensiones-de-anuxe1lisis-e-indicadores}

La selección de indicadores para el IVEC descansa en la aplicación simultánea de tres filtros que funcionan de forma integrada.
El primero es la pertinencia analítica, que exige que cada variable o indicador aporte información útil para los objetivos del índice y resulte operativa a la hora de interpretar las desigualdades y orientar la acción pública en la Comunitat Valenciana.
A continuación se aplica la relevancia conceptual: todo indicador ha de dialogar con los marcos teóricos de referencia ---MOVE, Cardona o PAR--- y representar de manera clara alguna dimensión reconocida de la vulnerabilidad, ya sea económica, demográfica, relacional o sus combinaciones.
Finalmente, cada variable candidata se somete a un examen de fiabilidad empírica, de modo que solo se incorporan aquellas que presentan series actualizadas, baja tasa de omisión y coherencia interna contrastada.
La intersección de estos tres filtros ---pertinencia, relevancia y fiabilidad--- asegura que los indicadores conserven valor analítico, anclaje teórico y solidez estadística al mismo tiempo.
Estos criterios generales se complementan, en las secciones siguientes, con procedimientos específicos de tratamiento y normalización que garantizan la comparabilidad y reproducibilidad técnica del índice.

Sobre esta base, el IVEC se organiza en cinco dimensiones principales que reflejan tanto los condicionantes estructurales como las capacidades sociales e institucionales que median la vulnerabilidad.
En conjunto, constituyen una arquitectura metodológica coherente con los principios de integralidad y transdisciplinariedad planteados en la sección 5.2 y con los marcos conceptuales revisados.

\begin{enumerate}
\def\labelenumi{\arabic{enumi}.}
\item
  Vulnerabilidad estructural (VE).
  Incluye las condiciones socioeconómicas y demográficas que predisponen a los individuos y hogares a experimentar situaciones de fragilidad frente a los riesgos.
  Factores como la renta, el empleo o la composición familiar reflejan desigualdades persistentes que operan como determinantes de fondo.
  Esta dimensión conecta con el modelo PAR, en el que la vulnerabilidad se entiende como resultado de procesos históricos de exclusión y acumulación de desventajas.
\item
  Vulnerabilidad contextual (VC).
  Considera los atributos del entorno físico, urbano y territorial que afectan directamente a las oportunidades de protección y a la exposición diferencial a los riesgos.
  Se incluyen aquí aspectos como la precariedad residencial, la calidad de las infraestructuras o la accesibilidad a servicios básicos.
  La noción de vulnerabilidad contextual responde a la crítica al \emph{modulus scalaris} presente en MOVE, al reconocer que las condiciones territoriales configuran escenarios desiguales de riesgo.
\item
  Capacidades sociales (CS).
  Evalúan los recursos comunitarios y relacionales que pueden actuar como amortiguadores frente a situaciones de crisis.
  La densidad de asociaciones o redes de apoyo vecinal constituyen indicadores clave de esta dimensión.
  Su inclusión responde a la perspectiva de Cardona, que señala la importancia de las capacidades colectivas para transformar la vulnerabilidad en resiliencia.
\item
  Capacidades de adaptación (CAD).
  Incorporan los mecanismos institucionales y colectivos que permiten anticipar y reducir riesgos futuros, en especial en el contexto del cambio climático.
  Ejemplos son la cobertura de planes de adaptación o la existencia de programas de gestión ambiental.
  Esta dimensión conecta directamente con el Marco de Sendai y con los debates actuales sobre gobernanza del riesgo y planificación adaptativa.
\item
  Capacidad de respuesta (CR).
  Hace referencia a los recursos disponibles para hacer frente de manera inmediata a una emergencia y reducir sus consecuencias.
  Incluye la infraestructura sanitaria, la disponibilidad de equipos de emergencia o la logística para la evacuación.
  Su reconocimiento como dimensión autónoma refuerza la idea de que la vulnerabilidad no se agota en los condicionantes estructurales y contextuales, sino que depende también de la capacidad de desplegar recursos efectivos en situaciones críticas.
\end{enumerate}

Estas cinco dimensiones permiten pasar de una concepción abstracta de la vulnerabilidad a un marco operativo que reconoce tanto las limitaciones estructurales como los recursos sociales e institucionales.
En apartados posteriores se presentará la forma en que cada dimensión se operacionaliza a través de indicadores específicos y se integrará en la fórmula del IVEC, junto con los factores de peligrosidad y exposición necesarios para la estimación del riesgo potencial.

\subsection{5.4.3 Operacionalización de indicadores}\label{operacionalizaciuxf3n-de-indicadores}

Definidas las cinco dimensiones que estructuran el IVEC, el paso siguiente consiste en traducirlas en indicadores concretos que permitan su medición empírica.
La operacionalización responde a los filtros de selección planteados en la introducción de esta sección ---pertinencia analítica, relevancia conceptual y fiabilidad empírica--- y se apoya en fuentes de información públicas, actualizables y georreferenciadas.

En la \textbf{vulnerabilidad estructural (VE)} se incluyen variables como renta media o tasa de desempleo, que permiten captar las desigualdades socioeconómicas persistentes.
Estos indicadores provienen de registros estadísticos consolidados (IVE, INE, APSIG) y aportan una visión clara de las condiciones materiales que condicionan la posición social de los hogares.

La \textbf{vulnerabilidad contextual (VC)} se aproxima mediante indicadores relativos a precariedad residencial, accesibilidad a servicios sanitarios o calidad del entorno urbano.
Estos datos se obtienen principalmente a partir de APSIG y del SIP, lo que permite asignarlos con precisión a la malla de 1 km² y poner en evidencia cómo el territorio configura escenarios desiguales de riesgo.

Las \textbf{capacidades sociales (CS)} se captan a través de indicadores sobre densidad de asociaciones comunitarias o existencia de redes locales de apoyo.
Su incorporación responde a la necesidad de reconocer que el capital social constituye un recurso fundamental frente a crisis y desastres, y que su ausencia incrementa la vulnerabilidad de los territorios.

Las \textbf{capacidades de adaptación (CAD)} se operacionalizan mediante variables que reflejan el grado de preparación institucional frente a amenazas futuras, como la cobertura poblacional de planes de adaptación o la presencia de estrategias de gestión climática a nivel local.
Estas variables provienen de registros oficiales de la administración autonómica y municipal.

Finalmente, la \textbf{capacidad de respuesta (CR)} se mide mediante indicadores como la disponibilidad de camas hospitalarias o de infraestructuras críticas para emergencias.
Esta dimensión aporta información clave para evaluar la capacidad inmediata de las instituciones y servicios públicos de reducir daños en caso de un evento adverso.

Todos los indicadores son sometidos a un procedimiento homogéneo de transformación que garantiza su comparabilidad: (a) normalización mediante la técnica min--max, que lleva los valores a un rango común entre 0 y 1; (b) reorientación de las escalas, de manera que valores más altos representen siempre mayor vulnerabilidad; y (c) asignación espacial a la malla regular de 1 km² que constituye la unidad mínima de análisis.
Este tratamiento permite integrar en un mismo marco variables de distinta naturaleza y escala, evitando sesgos derivados de unidades de medida heterogéneas.

En conjunto, este proceso de operacionalización asegura que cada dimensión del IVEC se apoye en indicadores con valor analítico, anclaje teórico y solidez empírica, respetando al mismo tiempo los principios de integralidad, transdisciplinariedad y reproducibilidad que guían toda la construcción del índice.

\subsubsection{5.4.2.3 Variables contextuales/individuales}\label{variables-contextualesindividuales}

Las variables seleccionadas para capturar la vulnerabilidad contextual se corresponden con atributos individuales que, según la evidencia acumulada en otros estudios y las posibilidades de la fuente, condicionan el grado de precariedad y exposición al riesgo.
Entre las principales variables consideradas, destacan:

\begin{longtable}[]{@{}
  >{\raggedright\arraybackslash}p{(\linewidth - 4\tabcolsep) * \real{0.3333}}
  >{\raggedright\arraybackslash}p{(\linewidth - 4\tabcolsep) * \real{0.3333}}
  >{\raggedright\arraybackslash}p{(\linewidth - 4\tabcolsep) * \real{0.3333}}@{}}
\toprule\noalign{}
\begin{minipage}[b]{\linewidth}\raggedright
Variable
\end{minipage} & \begin{minipage}[b]{\linewidth}\raggedright
Categorías principales
\end{minipage} & \begin{minipage}[b]{\linewidth}\raggedright
Fuente APSIG
\end{minipage} \\
\midrule\noalign{}
\endhead
\bottomrule\noalign{}
\endlastfoot
Tramo de renta individual & AC-50\%, AC, AC-60\%, AC-100\%, etc. & D9\_raf\_renta \\
Situación laboral & Trabaja, No trabaja pero puede, Desempleado, etc. & D4\_lab \\
Edad & Grupos etarios (e.g.~\textless18, 18--64, 65+) & f\_nacim \\
Sexo & Hombre, Mujer & sexo \\
Nacionalidad & Española, Extranjera, Sin nacionalidad & nacionalidad \\
Cronicidad & Nivel 0, Nivel 1, Nivel 2 & cronicidad \\
Situación administrativa & Residente habitual, Sin regularización, etc. & D2\_resid \\
Empadronamiento & Empadronado, No empadronado & empadronamiento \\
\end{longtable}

\subsubsection{5.4.2.4 Variables estructurales}\label{variables-estructurales}

Estas variables determinan la posición del individuo en el territorio a partir de los datos contenidos en el APSIG, lo cual no permite contextualizar la Vulnerabilidad Estructural.

\begin{longtable}[]{@{}
  >{\raggedright\arraybackslash}p{(\linewidth - 4\tabcolsep) * \real{0.3333}}
  >{\raggedright\arraybackslash}p{(\linewidth - 4\tabcolsep) * \real{0.3333}}
  >{\raggedright\arraybackslash}p{(\linewidth - 4\tabcolsep) * \real{0.3333}}@{}}
\toprule\noalign{}
\begin{minipage}[b]{\linewidth}\raggedright
Variable
\end{minipage} & \begin{minipage}[b]{\linewidth}\raggedright
Categorías/Escala
\end{minipage} & \begin{minipage}[b]{\linewidth}\raggedright
Fuente APSIG
\end{minipage} \\
\midrule\noalign{}
\endhead
\bottomrule\noalign{}
\endlastfoot
Coordenadas residencia & X, Y (UTM, grid 1 km²) & ST\_X, ST\_Y \\
Departamento/área de salud & Nombres específicos & dep\_resid, centro\_asig, zona\_resid \\
Área sanitaria & Valencia Sur, Alicante Norte, etc. & ASI \\
Lugar de asignación & Centro de salud asignado & centro\_asig \\
\end{longtable}

\subsubsection{5.4.2.5 Variables de la UCO y compensaciones intrahogar}\label{variables-de-la-uco-y-compensaciones-intrahogar}

El modelo IVEC incorpora la lógica relacional e intrahogar utilizando variables de la unidad de convivencia, permitiendo inferir mecanismos de protección o agravamiento de la vulnerabilidad individual según la estructura y composición del hogar:

\begin{longtable}[]{@{}
  >{\raggedright\arraybackslash}p{(\linewidth - 4\tabcolsep) * \real{0.3333}}
  >{\raggedright\arraybackslash}p{(\linewidth - 4\tabcolsep) * \real{0.3333}}
  >{\raggedright\arraybackslash}p{(\linewidth - 4\tabcolsep) * \real{0.3333}}@{}}
\toprule\noalign{}
\begin{minipage}[b]{\linewidth}\raggedright
Variable
\end{minipage} & \begin{minipage}[b]{\linewidth}\raggedright
Categorías principales
\end{minipage} & \begin{minipage}[b]{\linewidth}\raggedright
Fuente APSIG
\end{minipage} \\
\midrule\noalign{}
\endhead
\bottomrule\noalign{}
\endlastfoot
Tipo de hogar & Unipersonal, Pareja sin hijos, Familia, etc. & D8\_tipo\_res \\
Número de adultos & Numérico/categorizado & D8\_comp\_res \\
Número de menores & Numérico/categorizado & D8\_comp\_res \\
Composición básica & Dos adultos sin menores, Adultos con hijos, etc. & D8\_comp\_res \\
\end{longtable}

A partir de estas variables, se aplican tipologías y reglas funcionales (como la escala OCDE, equivalente a la escala de las unidad de consumo) para estimar compensadores de la vulnerabilidad individual, considerando la posible protección, redistribución o agravamiento del riesgo social según la composición del hogar.

\subsection{5.4.3 Fuentes de datos}\label{fuentes-de-datos}

El SIP/APSIG constituye la base troncal sobre la que se articula el IVEC.
La fuente principal utilizada para la construcción del IVEC es el Sistema de Información Poblacional (SIP) de la Comunitat Valenciana, complementado por la codificación APSIG.
Esta base de datos administrativa destaca por su cobertura exhaustiva de la población residente, su actualización continua y su nivel de desagregación individual y territorial.
Gracias al SIP/APSIG es posible trabajar con microdatos ---cada registro corresponde a un individuo---, lo que permite identificar y analizar patrones de vulnerabilidad que serían invisibles en fuentes más agregadas.

Una de las fortalezas metodológicas del SIP/APSIG radica en su capacidad para georreferenciar cada registro en una rejilla territorial de alta resolución (grids de 1 km²).
Esta granularidad espacial resulta clave para evitar el ``efecto de ocultamiento'' propio de los análisis municipales, y facilita la detección de desigualdades internas, incluso dentro de un mismo barrio o zona de salud.

El uso del SIP/APSIG responde también a criterios de robustez y replicabilidad: se trata de una fuente institucional, actualizada y sometida a estándares de calidad y control documental, lo que la convierte en un soporte idóneo para la construcción de indicadores comparables y útiles para la gestión pública.

En resumen, la utilización del SIP/APSIG permite articular un enfoque metodológico multiescalar ---individuo, hogar (UCO), grid territorial---, compatible con los estándares internacionales y adaptado al contexto valenciano.
Esta elección técnica no es neutral, sino que responde a la necesidad de captar tanto las diferencias individuales como las desigualdades territoriales en la experiencia y la distribución de la vulnerabilidad social.

\subsection{5.4.4 Estrategias ante limitaciones de APSIG}\label{estrategias-ante-limitaciones-de-apsig}

A pesar de la amplitud y riqueza de APSIG/SIP como fuente de microdatos,su naturaleza administrativa genera ciertas limitaciones estructurales que deben ser tenidas en cuenta en la construcción del IVEC.
Las principales carencias identificadas afectan a la representación completa de las relaciones intrahogar, el nivel educativo y la renta familiar agregada, así como a la información sobre acceso a redes de apoyo o servicios comunitarios.

\textbf{Principales limitaciones:}

\begin{itemize}
\tightlist
\item
  Ausencia de parentesco explícito: APSIG no codifica la relación exacta entre los miembros de la unidad de convivencia (UCO). Esto obliga a inferir relaciones funcionales a partir del número de adultos y menores, el sexo, la edad y la situación laboral.
\item
  Falta de información sobre renta familiar agregada: Solo se dispone de tramos individuales de renta vinculados al copago farmacéutico. La renta total del hogar debe ser aproximada mediante reglas y estimaciones.
\item
  Ausencia de nivel educativo: No se recoge el capital cultural formal de los miembros, lo que limita la capacidad de modelar este tipo de compensadores sociales.
\item
  Sin información directa sobre acceso a servicios, redes de apoyo o entorno urbano inmediato: Dimensiones importantes de la vulnerabilidad quedan fuera del alcance directo del análisis.
\end{itemize}

\textbf{Estrategias de compensación metodológica:}

\begin{itemize}
\item
  Inferencia de relaciones intrahogar: Se utilizan reglas plausibles para estimar la probable estructura y funcionalidad del hogar (por ejemplo, parejas adultas con menores, hogares unipersonales, núcleos extensos).
  A partir de estas tipologías, se definen algoritmos de compensación (como la escala OCDE y otros ponderadores funcionales) que permiten ajustar la vulnerabilidad individual en función de los recursos y protecciones disponibles en la UCO.
\item
  Estimación de renta y protección intrahogar: En APSIG, el nivel de renta de cada individuo está aproximado mediante el tramo asignado para el copago farmacéutico.
  Aunque no se dispone de información directa sobre los ingresos exactos ni la renta agregada familiar, este tramo constituye, junto con la situación laboral y otras variables (cronicidad, situación administrativa, composición de la UCO), un proxy razonable del estatus socioeconómico individual.

  En el caso de menores y adultos dependientes, el tramo de renta suele ser el del progenitor, tutor o titular principal de la unidad de convivencia, reflejando la situación económica del hogar en su conjunto.
  Sin embargo, es importante señalar que esta aproximación puede ocultar situaciones de desigual reparto intrahogar o la existencia de unidades de convivencia atípicas.

  Por tanto, el modelo IVEC utiliza el tramo de renta como variable central para estimar vulnerabilidad económica individual y, de manera cautelosa, para aproximar la situación económica familiar en aquellos casos en que no es posible asignar un tramo específico a cada miembro de la UCO.
  Esta estrategia, aunque limitada, es coherente con las prácticas empleadas en sistemas públicos de salud y protección social, y permite mantener la comparabilidad y la interpretación de los indicadores en el contexto valenciano.
\item
  Advertencia interpretativa: Todas estas inferencias y ajustes deben ser leídas como aproximaciones funcionales, nunca como representaciones exactas de la dinámica interna de los hogares.
  El análisis y la interpretación de resultados requieren siempre una actitud crítica, reconociendo las limitaciones y potenciales sesgos derivados de la fuente y las decisiones adoptadas en la propia investigación.
\end{itemize}

En síntesis, la estrategia metodológica adoptada para paliar las limitaciones de APSIG permite mantener la coherencia y la potencia explicativa del modelo IVEC, a la vez que advierte de la necesidad de complementar estos análisis con fuentes cualitativas o registros complementarios cuando sea relevante para la política pública o el diagnóstico social.

\subsection{5.4.5 Normalización y tratamiento estadístico}\label{normalizaciuxf3n-y-tratamiento-estaduxedstico}

La calidad y robustez del IVEC dependen en gran medida de la coherencia y comparabilidad de los datos de entrada.
El proceso de normalización y tratamiento estadístico se diseñó para garantizar que las variables seleccionadas, extraídas de APSIG/SIP, fueran plenamente aptas para su integración en un índice compuesto, manteniendo siempre la trazabilidad y el control de calidad.
El paso siguiente tras la depuración y estructuración de la base APSIG consiste en \textbf{transformar las variables disponibles en indicadores estadísticos comparables}, listos para ser integrados en el modelo IVEC.
Este proceso exige combinar criterios técnicos con sensibilidad teórica y social.

\subsubsection{5.4.5.1 Control de calidad y depuración}\label{control-de-calidad-y-depuraciuxf3n}

La base APSIG ha sido sometida a un proceso metodológico exhaustivo de tratamiento y control de calidad, documentado paso a paso en el Anexo 3 (véase el informe completo).
El objetivo ha sido transformar el fichero original del SIP en un conjunto de datos limpio, estructurado y enriquecido, apto para análisis estadístico y espacial reproducible.
El flujo de trabajo se desarrolló íntegramente en R, con registro de todos los scripts y transformaciones.

Principales pasos y garantías de calidad:

\begin{itemize}
\tightlist
\item
  Limpieza inicial y carga controlada: Eliminación de columnas y registros vacíos, interpretación homogénea de valores nulos.
\item
  Normalización e imputación de coordenadas: Unificación de identificadores, extracción y validación cruzada de coordenadas geográficas, imputación sistemática para registros incompletos.
\item
  Ingeniería de variables: Segmentación y descomposición de la variable APSI\_46 en más de 20 variables analíticas, con controles de formato y consistencia.
\item
  Etiquetado semántico: Enriquecimiento con diccionarios externos, generación automática de columnas descriptivas y validación de la integridad semántica.
\item
  Evaluación cuantitativa de calidad: Análisis gráfico y tabular de la proporción de valores ausentes antes y después de la depuración; supresión de 2.772 registros sin contenido relevante.
\item
  Revisión cruzada de procedimientos: Cada transformación ha sido verificada mediante resúmenes automáticos y gráficos, asegurando la mínima pérdida de información y la máxima trazabilidad.
\item
  Organización y exportación final: Selección y ordenación lógica de variables clave, asegurando la compatibilidad con análisis posteriores y la reproducibilidad total del proceso.
\end{itemize}

\textbf{Resultados principales:}

\begin{itemize}
\tightlist
\item
  La matriz final obtenida a partir de APSIG para la construcción del IVEC integra 5.029.092 individuos agrupados en 2.192.983 unidades de convivencia (UCO). El tratamiento y depuración de la base ha permitido disponer de 28 variables por individuo, incluyendo información sociodemográfica, sanitaria, administrativa y territorial relevante para la estimación de la vulnerabilidad contextual y estructural.
\item
  La base final presenta tasas de omisión muy bajas en todas las variables relevantes (\textless0,05\% en la mayoría), con excepción de las coordenadas geográficas (\textasciitilde18\% ausentes).\footnote{La asignación de coordenadas geográficas a cada individuo de la base APSIG depende de una cadena de procedimientos administrativos y técnicos que, aunque robustos, presentan inevitables márgenes de error y cobertura incompleta.
    En concreto, para que un registro individual pueda ser correctamente georreferenciado es necesario que:

    \begin{enumerate}
    \def\labelenumi{\arabic{enumi}.}
    \tightlist
    \item
      La dirección de residencia habitual esté correctamente codificada según el estándar de vías del INE.
    \item
      Exista correspondencia entre la calle codificada por el INE y la cartografía de referencia utilizada (cartografía IDESAN, derivada del ICV), de modo que cada vía pueda vincularse de manera única.
    \item
      El portal o número de policía de la dirección esté recogido y georreferenciado en la cartografía oficial, de esta manera, puede asignarse una coordenada precisa al individuo.
    \end{enumerate}

    La ausencia o inconsistencia en cualquiera de estos pasos puede dar lugar a la imposibilidad de georreferenciar un registro.
    Por ello, aproximadamente un 18\% de los individuos incluidos en APSIG no han podido ser localizados espacialmente con el nivel de detalle requerido, bien por direcciones incompletas, errores en la codificación, falta de correspondencia cartográfica o ausencia del portal en el registro geográfico.

    Este margen de no-georreferenciación, aunque esperable en bases administrativas de gran escala, debe ser tenido en cuenta en la interpretación territorial de los resultados del IVEC, especialmente en zonas con mayor proporción de direcciones inexactas o informales.}
\item
  Todas las variables de identificación y segmentación sociodemográfica mantienen integridad y completitud, lo que garantiza la calidad de los análisis y de los indicadores derivados.
\end{itemize}

knitr::include\_graphics(``figures/grafico\_calidad\_despues\_limpieza\_APSI46.png'') \textbackslash{}\\

\begin{quote}
Para mayor detalle sobre cada paso, códigos y justificación técnica, véase el Anexo Metodológico: Procesamiento y Enriquecimiento de la Base de Datos APSIG.
\end{quote}

\subsubsection{5.4.5.3 Construcción de indicadores compuestos}\label{construcciuxf3n-de-indicadores-compuestos}

Donde procede, se construyen \textbf{factores sintéticos} para representar dimensiones complejas (por ejemplo, ``precariedad laboral'' o ``vulnerabilidad sanitaria''), recurriendo a:

\begin{itemize}
\tightlist
\item
  \textbf{Análisis de Componentes Principales (PCA):} Permite sintetizar múltiples variables en un componente único, maximizando la varianza explicada y facilitando la parsimonia del índice.
\item
  \textbf{Clusterización (k-means):} Aplicada para definir tipologías de hogares y validar la lógica compensatoria intrahogar.
\end{itemize}

Todos los cálculos y agrupaciones se documentan en scripts reproducibles (véase Apéndice 3).

\subsubsection{5.4.5.2 Matriz final para el cálculo del IVEC**}\label{matriz-final-para-el-cuxe1lculo-del-ivec}

El resultado es una \textbf{matriz de indicadores estandarizados} para cada individuo/UCO, donde cada fila recoge los valores transformados de las variables principales que alimentan el cálculo del IVEC:

\begin{itemize}
\tightlist
\item
  Vulnerabilidad contextual individual (VC)
\item
  Vulnerabilidad estructural del entorno (VE)
\item
  Capacidades institucionales (CAD/CR)
\item
  Tipología y compensador intrahogar
\end{itemize}

Esta matriz es la base para el cálculo del índice compuesto, asegurando comparabilidad, interpretabilidad y transparencia.

\begin{quote}
Para mayor detalle de la construcción y validación de cada indicador, puede consultarse el Apéndice 4 y los scripts incluidos en el repositorio documental de la investigación.
\end{quote}

\subsection{5.4.6 Ponderación, agregación y fórmula del índice}\label{ponderaciuxf3n-agregaciuxf3n-y-fuxf3rmula-del-uxedndice}

En la creación del IVEC, la cuestión de la ponderación y la agregación cobra especial relevancia, ya que cada elección metodológica puede incidir de forma significativa en la capacidad del índice para captar desigualdades y patrones reales de vulnerabilidad social.

Siguiendo las recomendaciones sintetizadas por \citet{greco2019-composite}, el proceso de ponderación en el IVEC se apoya fundamentalmente en técnicas estadísticas data-driven, en particular el Análisis de Componentes Principales (PCA).
Esta aproximación permite derivar pesos objetivos a partir de la estructura interna de los datos, maximizando la varianza explicada y minimizando la arbitrariedad subjetiva en la asignación de importancia relativa a los diferentes indicadores.
El uso de PCA resulta especialmente adecuado cuando se trata de fenómenos complejos y multidimensionales, y garantiza que el índice sea sensible a los patrones de desigualdad realmente presentes en la población.

En cuanto a la agregación, es importante señalar una decisión metodológica específica y deliberada: \textbf{la agregación aritmética solo se utiliza en el IVEC para incorporar la influencia de las relaciones intrahogar dentro de la Vulnerabilidad Contextual Individual (VCind)}.
Esta operación responde a la necesidad de modelar cómo las condiciones y recursos compartidos en el hogar ---por ejemplo, el cuidado mutuo, la distribución de ingresos o la presencia de dependencias--- afectan la vulnerabilidad individual de sus miembros.
La suma ponderada, en este caso, permite capturar la interacción entre factores individuales y familiares, bajo el supuesto de cierta compensación interna que es propia de las dinámicas intrahogar.

Para el resto de dimensiones y etapas del índice, se emplean estrategias de agregación y ponderación acordes al marco estadístico y a la lógica multidimensional de la vulnerabilidad.
Esto implica evitar que los mecanismos de compensación oculten situaciones de precariedad extrema en alguna dimensión, lo cual podría ocurrir si se aplicara una agregación aritmética en todos los niveles.
Por ello, la estructura del IVEC prioriza la interpretación diferenciada de las dimensiones, y solo integra la compensación donde es conceptualmente justificable y empíricamente relevante ---como ocurre en el caso de la VCind.

El procedimiento se complementa con análisis de agrupamiento (k-means), lo que permite identificar patrones territoriales y perfiles internos de vulnerabilidad, y con pruebas de robustez y sensibilidad para asegurar que los resultados no dependan excesivamente de las decisiones técnicas tomadas en el proceso de construcción del índice.

Esta estrategia metodológica sitúa al IVEC en consonancia con las mejores prácticas internacionales, adaptando de manera reflexiva los métodos de ponderación y agregación a la especificidad del fenómeno estudiado y a las necesidades de interpretación y acción en el contexto valenciano.

\begin{quote}
\emph{Referencia: Greco, S., Ishizaka, A., Tasiou, M., \& Torrisi, G. (2019). On the methodological framework of composite indices: A review of the issues of weighting, aggregation, and robustness. Social Indicators Research, 141(1), 61--94. \url{https://doi.org/10.1007/s11205-017-1832-9}}
\end{quote}

Esta estructura \textbf{multiplicativa} ha sido elegida para:

\begin{itemize}
\tightlist
\item
  \textbf{Reflejar la interacción real} entre condiciones individuales y estructurales, y las capacidades institucionales existentes.
\item
  \textbf{Evitar compensaciones excesivas:} Una puntuación alta en una dimensión no puede anular completamente una carencia severa en otra, siguiendo las recomendaciones de \citet{greco2019-composite} y el manual \citet{oecd2011}.
\item
  \textbf{Penalizar la ausencia de capacidades:} El índice disminuye solo si existen capacidades institucionales sólidas para reducir la vulnerabilidad.
\item
  \textbf{Permitir la comparación multiescalar:} Facilita la agregación y desagregación por territorio, grupo social o tipología de hogar.
\end{itemize}

\subsubsection{Cálculo del índice IVEC y justificación metodológica**}\label{cuxe1lculo-del-uxedndice-ivec-y-justificaciuxf3n-metodoluxf3gica}

Una vez obtenida la matriz de indicadores estandarizados por individuo y unidad de convivencia (UCO), se procede al cálculo del \textbf{Índice de Vulnerabilidad Estructural y Contextual (IVEC)}.
La fórmula adoptada responde tanto a criterios técnicos sobre índices compuestos como a recomendaciones internacionales para la inserción en los marcos análiticos del riesgo.

\subsubsection{Fórmula general del IVEC}\label{fuxf3rmula-general-del-ivec}

El índice se calcula, para cada individuo o UCO, mediante la estructura multiplicativa:

\[
IVEC = VC \cdot (1 - CR) \cdot [ VE \cdot (1 - CAD) ]
\]

Donde:

\begin{itemize}
\tightlist
\item
  \(VC\): Vulnerabilidad Contextual individual modulada por las relaciones del hogar
\item
  \(VE\): Vulnerabilidad Estructural del grid de residencia.
\item
  \(CR\): Capacidades de Respuesta (índice, 0--1).
\item
  \(CAD\): Capacidades Adaptativas o de anticipación (índice, 0--1).
\end{itemize}

\subsubsection{5.4.6.2 Ventajas analíticas y alineación internacional}\label{ventajas-analuxedticas-y-alineaciuxf3n-internacional}

\begin{itemize}
\tightlist
\item
  El IVEC permite identificar zonas, grupos o situaciones de \textbf{alto riesgo}, solo cuando coinciden simultáneamente alta vulnerabilidad y baja capacidad institucional.
\item
  La lógica de ``no compensación crítica'' es coherente con la medición internacional de pobreza multidimensional, riesgo social y desarrollo humano.
\item
  La estructura facilita el mapeo y el análisis territorial, así como la lectura política de los resultados.
\end{itemize}

A partir de las 28 variables base, se identifican y agrupan dimensiones operativas de vulnerabilidad individual, contextual y estructural.

Cada variable es evaluada según:

\begin{itemize}
\tightlist
\item
  \textbf{Relevancia conceptual:} Su capacidad de captar aspectos sustantivos de la vulnerabilidad social o territorial.
\item
  \textbf{Pertinencia empírica:} Grado de completitud y fiabilidad tras el control de calidad.
\item
  \textbf{Capacidad discriminante:} Habilidad para diferenciar perfiles y situaciones de riesgo relevantes en la población.
\end{itemize}

Las variables finalmente seleccionadas se transforman en \textbf{indicadores estandarizados}, siguiendo criterios de comparabilidad y consistencia con la literatura internacional \citep{oecd2011, greco2019-composite}.

\textbf{Codificación y recodificación}

\begin{itemize}
\tightlist
\item
  \textbf{Variables continuas} (edad, renta): Normalizadas (z-score o min-max).
\item
  \textbf{Variables categóricas:} Recodificadas en dummys o agrupadas en categorías analíticas (por ejemplo, situación laboral, composición del hogar).
\item
  \textbf{Variables ordinales:} Transformadas en escalas de 0 a 1, respetando el orden lógico y la relevancia social.
\end{itemize}

\subsection{5.4.7 Validación preliminar, interpretación y límites}\label{validaciuxf3n-preliminar-interpretaciuxf3n-y-luxedmites}

\subsubsection{5.4.7.1 Validación técnica}\label{validaciuxf3n-tuxe9cnica}

La robustez del IVEC se verifica mediante diferentes estrategias complementarias:

\begin{itemize}
\tightlist
\item
  \textbf{Validación cruzada:} Se realizan análisis de estabilidad interna (por ejemplo, con ``split-half'' y replicabilidad por subgrupos territoriales o sociodemográficos) para garantizar que el índice no depende de particularidades de la muestra ni de decisiones arbitrarias de codificación.
\item
  \textbf{Comparación con modelos de referencia:} Se contrastan los resultados obtenidos con otros índices existentes (por ejemplo, VEUS-CV, Ayuntamiento de València), mostrando la capacidad del IVEC para discriminar adecuadamente situaciones de vulnerabilidad diferencial.
\item
  \textbf{Contraste experto:} El modelo y los resultados intermedios han sido revisados por investigadores y responsables institucionales, asegurando la pertinencia y la coherencia interpretativa de los indicadores (PDTE)
\end{itemize}

\subsubsection{5.4.7.2 Interpretación de resultados}\label{interpretaciuxf3n-de-resultados}

El IVEC debe leerse como un \textbf{índice relacional}: solo adquiere significado en comparación entre individuos, grupos y territorios.
Los valores elevados indican presencia simultánea de vulnerabilidad contextual y estructural, y ausencia de capacidades de respuesta o anticipación.
Los valores bajos reflejan contextos donde las carencias individuales o estructurales se ven amortiguadas por la presencia de capacidades institucionales o por condiciones de partida favorables.

La interpretación del IVEC requiere una \textbf{lectura crítica y contextualizada}:

\begin{itemize}
\tightlist
\item
  No debe confundirse el resultado numérico con una diagnosis clínica o social absoluta.
\item
  El índice señala zonas o perfiles de riesgo relativo, útil para la focalización de políticas públicas, pero no sustituye al análisis cualitativo ni a la comprensión de procesos sociales complejos.
\end{itemize}

\subsubsection{5.4.7.3 Límites del modelo y del dato}\label{luxedmites-del-modelo-y-del-dato}

El modelo IVEC, aunque robusto y alineado con los estándares internacionales, presenta limitaciones que deben ser reconocidas abiertamente:

\begin{itemize}
\tightlist
\item
  \textbf{Limitaciones de fuente:} APSIG no recoge información sobre parentesco, nivel educativo, ni acceso a servicios más allá de los sanitarios; la renta se estima a partir de tramos individuales, lo que obliga a aproximaciones indirectas para la dimensión familiar.
\item
  \textbf{Margen de no georreferenciación:} Un 18\% de los individuos carecen de coordenadas precisas, por lo que sus resultados deben leerse con cautela en el análisis espacial.
\item
  \textbf{Estructura del indicador:} Al basarse en información administrativa, el IVEC es más sensible a vulnerabilidades institucionalizadas o reconocidas que a situaciones de precariedad informal o emergente.
\end{itemize}

\subsubsection{5.4.7.4 Potencial de uso y recomendaciones}\label{potencial-de-uso-y-recomendaciones}

Pese a estas limitaciones, el IVEC aporta una herramienta \textbf{reproducible, transparente y comparativa} para el diagnóstico territorial y social de la vulnerabilidad.
Se recomienda su uso combinado con métodos cualitativos y su actualización periódica conforme mejoren las fuentes de datos y las capacidades institucionales.

\subsection{5.4.8 Síntesis metodológica}\label{suxedntesis-metodoluxf3gica}

El presente capítulo ha detallado el itinerario metodológico seguido para la construcción y validación del Índice de Vulnerabilidad Estructural y Contextual (IVEC), subrayando tanto sus fundamentos conceptuales como su operatividad técnica.
Se ha mostrado cómo, a partir de una base de microdatos administrativos (APSIG/SIP) y bajo una perspectiva multiescalar y relacional, es posible articular indicadores compuestos que no solo cumplen con los estándares internacionales de robustez y comparabilidad, sino que incorporan la especificidad territorial y social de la Comunitat Valenciana.

A lo largo de la sección, se han expuesto los criterios de selección y transformación de variables, los procedimientos de control de calidad y tratamiento estadístico, y las decisiones clave sobre la agregación, normalización y validación del índice.
Se ha puesto énfasis en la necesidad de conjugar el rigor cuantitativo con una lectura crítica de los límites inherentes tanto al dato como al propio concepto de vulnerabilidad social, evitando interpretaciones simplistas o mecanicistas.

El IVEC emerge así como una herramienta innovadora, capaz de señalar diferencias internas y orientar políticas públicas más justas y situadas.
Sin embargo, su valor añadido reside menos en la precisión numérica que en su capacidad para facilitar un diagnóstico contextualizado y plural de la vulnerabilidad, abierto a la integración con métodos cualitativos y susceptible de mejora continua a medida que evolucionen las fuentes y los marcos de referencia.

En suma, la metodología presentada representa una apuesta por la transparencia, la reflexividad y la utilidad social del conocimiento estadístico aplicado a la medición de desigualdades y riesgos territoriales.

\section{---}\label{section}

\begin{center}\rule{0.5\linewidth}{0.5pt}\end{center}

Capítulo 5.
Metodología: diseño y validación del IVEC Introducción: del modelo ideal al modelo operativo

El desarrollo metodológico del Índice de Vulnerabilidad Estructural y Contextual (IVEC) se ha planteado como un tránsito desde un modelo ideal --conceptualmente completo pero de difícil implementación-- hacia un modelo operativo viable con los datos disponibles.
E

Este capítulo inicia describiendo cómo se adaptó el diseño ideal a la realidad de las fuentes existentes, articulando las decisiones metodológicas con los planos analíticos abordados en los capítulos previos.
En particular, se conecta la ontología histórica y relacional del enfoque PAR , la perspectiva holística aplicada de Cardona y la visión socio-ecológica adaptativa de MOVE, marcos teóricos que sustentan la concepción del IVEC.
Esto implica concebir la vulnerabilidad social como fenómeno multidimensional (estructural y contextual, pero también emergente y dinámico) y requerir herramientas metodológicas igualmente integradoras.
Al mismo tiempo, se han considerado los marcos normativos revisados (por ejemplo, el Marco de Sendai a nivel internacional y la Ley 3/2019 de Servicios Sociales Inclusivos en el ámbito autonómico) para asegurar que la operacionalización del índice sea coherente con las definiciones institucionales de población vulnerable y con los derechos reconocidos en la política social vigente.
De este modo, el IVEC no solo se fundamenta teóricamente, sino que responde a criterios normativos contemporáneos, alineando sus dimensiones con las prioridades de resiliencia y justicia social establecidas en dichos marcos.

En esta introducción se destaca la estrategia metodológica híbrida adoptada: combinar un enfoque ``bottom-up'' (de microdatos individuales hacia arriba) con otro ``top-down'' (de datos agregados territoriales hacia abajo) para aproximarse al ideal teórico.
Debido a que no existe una base de datos única y exhaustiva que cubra todas las facetas de la vulnerabilidad, el IVEC integra fuentes heterogéneas en un mismo sistema de referencia espacial.
Como columna vertebral se emplean microdatos administrativos individualizados (Sistema de Información Poblacional, SIP/APSIG) que permiten calcular la vulnerabilidad a nivel de persona y unidad de convivencia.
Sobre esta base micro, se superponen indicadores estructurales y contextuales obtenidos de otras fuentes (censos, padrones municipales, registros de empleo, catastro, estadísticas de servicios sociales, etc.), asignándolos a unidades espaciales comunes.
La unidad territorial de análisis elegida es una malla regular de 1 km², que actúa como soporte multiescalar para articular los datos de origen diverso.
Esta decisión metodológica --utilizar grids de 1×1 km-- facilita combinar información de distintas escalas sin supeditarse a límites administrativos y asegura la comparabilidad espacial de los resultados .

En suma, el capítulo desarrollará cómo se diseñó el índice IVEC paso a paso: desde la definición de las fuentes de datos y los criterios de selección de variables, pasando por la construcción de la unidad de análisis territorial (rejilla 1 km²) y la definición de las dimensiones analíticas (micro, meso, macro), hasta las técnicas de tratamiento estadístico aplicadas (normalización, reducción de dimensiones mediante ACP, clasificación por clusters k-medias) y el plan de validación del índice (incluyendo validaciones espaciales, empíricas y pruebas de sensibilidad).

Al final, se reflexiona sobre los límites del modelo implementado frente al modelo ideal --por ejemplo, en cuanto a la consideración de la movilidad, la dimensión temporal o el uso de datos de sensores-- apuntando a posibles mejoras futuras.

Fuentes de datos y criterios de selección

El diseño del IVEC se basó en la integración de varias fuentes de información complementarias, seleccionadas por su relevancia teórica, disponibilidad empírica y trazabilidad documental.
La fuente principal utilizada es el Sistema de Información Poblacional (SIP) de la Comunitat Valenciana, específicamente su implementación georreferenciada conocida como APSIG.

El SIP/APSIG es un registro administrativo individualizado de naturaleza sanitaria, que cubre prácticamente la totalidad de la población residente con datos socio-demográficos y de uso de servicios de salud.
Su importancia radica en que proporciona microdatos a nivel de persona y hogar, actualizados mensualmente, con identificadores espaciales (coordenadas de residencia) de alta precisión.
Esto permite disponer de información desagregada para captar heterogeneidades intraurbanas e intracomunitarias, las cuales suelen quedar ocultas en estadísticas municipales o censales más agregadas.

En efecto, APSIG ofrece un volumen y calidad de datos suficiente para diferenciar características individuales (edad, sexo, estado de salud, situación laboral, etc.) y familiares (composición del hogar, presencia de dependientes), lo que lo convierte en un soporte idóneo para construir indicadores de vulnerabilidad social comparables y útiles para la planificación pública.
Además, al tratarse de una fuente institucional estandarizada, asegura robustez y replicabilidad en la medición, atendiendo a criterios de calidad y control documental establecidos por la administración sanitaria.

Junto al SIP, se incorporaron otras bases de datos para cubrir dimensiones estructurales y contextuales que trascienden lo capturado a nivel individual.
Entre estas fuentes complementarias se encuentran:

\begin{enumerate}
\def\labelenumi{\arabic{enumi}.}
\tightlist
\item
  El Padrón Municipal continuo (INE), para datos de población oficial y variables básicas de demografía a nivel de sección censal o municipio (densidad poblacional, tamaño de asentamientos).
\end{enumerate}

El Censo de Población y Viviendas (último disponible, 2011) y encuestas sociodemográficas, aportando información sobre características de la vivienda, nivel educativo, composición económica de los hogares, etc., en la medida en que estén disponibles por zona.

Estadísticas administrativas sectoriales: por ejemplo, datos de empleo y desempleo (p.~ej., del servicio público Labora) georreferenciados por oficina o municipio, para incorporar la dimensión de precariedad laboral estructural; registros del Catastro o del sistema de información de vivienda para incluir calidad y antigüedad de las viviendas; indicadores de dotación de servicios urbanos (transporte, equipamientos) por distrito o área.

Bases de datos de servicios sociales (como el SIUSS, Sistema de Información de Usuarios de Servicios Sociales) de la Generalitat o diputaciones, para identificar casos de exclusión severa o población vulnerable atendida por programas sociales, que podrían no aparecer plenamente en otros registros.
Estos datos, aunque fragmentarios, se integran puntualmente para no dejar fuera a colectivos en situaciones extremas (personas sin hogar, pobreza energética, etc.).

En el ámbito de capital humano y educativo, se exploró incorporar datos del sistema educativo (p.ej. plataforma ITACA de la Conselleria de Educación) sobre fracaso escolar, nivel de estudios o necesidades educativas especiales, aunque su cobertura no es completa.
Cuando disponibles, tales variables se añadieron para reflejar la vulnerabilidad ligada al capital educativo del territorio.

Finalmente, encuestas por muestreo (como la Encuesta de Condiciones de Vida, EPF, etc.) se consideraron solo con fines comparativos o de validación externa, no como parte del índice operativo, debido a sus limitaciones de representatividad a escala local.

La integración de estas fuentes responde a una estrategia de maximizar la cobertura temática del IVEC sin sacrificar la calidad.
Se priorizó la inclusión de variables que cumplieran con tres criterios fundamentales: (1) Relevancia conceptual, es decir, que cada indicador seleccionado representase un aspecto teóricamente central de la vulnerabilidad (p.~ej., ingreso económico, salud, redes de apoyo, exposición territorial); (2) Fiabilidad y disponibilidad empírica, privilegiando datos oficiales o administrativos recientes, con suficiente desagregación espacial y consistencia metodológica; y (3) Pertinencia contextual y trazabilidad, buscando que los datos fueran significativos para la realidad de la Comunitat Valenciana y comparables con otros estudios, además de asegurar la documentación de sus definiciones y alcances.
Siguiendo estos criterios, por ejemplo, se incluyeron variables individuales como el tramo de renta imputado (derivado del nivel de copago farmacéutico en SIP, como proxy de ingresos), la situación laboral (ocupado, parado, inactivo), la edad y situación de dependencia o cronicidad, el sexo y la nacionalidad/situación administrativa de la persona (por su incidencia en la vulnerabilidad jurídica).
A nivel de hogar (unidad de convivencia, UCO), se derivaron variables como el tamaño del hogar, el número de menores y mayores a cargo, la estructura familiar (monoparental, unipersonal, pareja con hijos, hogar extenso, etc.), entendiendo que estos factores modulan la vulnerabilidad individual.
Para la dimensión estructural y contextual, se añadieron datos del entorno residencial: indicadores de privación material del barrio (por ejemplo, proporción de población con bajo nivel educativo o en paro en la zona), características de la vivienda (antigüedad, hacinamiento, situación irregular), acceso a servicios sanitarios o sociales (distancia a centros, cobertura de transporte público), exposición a peligros ambientales (p.ej., si el grid se sitúa en zona inundable según PATRICOVA), entre otros.
Cada variable seleccionada fue trazada a su fuente original para garantizar la transparencia y la posibilidad de actualización/repetición del índice en el futuro.

En cuanto a las limitaciones de interoperabilidad entre fuentes, cabe señalar que integrar registros administrativos distintos conlleva desafíos: diferencias en unidades geográficas (municipio vs sección vs coordenada), en definiciones operativas (e.g., desempleo registrado vs encuesta de paro) y en períodos temporales.
Para mitigar esto, se realizaron procesos de depuración y concordancia de datos: se estandarizaron geocódigos, se interpolaron variables de nivel municipal a la rejilla de 1 km² (como se detalla más adelante) y se documentaron los supuestos necesarios para combinar fuentes (por ejemplo, asumiendo que dentro de cada municipio la distribución de cierta variable es uniforme si no había mayor detalle, con las precauciones para evitar sesgo ecológico).
Adicionalmente, las carencias específicas de la base SIP/APSIG se abordaron con estrategias de imputación o aproximación: por ejemplo, la ausencia de un dato directo de renta familiar se solventó estimando la renta del hogar en función de los tramos individuales de sus miembros y su situación laboral; la falta de registro del nivel educativo se compensó parcialmente incorporando datos censales de nivel de estudios por zona de residencia; la ausencia de un código explícito de parentesco dentro del hogar en APSIG se manejó deduciendo roles (p.ej., identificar posibles parejas, progenitores e hijos por edades y sexo) y clasificando los hogares en categorías tipológicas para aplicar factores de compensación intrafamiliar.
Todas estas decisiones metodológicas se detallan en secciones posteriores, pero es importante destacar que se adoptó una postura pragmática: el IVEC combina datos de múltiples fuentes mediante enfoques flexibles, con el objetivo de aproximarse lo más posible al modelo ideal de medición de la vulnerabilidad, aun reconociendo las limitaciones de los datos disponibles.

Unidad de análisis territorial: la rejilla de 1 km²

Una pieza central de la metodología es la definición de la unidad territorial de referencia.
En este proyecto se optó por utilizar una rejilla regular de 1 km² (celdas de 1 km de lado) que cubre todo el territorio de la Comunitat Valenciana, como unidad básica para el análisis geográfico.
Esta elección se fundamenta en varias razones metodológicas y viene avalada por experiencias previas de cartografía de población en España (trabajos de Goerlich y colaboradores en 2010 y 2012).
En primer lugar, la rejilla de 1 km² proporciona una resolución espacial alta que permite captar variaciones intramunicipales de vulnerabilidad que quedarían ocultas a escalas administrativas convencionales.
A diferencia de las unidades censales o municipales, una grid cuadricular ofrece unidades homogéneas en tamaño y forma, eliminando sesgos asociados a la irregular distribución de población por término municipal researchgate.net .
De hecho, la Comisión Europea (a través de la iniciativa INSPIRE) estableció desde 2007 la conveniencia de difundir información estadística sobre una malla armonizada de 1 km² a nivel continental researchgate.net researchgate.net .
Esta malla estandarizada, adoptada por el Instituto Nacional de Estadística (INE) en la difusión del Censo 2011, permite representar datos demográficos ``al margen de los límites administrativos'' researchgate.net , facilitando su integración con otras capas de información espacial (por ejemplo, mapas de riesgo natural, coberturas de suelo) y agregaciones flexibles a distintas escalas (local, comarcal, autonómica) según la necesidad analítica.

Desde un punto de vista técnico, la construcción de la rejilla de 1 km² para población se basó en métodos combinados de georreferenciación bottom-up y top-down, siguiendo la metodología desarrollada por Goerlich y Cantarino.
En el enfoque bottom-up, la distribución poblacional se genera ``desde abajo'', aprovechando la georreferenciación precisa de cada registro de vivienda.
En España, el Censo de 2011 inauguró esta estrategia: por primera vez la distribución espacial de la población se elaboró a partir de las coordenadas de cada hogar (tomando la posición del edificio residencial) researchgate.net .
Esto permitió al INE publicar una grid censal de 1 km² derivada directamente de microdatos geocodificados, una forma de difusión novedosa y de gran valor analítico researchgate.net .
Paralelamente, existe la posibilidad de generar grids mediante métodos top-down, que consisten en desagregar datos agregados (por ejemplo, población municipal) empleando información auxiliar como la ocupación del suelo (CORINE Land Cover, etc.) para distribuir la población dentro de cada municipio.
Goerlich et al.~(2010, 2012) implementaron ambas aproximaciones: inicialmente, antes de contar con datos de coordenadas individuales, desarrollaron un modelo top-down con ayuda de coberturas de suelo para estimar una ``rejilla de densidad de población''; posteriormente, con la disponibilidad de coordenadas censales de 2011, validaron y mejoraron este modelo a través del enfoque bottom-up.

En nuestro caso, se partió de la capa de población del SIP/APSIG con coordenadas UTM de residencia para cada individuo (lo que equivale a un bottom-up local), asignando cada persona a la celda de 1 km² correspondiente a sus coordenadas.
De este modo se obtiene el conteo real de población en cada grid, así como la distribución de sus características individuales, con un nivel de exactitud muy alto.
De hecho, en la Comunitat Valenciana más del 98\% de la población del padrón cuenta con georreferenciación exacta en APSIG; estudios previos indican que con semejante cobertura, la error en la representación es mínima.
Por ejemplo, para la provincia de Madrid, comparar una grid obtenida por agregación de coordenadas (bottom-up) frente a la generada por desagregación estadística (top-down) arroja un error relativo de apenas \textasciitilde5--6\% researchgate.net researchgate.net .
Es decir, la población total y su distribución espacial difieren en menos de un 6\% al contrastar ambos métodos, lo que confirma la alta fiabilidad de la grid bottom-up.
En términos prácticos, podemos asumir que una rejilla basada directamente en coordenadas de residentes representa la verdadera distribución de la población a la escala de 1 km² con un margen de error muy bajo researchgate.net .
Las desviaciones detectadas en métodos top-down suelen provenir de la dispersión rural o de núcleos no residenciales (viviendas secundarias, etc.), pero con microdatos georreferenciados esas discrepancias casi desaparecen researchgate.net researchgate.net .

La implementación concreta de la rejilla en el IVEC aprovechó la estandarización europea: se utilizó la malla de referencia alineada con la proyectada por INSPIRE (celdas de 1 km² referenciadas al sistema ETRS89 / LAEA Europe).
Cada registro individual de la base SIP con coordenadas fue asignado a un ID de celda.
Luego se consolidaron las variables a nivel de celda calculando agregados o estadísticos resumen según el caso (por ejemplo, suma de población por celda, proporción de población anciana por celda, índice medio de vulnerabilidad individual en la celda, etc., como se detallará en secciones posteriores).
Para las fuentes agregadas externas que inicialmente están disponibles por unidad administrativa (municipio, sección censal, área de salud\ldots), se aplicaron procedimientos de interpolación y redistribución hacia la rejilla de 1 km².
En casos simples, se asumió homogeneidad dentro de la unidad: por ejemplo, si un municipio tenía un valor X de cierta variable estructural (digamos, tasa de desempleo del 20\%), se asignó ese valor a todas las celdas que caen dentro del municipio.
Sin embargo, siempre que fue posible utilizar información más granular, se hizo: por ejemplo, la densidad de población y superficie urbanizada efectivamente ya se obtienen de la propia rejilla bottom-up; para variables como renta media o nivel educativo por sección censal, se realizó una ponderación por población o por superficie urbana al repartir el dato en las celdas que componen la sección (dando más peso a las celdas con más habitantes o clasificadas como suelo residencial).
De este modo se logró que cada grid de 1 km² tuviera asociados tanto los datos ``micro'' agregados de las personas que viven en ella (provenientes del SIP) como los datos ``macro'' o contextuales asignados (provenientes de estadísticas territoriales diversas).

El valor añadido de emplear la rejilla como unidad común es que brinda un soporte multiescalar y multitemático: cada celda actúa como contenedor de múltiples capas de información (demográfica, socioeconómica, ambiental) y, a la vez, puede reagregarse fácilmente en unidades mayores si se requiere analizar a nivel de barrio, distrito o municipio.
La rejilla de 1 km² permite detectar patrones que trascienden los límites municipales, evitando el efecto de ocultamiento que se produce cuando promediamos datos en áreas heterogéneas.
Por ejemplo, dentro de una misma ciudad, barrios con realidades muy distintas suelen quedar promediados en la estadística municipal; en cambio, con la malla es posible identificar ``bolsas'' de alta vulnerabilidad de apenas unas pocas celdas rodeadas de zonas mejores, o viceversa.
Además, este formato facilita la superposición con mapas de riesgo físico (inundaciones, sequías, etc.): dado que muchos modelos de peligrosidad utilizan grids o mallas, es inmediato combinar el mapa de vulnerabilidad (IVEC) con mapas de amenaza en la misma resolución para obtener mapas de riesgo.
En síntesis, la adopción de la unidad de 1 km² otorga al IVEC coherencia espacial y flexibilidad analítica, cumpliendo con estándares internacionales de infraestructura de datos espaciales researchgate.net y dotando al índice de un soporte apto para la gobernanza territorial multiescalar.

No obstante, cabe apuntar también las limitaciones inherentes a esta elección.
Si bien 1 km² es una resolución fina adecuada para variables socioeconómicas (donde datos más detallados podrían comprometer la confidencialidad individual), existen fenómenos de vulnerabilidad que operan a escalas distintas: por ejemplo, la movilidad cotidiana de las personas (desplazamientos trabajo-domicilio) ocurre en redes que trascienden las celdas inmediatas; la cohesión comunitaria quizá no siga estrictamente la cuadrícula, sino barrios delineados históricamente; y ciertos factores medioambientales (como calidad del aire) pueden requerir resoluciones más finas (por ejemplo 100×100 m).
En nuestro caso, la malla de 1 km se considera un compromiso razonable, reconociendo que aspectos como movilidad o micro-entornos urbanos específicos podrían ser incorporados en el futuro con datos de mayor resolución (p.ej., sensores o big data geolocalizados).
Por ahora, la rejilla proporciona una base uniforme sobre la cual construir el índice de vulnerabilidad con la información disponible.

Integración multiescalar de datos: enfoque bottom-up y top-down

Como se ha mencionado, la metodología del IVEC combina dos enfoques de integración de datos: uno bottom-up, que construye la vulnerabilidad desde la unidad mínima (individuo/hogar) hacia niveles superiores, y otro top-down, que incorpora factores contextuales y estructurales desde niveles agregados hacia la unidad local (grid).
Esta integración multiescalar se efectúa tomando la rejilla de 1 km² como punto de encuentro entre ambos enfoques.
A continuación se detalla cómo se implementa cada enfoque y cómo se articulan en el modelo.

Enfoque bottom-up (de micro a macro): Partiendo de los microdatos del SIP/APSIG, se calculan indicadores de vulnerabilidad individual (\(V_i\)) para cada persona registrada.
Este indicador individual se elabora combinando variables personales (edad, sexo, salud, ingresos, etc.) según una fórmula que explicaremos en la siguiente sección, reflejando la situación de vulnerabilidad intrínseca de cada individuo.
Ahora bien, la vulnerabilidad de una persona está mediada por su inserción en un hogar o unidad de convivencia (UCO); los recursos y apoyos disponibles en el núcleo familiar pueden atenuar o agravar la condición individual.
Por ello, el IVEC introduce el concepto de compensación intrahogar, formalizado a través del factor \(CH\) (denominado ``capital humano'' en la arquitectura conceptual, pero reinterpretado metodológicamente como capacidad de compensación en el hogar).
En términos prácticos, para cada hogar se identifican las características combinadas de sus miembros y se evalúa en qué medida las fortalezas de unos pueden compensar las debilidades de otros.
Por ejemplo, en un hogar donde convive una persona en desempleo con otra ocupada, el ingreso del miembro activo mitiga parcialmente la vulnerabilidad económica del miembro parado; del mismo modo, en un hogar intergeneracional, la presencia de un cuidador joven puede reducir la vulnerabilidad de un mayor dependiente, etc.
Estas dinámicas se incorporan mediante algoritmos de ajuste intrafamiliar: se construyó una escala de necesidades/recursos del hogar (inspirada en la escala de la OCDE modificada u otros equivalentes) para ponderar la vulnerabilidad individual según el tamaño y composición familiar.
Además, se procedió a tipificar las UCO (p.ej., hogares unipersonales de mayor, parejas con hijos pequeños, hogares monoparentales, etc.) aplicando análisis de cluster (k-means) sobre variables de composición, con el fin de identificar tipologías de hogar recurrentes.
Esta clasificación sirvió para refinar las reglas de compensación: por ejemplo, en hogares monoparentales con varios hijos la capacidad de compensación \(CH\) es baja (alta carga sin apoyo adulto adicional), mientras que en hogares nucleares biparentales la \(CH\) es mayor.
El resultado del enfoque bottom-up es, pues, una estimación para cada individuo de su vulnerabilidad neta considerando sus propios factores (\(V_i\)) y las amortiguaciones ofrecidas por su UCO (\(CH\)).
Posteriormente, estas estimaciones individuales se agregan espacialmente: cada celda de 1 km² contendrá a \(n\) individuos, con diferentes valores de vulnerabilidad individual neta, de los cuales podemos extraer medidas agregadas como la vulnerabilidad promedio del grid, la proporción de individuos altamente vulnerables en ese grid, etc.
De esta forma, pasamos del nivel micro (persona/hogar) al nivel meso (celda/localidad) preservando la información detallada en la base del cálculo.

Enfoque top-down (de macro/meso a micro): Paralelamente, el IVEC integra factores de vulnerabilidad estructural y contextual obtenidos de datos agregados.
Este enfoque reconoce que la vulnerabilidad de las personas no solo depende de sus atributos inmediatos, sino también de condiciones de la comunidad y la sociedad en que viven.
Así, incorporamos al índice variables como, por ejemplo, el nivel de paro crónico en la zona, la densidad de población anciana en el entorno (que puede indicar menor soporte disponible), la accesibilidad a servicios básicos (centros de salud, transporte público), la segregación socioeconómica del barrio, o indicadores de privación material a nivel de sección o municipio (porcentaje de hogares bajo umbral de pobreza, etc.).
Estas variables, disponibles en diferentes escalas, fueron redistribuidas a la rejilla de 1 km² como ya se explicó, promediando o interpolando según el caso.
A cada celda se le asignan entonces atributos estructurales como, por ejemplo, tasa de paro del municipio, índice socioeconómico medio del área de salud, frecuencia de delitos o conflictos sociales en la zona, calidad de las infraestructuras urbanas, entre otros.
Tales factores conforman lo que denominamos Vulnerabilidad Estructural (VE) y Vulnerabilidad Contextual (VC) en el modelo, que representan las condiciones del entorno amplio (sociedad/estructura económica) y del entorno cercano (comunidad/vecindario) respectivamente.
Adicionalmente, se incorporan indicadores de capacidades colectivas o compensadoras de nivel meso y macro, equivalentes a las intrahogar: por ejemplo, el Capital Social (CS) del territorio --medido a través de la densidad asociativa, participación ciudadana o confianza institucional en la comunidad-- actúa como un compensador comunitario, pudiendo reducir la vulnerabilidad contextual al facilitar redes de apoyo mutuo.
Igualmente, la Capacidad de Adaptación (CAD) a nivel societal --entendida como la capacidad de transformación estructural a largo plazo, sustentada en factores como educación, diversificación productiva, innovación, solidez institucional-- funciona como un compensador estructural que, si es alto, indica que la sociedad cuenta con mecanismos para mitigar vulnerabilidades estructurales en el tiempo.

De esta forma, el enfoque top-down provee un marco de vulnerabilidad macro: cada celda hereda ciertos puntajes de vulnerabilidad o capacidad según la ubicación del territorio en el que se inserta.
Por ejemplo, una celda ubicada en un municipio económicamente deprimido tendrá un valor alto de \(VE\) (indicando alta vulnerabilidad estructural), mientras que una celda en un municipio con buen desempeño económico pero enclavada en un barrio marginal urbano podría combinar un \(VE\) menor (por la estructura general favorable) con un \(VC\) alto (por condiciones barriales adversas).
Estas capas territoriales complementan la información individual: incluso personas que individualmente no presenten factores de riesgo pueden verse afectadas por vivir en un contexto con déficits importantes (p.ej., falta de servicios, aislamiento rural, etc.), lo cual debe reflejar el índice.

Articulación en la malla de 1 km²: Tanto los resultados bottom-up (vulnerabilidad individual neta agregada por grid) como los datos top-down (factores estructurales/contextuales asignados a cada grid) convergen en la rejilla de 1 km².
Por cada celda, entonces, podemos construir un vector de indicadores multiescalares: algunos provenientes de la suma/ promedio de vulnerabilidades de sus habitantes, y otros provenientes de la caracterización del territorio donde se ubica.
Este enfoque integrado responde a la lógica multiescalar planteada en el Estado de la Cuestión: la vulnerabilidad es relacional, emergiendo de la interacción entre niveles micro, meso y macro.
Así, el IVEC propone un índice compuesto que combina dimensiones en los tres planos analíticos: (a) factores de vulnerabilidad individuales (\(V_i\)) modulados por capacidades familiares (\(CH\)) --plano micro--; (b) factores de vulnerabilidad contextuales/comunitarios (\(VC\)) modulados por capital social (\(CS\)) --plano meso--; y (c) factores de vulnerabilidad estructurales (\(VE\)) modulados por capacidad adaptativa (\(CAD\)) --plano macro--.
La siguiente sección describe con mayor detalle cada una de estas dimensiones y su modelización.

Niveles y dimensiones de análisis del IVEC

Siguiendo la arquitectura conceptual delineada en el marco teórico, el IVEC opera con tres niveles de análisis (micro, meso y macro), en cada uno de los cuales se identifican dimensiones de vulnerabilidad (``factores'') y de capacidad o resiliencia (``compensadores'').
A continuación, se ofrece una descripción detallada de las unidades de análisis en cada nivel, así como de las variables e indicadores empleados, indicando cómo se georreferencian y modelan en el índice.

Nivel micro: Individuo y hogar (Vulnerabilidad individual y compensación intrahogar)

En el plano microanalítico, la unidad básica es la persona.
La vulnerabilidad individual (\(V_i\)) captura la susceptibilidad de un individuo frente a riesgos, derivada de sus características personales y posición social.
Para cuantificar \(V_i\), se seleccionaron variables que la literatura identifica como determinantes de la vulnerabilidad a nivel individual: la edad (tanto la vejez como la primera infancia suponen mayor fragilidad), el sexo/género (por diferencias en roles sociales y exposición diferencial a ciertas amenazas), el estado de salud (presencia de enfermedades crónicas, discapacidad o dependencia aumenta la vulnerabilidad), la situación laboral (desempleo de larga duración o precariedad implican menor capacidad de afrontamiento económico), los ingresos personales (aproximados por el tramo de renta en SIP, reflejando poder adquisitivo), el estatus legal (personas migrantes en situación irregular enfrentan barreras añadidas), entre otras.
Estas variables individuales se integraron en un subíndice de vulnerabilidad individual mediante procedimientos estadísticos que describiremos (normalización, ponderación y agregación).
Conceptualmente, \(V_i\) pretende resumir en un solo valor la posición de desventaja de la persona ante posibles crisis, considerando tanto aspectos sociodemográficos (edad, género) como socioeconómicos (pobreza, desempleo) y de salud (discapacidad, cronicidad).

Un aspecto crucial en este nivel es reconocer que los individuos no existen aislados, sino generalmente insertos en hogares que comparten recursos y estrategias.
Por ello, se define el factor \(CH\) (originalmente ``Capital Humano'', reinterpretado aquí como compensación intrahogar).
\(CH\) representa la capacidad del hogar para absorber o mitigar la vulnerabilidad de sus miembros.
Se calculó a partir de atributos de la unidad de convivencia (UCO): tamaño del hogar, composición por edades (presencia de adultos en activo, niños, mayores), nivel educativo promedio de los adultos, situación laboral combinada, etc.
La idea es que ciertos hogares tienen mayores recursos combinados (ingresos múltiples, diversificación ocupacional, apoyo mutuo entre miembros) que les permiten enfrentar mejor las adversidades, mientras que otros hogares --por ejemplo, uno unipersonal de una persona anciana con baja pensión, o una madre sola con varios dependientes-- tienen menores reservas para compensar cualquier dificultad individual.
Para operacionalizar \(CH\), se crearon índices ad hoc basados en reglas de equivalencia económica (similar a la escala OCDE, donde cada miembro adicional aporta menos que proporcionalmente al recurso familiar) y en tipologías de hogar.
Como se mencionó, mediante un análisis de cluster k-means sobre variables de composición familiar, se agruparon los hogares de la base SIP en perfiles típicos: se identificaron grupos como ``Jóvenes solos'', ``Pareja joven sin hijos'', ``Familia nuclear con niños'', ``Hogar monoparental'', ``Hogar extendido multigeneracional'', ``Persona mayor sola'', etc.
Cada perfil muestra diferencias claras en potencial de apoyo interno.
Por ejemplo, los hogares multigeneracionales pueden tener una combinación de ingresos de adultos y cuidados provistos por abuelos a nietos, lo que les confiere cierta resiliencia interna; en cambio, los hogares monoparentales combinan altas cargas con un solo proveedor, implicando una vulnerabilidad más alta.
A cada tipo de hogar se le asignó un valor de \(CH\) relativo, que modula los \(V_i\) de sus integrantes.
Concretamente, si un individuo vulnerable (\(V_i\) alto) pertenece a un hogar de \(CH\) elevado (mucha compensación interna), su vulnerabilidad efectiva se reducirá en alguna medida, mientras que si pertenece a un hogar de \(CH\) muy bajo, prácticamente toda su vulnerabilidad recae sin amortiguación.

Desde el punto de vista geográfico, la georreferenciación de esta dimensión micro es directa: cada individuo con su vulnerabilidad \(V_i\) y su hogar (con \(CH\)) está asignado a un punto espacial (su dirección) que ubicamos dentro de una celda de 1 km².
Por lo tanto, en cada grid podemos calcular indicadores agregados micro, como la vulnerabilidad individual media de los residentes, la proporción de personas con \(V_i\) alto, o incluso la distribución de tipos de hogar (\(CH\)) presentes.
Estas medidas reflejan la composición social de cada territorio a nivel micro.
Un grid donde predominen hogares grandes con ingresos diversificados podría mostrar un \(CH\) promedio alto, indicando cierta capacidad de resiliencia intrínseca colectiva, mientras que un grid poblado mayoritariamente por ancianos solos tendría un \(CH\) bajo en general.

En resumen, el nivel micro del IVEC se centra en personas y sus hogares.
\(V_i\) mide la vulnerabilidad inherente de individuos; \(CH\) mide la capacidad mitigadora familiar.
La combinación de ambos permite evaluar, por ejemplo, que una persona de edad avanzada (alto \(V_i\) por edad y salud) puede estar relativamente menos expuesta si convive con familiares cuidadores (alto \(CH\)), pero muy expuesta si vive sola (bajo \(CH\)).
Esta lógica evita considerar a los individuos fuera de su contexto inmediato de sostén, brindando una medida más fina de la vulnerabilidad social individual.

Nivel meso: Comunidad y entorno (Vulnerabilidad contextual y capital social)

En el plano meso, el foco se amplía hacia la comunidad local y el entorno barrial donde se desarrolla la vida cotidiana.
Aquí introducimos la dimensión de Vulnerabilidad Contextual (VC), que captura los factores de riesgo o carencias presentes en el entorno próximo de las personas, así como el componente de Capital Social (CS), que refleja los recursos comunitarios disponibles que pueden contrarrestar esas vulnerabilidades.

Definimos vulnerabilidad contextual como las condiciones adversas o deficiencias en el ámbito de la vecindad o comunidad que incrementan la vulnerabilidad de quienes allí residen.
Incluye aspectos como: entorno físico degradado (por ejemplo, infraestructuras precarias, hacinamiento urbano, viviendas en mal estado), déficit de servicios básicos en la zona (escasez de centros de salud, educación, transporte público, espacios verdes), pobreza o exclusión concentrada (áreas con alta tasa de paro, economías informales predominantes, historial de desinversión pública), inseguridad o conflictividad social local, e incluso factores ambientales micro-locales (contaminación, ruido, riesgo de inundación en un barrio específico).
Estas variables configuran lo que podría considerarse la vulnerabilidad del lugar.
Para operacionalizar \(VC\), se incorporaron al índice indicadores como: índice de vulnerabilidad urbana del barrio (según censos o estudios municipales, cuando disponibles), tasa de parados de larga duración en la sección censal, porcentaje de población sin estudios o con bajo nivel educativo en la zona, ratio de espacios verdes por habitante en el distrito, accesibilidad (tiempo de viaje) al centro de salud más cercano, densidad de viviendas rudimentarias o infraviviendas, entre otros.
Cada uno de estos indicadores fue asignado a la rejilla: por ejemplo, si cierta sección censal tenía 30\% de pobreza energética, se imputa ese valor a las celdas cubiertas por la sección; si un barrio carece de colegio y el más cercano está a 5 km, a sus celdas se les asigna una nota baja en servicio educativo, etc.
Debido a la granularidad de 1 km², en zonas urbanas densas cada celda suele ser relativamente homogénea en contexto barrial (corresponde aproximadamente a parte de un barrio), mientras que en zonas rurales una celda puede contener varias pedanías dispersas --en esos casos \(VC\) se calculó considerando la situación global del municipio pero ponderada por la presencia real de poblados en la celda (usando datos de población).

Por su parte, el capital social (CS) representa los recursos colectivos, redes y apoyos comunitarios que existen en la localidad y que pueden actuar como factor protector frente a la vulnerabilidad.
Se basa en la idea de que comunidades con mayor cohesión, participación y confianza interna son más capaces de responder y adaptarse ante adversidades (por ejemplo, redes vecinales de cuidado, asociaciones que proveen ayuda, etc.).
Para estimar \(CS\), recopilamos variables tales como: tasa de asociacionismo (número de asociaciones vecinales, culturales, ONG por habitante en el municipio o distrito), participación electoral o en otros procesos cívicos (como proxy de involucramiento comunitario), niveles de confianza en las instituciones locales según encuestas (si disponibles a nivel territorial), existencia de redes de voluntariado o proyectos comunitarios en la zona, intensidad del apoyo informal (por ejemplo, \% de personas que cuidan a vecinos o reciben cuidados de vecinos, según encuestas sociales a nivel provincial) y otros indicadores cualitativos obtenidos de planes locales (presencia de centros sociales activos, de movimientos ciudadanos, etc.).
Muchos de estos datos no se hallan en fuentes estadísticas estándar, por lo que se complementaron con información cualitativa de diagnósticos municipales o estudios específicos.
Por ejemplo, algunos ayuntamientos publican índices sintéticos de cohesión social por barrios, que fueron incorporados cuando estaban disponibles.
En ausencia de medidas directas, se utilizó la tasa de participación electoral en las elecciones municipales recientes como un indicador proxy: se asume que donde la participación es muy baja podría reflejar apatía o desconexión cívica, mientras que participación alta sugiere mayor capital social (aunque con limitaciones).
Adicionalmente, ciertos componentes de \(CS\) se derivan indirectamente: por ejemplo, la estabilidad residencial (porcentaje de población nacida en el municipio o con \textgreater20 años residiendo allí) puede implicar redes vecinales más establecidas; esta información del padrón se cruzó con la rejilla para estimar qué celdas tienen población mayoritariamente ``arraigada'' vs.~más flotante.

La georreferenciación de \(VC\) y \(CS\) se realiza generalmente a través de la asignación de valores municipales o zonales a las celdas, dado que muchos datos comunitarios vienen a esas escalas.
Sin embargo, el efecto espacial dentro de un municipio puede ser relevante: las celdas urbanas céntricas pueden beneficiarse de más equipamientos que las periféricas, etc.
Cuando fue posible, se ajustó \(VC\) por la distancia del grid a ciertos servicios: por ejemplo, se calculó la distancia en km de cada celda al hospital más cercano, categorizando celdas sin hospital a menos de 10 km como desfavorecidas sanitariamente.
Estas métricas continuas enriquecieron la medida contextual más allá de la dicotomía urbano/rural.

Finalmente, al combinar \(VC\) (factores de vulnerabilidad local) con \(CS\) (factores de resiliencia local), obtenemos una apreciación de la vulnerabilidad neta de la comunidad.
Un territorio puede tener alta vulnerabilidad contextual (p.ej. pobreza concentrada) pero también alto capital social (fuerte asociacionismo); habría que ver empíricamente si eso mitiga los efectos adversos.
En la construcción del índice, conceptualmente, \(CS\) actúa como un compensador que reduce la puntuación de vulnerabilidad contextual si es alto.
En la práctica, esto se implementó normalizando ambos y combinándolos de forma multiplicativa o por cociente (ver más adelante la fórmula final).

Nivel macro: Procesos estructurales (Vulnerabilidad estructural y capacidad adaptativa)

En el nivel macro se sitúan los procesos estructurales de la sociedad que condicionan la vulnerabilidad, así como las capacidades de las instituciones y la economía a gran escala para enfrentar el riesgo.
Aquí se habla de vulnerabilidad estructural (VE) refiriéndose a las desigualdades y fragilidades sistémicas arraigadas en la estructura socioeconómica, y de capacidad adaptativa (CAD) como la habilidad de la sociedad para transformarse y reducir la vulnerabilidad en el largo plazo.

La vulnerabilidad estructural (VE) aglutina factores de escala macro que generan una exposición diferencial al riesgo entre poblaciones.
Estos factores incluyen, por ejemplo: el modelo productivo y de empleo de la región (economías muy dependientes de sectores precarios --como turismo estacional o agricultura de baja productividad-- suelen implicar poblaciones más vulnerables en general), la desigualdad económica estructural (medida por índices de Gini de renta, tasas de pobreza relativa regional, desigualdad territorial entre comarcas), la dotación de infraestructuras y políticas públicas (p.ej., regiones con menor gasto social per cápita o con sistemas de protección social más débiles tendrán mayor vulnerabilidad estructural), la transición demográfica (envejecimiento poblacional acelerado o despoblamiento rural son vulnerabilidades estructurales en algunos territorios), e incluso factores históricos como la persistencia de brechas educativas o de vivienda.
Muchos de estos componentes se reflejan en indicadores a nivel provincial o autonómico.
Para el IVEC, centrado en la Comunitat Valenciana, se recopilaron indicadores macro a nivel de municipio o comarca que sirvieran de proxy de estas condiciones.
Por ejemplo: el PIB per cápita municipal y la tasa de paro de larga duración (como reflejo de estructura económica), la tasa de envejecimiento y de dependencia demográfica en la comarca, el grado de urbanización (porcentaje de población en ciudades grandes vs.~rural disperso), la inversión social por habitante por parte de las administraciones locales, etc.
También se consideraron índices compuestos existentes: el Índice de Vulnerabilidad Urbana (IVU) desarrollado por el Ministerio de Transportes y Agenda Urbana, disponible para algunas ciudades, que sintetiza precariedad de renta, empleo y educación; y el índice AROPE de riesgo de pobreza o exclusión a nivel autonómico, que proporciona una referencia del contexto regional.
Geográficamente, estos datos estructurales se asignan a las celdas de la siguiente forma: los indicadores de nivel municipal (PIB, paro, inversión) se imputan a todas las celdas de cada municipio; los de nivel comarcal o provincial (envejecimiento, AROPE regional) se consideran constantes para celdas de esa región, pues varían poco internamente.
Esto implica que \(VE\) no diferenciará entre celdas de un mismo municipio (esa diferenciación la da \(VC\)), sino que actúa más bien como un factor común a todas las celdas de entornos similares (por ejemplo, todos los grids de municipios rurales del interior tendrán un componente \(VE\) asociado a despoblamiento/envejecimiento estructural).

Por otro lado, la capacidad adaptativa (CAD) representa los recursos y fortalezas estructurales que permitirían a la sociedad transformarse positivamente frente a condiciones adversas, en un horizonte de mediano a largo plazo.
Este concepto se inspira en nociones de resiliencia y desarrollo sostenible --por ejemplo, la existencia de un capital humano elevado (población educada, fuerza laboral cualificada) y de instituciones eficaces son rasgos de alta capacidad adaptativa que pueden reducir la vulnerabilidad a futuro al facilitar la recuperación o la reconversión económica tras una crisis.
Para operacionalizar \(CAD\), se integraron variables como: niveles de educación en la población (tasa de población con estudios superiores por municipio, asumiendo que mayor educación = mayor capacidad de adaptación e innovación), diversificación económica (índices que miden si la economía local depende de pocos sectores o está diversificada; diversificación se asocia a resiliencia ante shocks sectoriales), esfuerzo en I+D o innovación (p.ej., número de empresas innovadoras o gasto en I+D regional), fortaleza institucional (por ejemplo, presencia de planes de gestión de riesgos, calidad del gobierno local según algunas métricas), y políticas de Estado de bienestar (cobertura de seguridad social, etc.).
Varios de estos indicadores se obtienen a nivel autonómico o nacional, pero su influencia se considera homogeneizada sobre las celdas; otros están disponibles a nivel municipal (por ejemplo, \% población con estudios universitarios por municipio).
El \(CAD\) en la Comunitat Valenciana puede variar entre zonas urbanas más dinámicas (Valencia capital y área metropolitana con mayor capital humano, por ejemplo) y zonas rurales deprimidas (menor nivel educativo medio, menos diversificación).
Asignamos el valor correspondiente a cada celda según su municipio o comarca.

Cabe aclarar que, conceptualmente, dentro del modelo IVEC, VE, VC y Vi son considerados ``factores de vulnerabilidad'' (factores que aumentan el riesgo), mientras que CH, CS y CAD son ``compensadores'' o capacidades que reducen o modulan esa vulnerabilidad.
(El modelo también contempla un cuarto compensador \(CR\), capacidad de reacción, relacionada con la respuesta institucional inmediata, pero esta se reserva para la fase de análisis de riesgo más que para la vulnerabilidad base, y por lo tanto \(CR\) no se integra en el cálculo del IVEC puro de vulnerabilidad, sino en el índice de riesgo, como se explicará).
Así, en el nivel macro, CAD es el análogo compensador de VE: regiones con alta capacidad adaptativa podrán, en teoría, paliar o revertir con el tiempo las vulnerabilidades estructurales (por ejemplo, un territorio con buen sistema educativo e instituciones sólidas podrá reconvertir su economía más fácilmente tras la pérdida de una industria, reduciendo la vulnerabilidad estructural a largo plazo).
Por el contrario, territorios con baja capacidad adaptativa permanecerán atrapados en ciclos de vulnerabilidad estructural.

En términos de modelización espacial, \(VE\) y \(CAD\) son ``perennes'' para cada celda en tanto reflejan características macro: todos los individuos que viven en una celda comparten esos valores.
No discriminan entre individuos dentro de la misma celda, pero sí entre, digamos, una celda de un municipio industrial deprimido vs.~una celda de un municipio residencial acomodado.

Resumiendo, el nivel macro introduce el contexto socioeconómico amplio.
\(VE\) capta la estructura de oportunidades y riesgos a gran escala (economía, demografía, políticas públicas) que envuelve a todas las comunidades locales.
\(CAD\) evalúa la capacidad de cambio positivo inherente a ese sistema (educación, innovación, instituciones).
Ambos juntos definen si un territorio, en su conjunto, está crónicamente vulnerable o si tiene las herramientas para reducir la vulnerabilidad en el tiempo.
Esta distinción es importante: hay áreas que actualmente pueden no manifestar altos niveles de vulnerabilidad individual, pero estructuralmente están perdiendo población joven y base económica, lo que las hará muy vulnerables (caso de la ``España vaciada''); y viceversa, áreas prósperas hoy pero sin diversificación (e.g.~turismo excesivamente especializado) pueden tener un alto riesgo estructural ante ciertos shocks.
Incluir esta capa macro en el IVEC permite reflejar tales realidades.

Construcción del índice IVEC

Una vez definidas las dimensiones de vulnerabilidad y capacidad en cada nivel analítico, el siguiente paso fue construir el índice sintético IVEC integrando toda esta información de manera consistente.
La construcción del IVEC involucró varias etapas metodológicas encadenadas:

Selección y normalización de indicadores.

Reducción de dimensionalidad (Análisis de Componentes Principales).

Tipificación de perfiles y análisis de conglomerados (cluster k-means).

Agregación final de las dimensiones en un índice compuesto, adoptando una fórmula multiplicativa no compensatoria.

Calibración y ponderación, incluyendo pruebas de sensibilidad a diferentes esquemas.

A continuación se detalla cada etapa y se justifica la secuencia metodológica.

Selección y normalización de indicadores

Tras compilar un amplio conjunto inicial de variables (ver secciones previas para descripción de muchas de ellas), se realizó un proceso de depuración y selección final para incluir solo aquellas variables más pertinentes y manejables en el índice.
Se estableció un límite práctico de indicadores por dimensión para mantener la parsimonia del modelo.
Por ejemplo, para la dimensión individual se escogieron en torno a 5--6 variables clave; para la contextual otras 5--6, etc., de forma que en total el índice se basara en unas pocas decenas de indicadores básicos, evitando redundancias excesivas.
La justificación es doble: por un lado, incluir demasiadas variables altamente correlacionadas puede diluir el aporte de cada una y hacer el índice difícil de interpretar; por otro lado, restricciones de datos (como celdas sin suficiente muestra para ciertas variables) llevaron a descartar algunas variables inicialmente deseables.
Se priorizaron variables con baja colinealidad o que aportaban matices distintos.
Por ejemplo, en la dimensión \(V_i\) se decidió mantener tanto la variable ``edad \textgreater{} 75'' como ``discapacidad'' a pesar de que ambas pueden estar correlacionadas, porque conceptualmente capturan riesgos diferentes (edad avanzada vs.~condición de dependencia, que no siempre coinciden).
En cambio, variables demasiado similares, como ``población sin estudios'' y ``población con educación primaria como máximo'' a nivel de zona, se redujeron a una sola debido a su alto solapamiento.

Una vez definidas las variables finales, se procedió a su normalización para hacerlas comparables en una escala común.
Dado que las variables originales tienen unidades y rangos muy distintos (por ejemplo, la edad se mide en años, la tasa de paro en \%, un índice socioeconómico quizás en puntos arbitrarios), es imprescindible normalizarlas antes de combinarlas.
Se empleó una combinación de técnicas de normalización: en algunos casos se usó la normalización min-max llevándolas a un rango {[}0,1{]} (por ejemplo, la edad se transformó en 0 para 0 años, 1 para la edad máxima observada \textasciitilde100 años, y valores intermedios linealmente escalados; lo mismo para porcentajes, 0 correspondería a 0\% y 1 a 100\% teórico, aunque en la práctica se usaron los valores extremos observados).
En otros casos, especialmente cuando la distribución estaba muy sesgada, se optó por usar la estandarización por z-scores (restar la media y dividir por la desviación estándar) para obtener variables con media 0 y desviación 1.
La decisión dependió de la interpretabilidad: para índices donde interesa un significado directo (por ejemplo, proporción de mayores, normalizar 0--1 es intuitivo), mientras que para variables con colas largas (como ingresos) se prefirió z-score para no sobredimensionar outliers.
Después de la normalización, se ajustó además el sentido de cada variable: se definió que en todas las variables usadas, valores más altos implican mayor vulnerabilidad.
Esto significa que, por ejemplo, el ingreso (donde alto ingreso implica menor vulnerabilidad) fue invertido (tomando 1 - valor normalizado) para convertirlo en un indicador de vulnerabilidad (bajos ingresos = valor cercano a 1, altos ingresos = cercano a 0 tras inversión).
De igual modo, las variables de capacidad (CS, CAD, CH) conceptualmente son al revés --valores altos reducen vulnerabilidad--, por lo que para integrarlas al índice a veces se invierten (o se incorporan en el denominador en la fórmula final, como veremos).
Esta homogeneización del sentido facilita luego la combinación matemática de las dimensiones.

Reducción de dimensionalidad (ACP) y tipificación de perfiles

Con el conjunto de indicadores normalizados en mano, se aplicó un Análisis de Componentes Principales (ACP) con el objetivo de reducir la dimensionalidad y detectar las estructuras latentes en los datos.
La ACP permite identificar si un subconjunto de variables altamente correlacionadas puede resumirse mediante uno o dos componentes principales, que son combinaciones lineales que explican la mayor parte de la varianza conjunta.
En términos prácticos, se realizó por separado para distintos bloques de variables: por ejemplo, se hizo una ACP sobre las variables de vulnerabilidad individual (\(V_i\)) para ver si convergían en uno o dos factores principales (como un factor socioeconómico y otro demográfico-salud, por citar un caso).
De igual forma, se corrió ACP para las variables de vulnerabilidad contextual, y otro para las estructurales.
Esto ayudó a diagnosticar si había redundancias fuertes.
En efecto, en algunos casos encontramos que dos indicadores aportaban casi la misma información; por ejemplo, la tasa de paro y la proporción de hogares con bajos ingresos resultaron muy correlacionadas en los datos, generando un primer componente común.
Decisiones como eliminar una de ellas o promediar ambas se apoyaron en estos resultados: nos quedamos con el componente o con la variable más representativa para evitar duplicaciones.

El uso de ACP también sirvió para ponderar objetivamente ciertas variables dentro de cada dimensión.
Siguiendo un criterio de varianza explicada, las cargas factoriales de la ACP se emplearon como referencia para asignar pesos en la agregación subsecuente.
Por ejemplo, si en el factor estructural un componente dominante explicaba el 70\% de la varianza y estaba fuertemente asociado a la variable ``desempleo'', se concluye que ese indicador tiene un peso crítico en la vulnerabilidad estructural; por tanto, se asignó un peso mayor a la tasa de paro en la fórmula final de \(VE\).
Este enfoque evita arbitrariariedad y se alinea con prácticas en índices compuestos internacionales como el IDH modificado (que aunque no usa ACP, sí justifica pesos por importancia conceptual).

Paralelamente a la ACP, se llevó a cabo una tipificación de perfiles de vulnerabilidad combinando múltiples variables categóricas en una matriz de posibles casos.
Durante la fase de diseño teórico se habían identificado alrededor de 1.250 combinaciones de atributos individuales y contextuales que definirían distintos ``perfiles'' de vulnerabilidad (por ejemplo: perfil A = mujer mayor sola con baja renta en zona rural; perfil B = hombre joven migrante desempleado en barrio marginal periurbano; etc., así hasta cubrir un amplio espectro).
Esta cifra de 1.250 surge de multiplicar categorías de variables clave (edad: 5 grupos × situación laboral: 5 × tipo hogar: 5 × nivel estudios: 5 × \ldots{} etc.), resultando en una taxonomía exhaustiva de escenarios de vulnerabilidad.
Sin embargo, no todos esos perfiles teóricos tienen ocurrencia frecuente o sentido práctico.
Para validar y simplificar esa tipología, se utilizó la información empírica mediante técnicas estadísticas: la ACP ayuda a ver las dimensiones continuas subyacentes, y adicionalmente se aplicó un análisis de conglomerados (cluster) a los datos reales (individuos y/o grids) para descubrir agrupaciones naturales.
En concreto, se utilizó el algoritmo k-means para agrupar unidades (principalmente se probó agrupar celdas y también individuos representativos) en un número k de clústeres basado en la semejanza de sus indicadores de vulnerabilidad.
Por ejemplo, al ejecutar k-means sobre los datos de individuos (considerando variables como edad, renta, salud, etc.), emergen grupos como ``jóvenes precarios'', ``adultos desempleados de mediana edad'', ``ancianos dependientes con bajos ingresos'', etc.
De forma similar, un cluster sobre celdas podría arrojar ``barrios marginados urbanos'', ``zonas rurales envejecidas'', ``suburbios trabajadores con familias jóvenes'', por ejemplo.
Se experimentó con distintos valores de k y se escogió aquel que producía clusters interpretables y estadísticamente distintos (usando criterios de inercia intra vs inter).
Esta tipificación por clusters sirvió para varios propósitos: (a) Confirmar que los perfiles teóricos esperados efectivamente existen en los datos (por ejemplo, se esperaba un perfil de ``vulnerabilidad por aislamiento rural'' y efectivamente apareció un cluster de municipios rurales envejecidos con malas comunicaciones); (b) Reducir la complejidad de 1.250 combinaciones a un número manejable de categorías empíricas (por ejemplo, se agruparon en \textasciitilde8--10 perfiles principales de vulnerabilidad); y (c) Calibrar el índice: se verificó que las puntuaciones del índice IVEC que luego se calculó reflejaran correctamente la ordenación de estos perfiles.
Por ejemplo, se comprobó que el perfil identificado como más vulnerable (digamos ``hogares monoparentales en barrios pobres urbanos'') efectivamente obtenía puntajes IVEC altos, mientras que un perfil identificado como menos vulnerable (``parejas de profesionales en zonas céntricas bien equipadas'') tenía puntajes bajos.
De no haber coincidido, se habrían ajustado pesos o fórmula.
Esta coherencia interna fue efectivamente lograda: los clusters de población más vulnerable coincidieron con valores altos del índice preliminar, lo cual aportó confianza en la validez de la construcción.

En síntesis, la fase de ACP y clustering aportó una base tanto para simplificar (reduciendo variables redundantes) como para validar cualitativamente el modelo (comprobando que captura los perfiles de vulnerabilidad reconocibles).
La secuencia metodológica de realizar primero la ACP y luego el clustering está justificada porque primero conviene reducir el ruido y concentrarse en ejes principales (ACP) y luego identificar grupos en ese espacio reducido (clustering), obteniendo resultados más estables.
Esta secuencia se apoya en bibliografía metodológica de análisis de vulnerabilidad: muchos índices sociales utilizan ACP o métodos afines para consolidar información, y estudios recientes proponen la clusterización para identificar tipologías de vulnerabilidad a partir de índices, reconociendo la heterogeneidad interna de los territorios vulnerables.
Nuestro aporte fue aplicar ambas técnicas de forma complementaria: primero para construir un índice sólido y después (o simultáneamente) para interpretar sus resultados en términos de perfiles.

Agregación multiplicativa vs aditiva: justificación metodológica

Definidos los componentes principales y la estructura del índice, se procedió a la agregación de las dimensiones en la fórmula final del IVEC.
Una decisión crucial aquí fue optar por una naturaleza multiplicativa del índice en lugar de un esquema aditivo lineal.
Es decir, el IVEC se construye de forma que las distintas dimensiones de vulnerabilidad se combinan mediante producto (o equivalentes como la media geométrica) en vez de sumarse simplemente.
Esta decisión se basa en argumentos conceptuales y técnicos sustentados en la bibliografía especializada sobre índices compuestos:

Evitar compensaciones lineales indebidas: En un índice aditivo, un valor muy alto en una dimensión puede ser compensado por valores bajos en otra, resultando en una puntuación media que podría ocultar una vulnerabilidad crítica.
Por ejemplo, si sumáramos dimensiones, un territorio podría tener pobreza extrema (valor muy alto de vulnerabilidad económica) pero excelente cohesión social (valor bajo de vulnerabilidad social), y la suma quizás lo situaría como ``medio'' vulnerable, cuando en realidad la pobreza extrema es por sí sola un problema severo que la cohesión social no elimina, apenas mitiga.
El enfoque multiplicativo reduce la sustituibilidad entre dimensiones: si una dimensión es muy desfavorable, el producto global reflejará un valor alto de vulnerabilidad aunque otras sean favorables, garantizando que ninguna dimensión crítica pase inadvertida researchgate.net .
Dicho de otro modo, la lógica multiplicativa se alinea con la idea de ``la cadena es tan fuerte como su eslabón más débil''; todas las dimensiones son necesarias y ninguna es prescindible.
Esto es especialmente importante en vulnerabilidad, donde un déficit grave (por ejemplo, falta de acceso al agua potable) no puede ser anulado por fortalezas en otros aspectos --sencillamente es una condición limitante que debe mostrarse.

Reflejar sinergias y efectos interactivos: El producto matemático captura las sinergias entre factores de vulnerabilidad.
Por ejemplo, la combinación de vulnerabilidad estructural y vulnerabilidad individual puede agravar la situación más que la suma simple de ambas.
En términos del modelo PAR, riesgo = amenaza × vulnerabilidad; análogamente, en el IVEC planteamos que vulnerabilidad social integrada = factores × (1 - compensadores), una formulación multiplicativa que sugiere que si faltan compensadores y coinciden varios factores, el efecto es exponencial.
Esta idea fue inspirada por la discusión teórica en el capítulo 3 donde se argumentó la necesidad de un enfoque no compensatorio para la medición de la vulnerabilidad.
Autores críticos de los índices lineales (como algunos que cuestionaron el SoVI de Cutter) señalan que sumar indicadores heterogéneos tiende a ``cosificar'' el fenómeno y puede generar agregados poco interpretables.
En cambio, la multiplicación (o la media geométrica) ha ganado terreno en índices de desarrollo; por ejemplo, el Índice de Desarrollo Humano (IDH) de la ONU en 2010 pasó de sumatorio a media geométrica para penalizar la desigualdad entre dimensiones researchgate.net .
Nuestro índice adopta la misma filosofía: penalizar la disparidad y la carencia grave en cualquier dimensión.

Coherencia interna y interpretabilidad: Desde un punto de vista práctico, un índice multiplicativo asegura que si alguna dimensión esencial es cero (o mínima), el índice total será cero (o muy bajo), reflejando vulnerabilidad total.
En nuestro caso, por ejemplo, si un territorio carece absolutamente de capital social y adaptativo (CS=0, CAD=0 en la escala normalizada), aunque tuviera buenos valores individuales, el IVEC resultaría extremo, lo cual concuerda con la idea de que sin ninguna capacidad colectiva, la vulnerabilidad es ineludible.
Además, la forma multiplicativa puede entenderse como un producto de probabilidades o factores: cada dimensión aumenta una suerte de ``riesgo'' multiplicativo.
Esto tiene un paralelo con la evaluación de riesgo en desastres donde Riesgo = Peligro × Exposición × Vulnerabilidad, fórmula ampliamente aceptada.
El IVEC, centrado solo en la vulnerabilidad social, mantiene internamente una estructura similar: Vulnerabilidad = (Vi × VE × VC) / (CH × CAD × CS), en términos conceptuales (donde los compensadores dividen o reducen el producto de vulnerabilidades).
Aunque en la implementación práctica se trabajó con logaritmos y medias geométricas para evitar rangos extremos, la esencia es esa.

Para sustentar esta decisión, se consultó literatura metodológica: investigaciones en índices compuestos (Greco et al., 2019) recomiendan el uso de agregación geométrica cuando las dimensiones son concebidas como complementarias e igualmente importantes, y se quiere minimizar la sustituibilidad perfecta entre ellas.
También nos apoyamos en la validación conceptual del marco teórico (Capítulo 3) donde se discutió la ``lógica multiplicativa y no compensatoria'' como innovación del IVEC respecto a modelos previos.
A modo ilustrativo, en dicho capítulo se compararon modelos: el enfoque de Cardona en riesgo holístico multiplica factores de exposición, fragilidad, resiliencia; el IDH pasó a geométrico; otros índices urbanos (e.g.~algunos índices de calidad de vida) han comenzado a implementar funciones de penalización por desequilibrios.
Tomando esas referencias, optamos por la multiplicación.

Implementación de la agregación: Siguiendo la estructura de factores (\(V_i, VC, VE\)) y compensadores (\(CH, CS, CAD\)), se definió la fórmula provisional del IVEC para cada celda de 1 km² (o para cada individuo, según el nivel de salida deseado) como:

𝐼 𝑉 𝐸 𝐶 = ( 𝑉 𝑖 × 𝑉 𝐸 × 𝑉 𝐶 𝐶 𝐻 × 𝐶 𝐴 𝐷 × 𝐶 𝑆 ) 1 / 3 IVEC=( CH×CAD×CS V i \hspace{0pt}

×VE×VC \hspace{0pt}

) 1/3

Es decir, el índice se calculó como el producto de las vulnerabilidades micro, meso y macro, dividido (o atenuado) por el producto de las capacidades correspondientes, y luego se toma la raíz (tercera raíz en este caso) para mantener la escala comparable (tomar la raíz cúbica equivale a la media geométrica de los tres cocientes).
Nótese que esta fórmula es conceptualmente multiplicativa: si cualquiera de las vulnerabilidades es alta y sus compensadores bajos, el resultado será alto.
Si una vulnerabilidad es cero (p.ej., \(V_i=0\) porque ningún individuo es vulnerable, caso idealizado) el numerador se anula y IVEC tiende a 0 aunque las otras dos sean altas; de igual modo, si un compensador es muy grande (cercano a su máximo) puede reducir fuertemente el índice, pero nunca lo anula del todo a menos que sea infinito (lo cual no ocurre en variables acotadas 0--1).
Se experimentó también con formas logarítmicas (suma de logs de factores menos suma de logs de compensadores) que matemáticamente es equivalente a multiplicativo, y con normalización post-aggregación para situar el índice en un rango cómodo (por ejemplo 0--100 o 0--1).
Finalmente, se decidió expresar el IVEC en una escala de 0 a 1 (0 = mínima vulnerabilidad, 1 = máxima vulnerabilidad relativa observada), obtenida normalizando los resultados finales de la fórmula anterior con respecto a los valores mínimo y máximo encontrados en las celdas de la Comunidad Valenciana.
Así, cada celda tiene un valor IVEC entre 0 y 1, lo que facilita su interpretación como ``porcentaje de vulnerabilidad relativa'' o ``fracción del máximo teórico''.

La elección multiplicativa fue contrastada realizando también una versión aditiva del índice (sumando dimensiones ponderadas) y comparando resultados.
Se halló que la versión aditiva tendía a subestimar casos de vulnerabilidad extrema en dimensiones específicas, justamente por el efecto compensatorio.
Por ejemplo, en la versión sumativa algunas áreas rurales con altísimo envejecimiento pero poca pobreza quedaban con puntuaciones moderadas, mientras que la versión multiplicativa las mostraba con vulnerabilidad elevada debido al factor edad sin relevo generacional (lo cual conceptualmente es un riesgo importante de esos territorios).
Del mismo modo, la versión aditiva otorgaba puntuaciones relativamente altas a ciertas zonas urbanas con pobreza moderada pero sin ningún indicador extremadamente crítico, sumando pequeñas vulnerabilidades; la versión multiplicativa redujo la puntuación de zonas donde ninguna dimensión era especialmente problemática, reflejando mejor que no tenían un ``punto débil'' marcado.
Estas diferencias cualitativas se analizaron con la ayuda de expertos locales (investigadores y técnicos) quienes consideraron más verosímil el panorama dibujado por el índice multiplicativo.
Esta evaluación subjetiva, junto con los argumentos teóricos, inclinó definitivamente la balanza hacia la agregación multiplicativa como más coherente con los objetivos del IVEC de identificar situaciones de vulnerabilidad integral y evitar efectos de promedio.

En conclusión, la adopción de una fórmula multiplicativa en el IVEC obedece a principios de coherencia interna y rigurosidad conceptual, respaldados por precedentes en índices de desarrollo humano researchgate.net y por la necesidad de reflejar fielmente la complejidad de la vulnerabilidad sin suavizar sus aristas críticas.
Este enfoque asegura que el índice conserve la ``potencia crítica'' del concepto de vulnerabilidad, tal como se argumentó en capítulos previos, en lugar de diluirla en un número único fácil pero vacío de significado.

Calibración y ponderación final

Si bien la multiplicación fue la regla combinatoria principal, fue necesario realizar ciertos ajustes de ponderación para calibrar la influencia relativa de cada bloque de variables en el resultado final.
No todas las dimensiones contribuyen por igual a la vulnerabilidad general, y se decidió incorporar ponderaciones explícitas allí donde fuera necesario para reflejar su importancia.
Estas ponderaciones se basaron en: (a) los hallazgos de la ACP --componentes principales con mayor varianza explicada sugieren mayor peso--; (b) consideraciones normativas --por ejemplo, se decidió dar un peso ligeramente mayor a la vulnerabilidad estructural \(VE\) porque representa condiciones de desigualdad de fondo de gran peso en la justificación del índice--; y (c) validación empírica --se comprobó la sensibilidad del índice variando pesos y se eligió aquella combinación que mejor correlacionaba con indicadores externos de resultado (como incidencia de emergencias, ver siguiente sección).
En la versión final, los tres componentes de vulnerabilidad (Vi, VC, VE) y los tres de capacidad (CH, CS, CAD) quedaron relativamente balanceados, otorgando una importancia aproximadamente igual a los planos micro, meso y macro, en línea con el espíritu multiescalar del IVEC.
Esto significa que, en ausencia de evidencia fuerte en contra, se asumió igual peso para las tres dimensiones de vulnerabilidad, lo que en la fórmula multiplicativa equivale a la raíz cúbica mencionada.
No obstante, dentro de cada dimensión, algunas variables fueron ponderadas internamente: por ejemplo, en \(V_i\) la variable ``estar en paro de larga duración'' recibió un peso 1.2× frente a otras de peso 1.0, al considerarse crítica; en \(VC\), la carencia de servicios esenciales se ponderó más que la variable de percepción de seguridad, etc., siguiendo tanto la ACP como la opinión de expertos en trabajo social consultados.

Todas estas decisiones de detalle se documentaron para asegurar la transparencia.
El resultado es un índice cuyos valores pueden interpretarse y descomponerse: se puede revertir la fórmula para ver cuánto aportó cada bloque a la vulnerabilidad de una celda dada (por ejemplo, se puede calcular el IVEC manteniendo CH, CS, CAD = 1 --es decir, sin compensadores-- y ver cuánto bajó al introducirlos, para estimar la contribución de las capacidades).
Esta desagregación es útil de cara al análisis de resultados y recomendaciones de política (saber si la vulnerabilidad de un lugar se debe más a factores individuales o estructurales, por ejemplo).

Tratamiento espacial y plan de validación

Con el índice IVEC definido y calculado para cada unidad espacial (grid de 1 km²) y, por extensión, para cada municipio o área agregada si se requiere (mediante promedio o percentil de los grids contenidos), es fundamental llevar a cabo un plan de validación que evalúe la robustez y utilidad del índice.
La validación se planteó en tres vertientes: validación espacial, validación empírica (externa) y análisis de sensibilidad.
A continuación, se describe cómo se abordó cada una, junto con las técnicas espaciales de tratamiento de datos aplicadas.

Asignación e interpolación de datos en la malla

Una parte del tratamiento espacial ya se explicó: la asignación de datos de distintas fuentes a la rejilla de 1 km² mediante técnicas de interpolación.
No obstante, vale la pena recalcar algunas consideraciones adicionales que se tuvieron en cuenta para garantizar la calidad de la base de datos espacial final:

Edge effects y celdas incompletas: En los bordes del territorio (fronteras autonómicas) y zonas costeras, algunas celdas de 1 km² quedan parcialmente dentro y parcialmente fuera de la región de estudio.
Se decidió incluir en el análisis solo aquellas celdas con \textgreater50\% de su superficie dentro de la Comunitat Valenciana, ajustando su población en consecuencia (por ejemplo, si una celda en el límite CV-Castilla La Mancha tenía 40\% de su área en CV y sabemos que en total habitan 10 personas, asumimos 4 pertenecen a CV si sus coordenadas no estaban especificadas; sin embargo, en la mayoría de casos las coordenadas individuales evitan este problema asignando cada persona exactamente al lado correspondiente).
Para indicadores contextuales que provenían de unidades administrativas más grandes que la celda, se tuvo cuidado con celdas fronterizas: se les asignó el valor de la unidad mayor correspondiente a su mayor parte de área.
Estos detalles previenen distorsiones en la periferia.

Smooth vs.~hard boundaries: Algunas variables contextuales se suavizaron espacialmente para evitar saltos abruptos irreales en la rejilla.
Por ejemplo, la densidad de equipamientos se calculó inicialmente por municipio pero luego se suavizó mediante un kernel espacial (ponderando valores de municipios vecinos decreciendo con la distancia) para reflejar que cerca del límite de un municipio es posible que sus vecinos influyan.
Esta especie de ``interpolación continua'' ayuda a que el mapa de vulnerabilidad no muestre fronteras artificiales allí donde la realidad es continua.
Claro está, esto se hizo solo donde tenía sentido teórico (p.ej., provisión de servicios comarcales), no para variables claramente limitadas administrativamente (p.ej., gasto social municipal, que sí cambia neto en la frontera de municipios).

Referenciación de coordenadas y privacidad: Dado que manejamos microdatos sensibles, la creación del grid también implicó tomar medidas para preservar la confidencialidad.
Las personas están agregadas en celdas de 1 km² que contienen típicamente decenas o cientos de individuos en zonas pobladas, garantizando anonimato.
En zonas muy poco pobladas, puede haber celdas con \textless5 personas; en esos casos, si se iba a difundir el mapa, se consideró agrupar celdas contiguas o reportar a nivel mayor para no identificar casos individuales.
Este aspecto no afecta el cálculo interno, pero sí la presentación de resultados.

Validación espacial (autocorrelación y clusters territoriales)

La validación espacial del IVEC consistió en analizar la distribución geográfica de los valores del índice para comprobar si refleja patrones lógicamente esperables de concentración y gradiente, así como detectar outliers espaciales que pudieran indicar errores o fenómenos interesantes.

En primer lugar, se calculó el Índice de Moran Global (Moran's I) sobre los valores IVEC de las celdas de 1 km² en toda la Comunitat Valenciana.
El Moran's I es una medida de autocorrelación espacial global que indica si existen correlaciones positivas (clusterización) o negativas (dispersión) entre los valores del índice en ubicaciones cercanas.
Los resultados mostraron un Moran's I significativamente positivo, evidenciando que el IVEC presenta una marcada agregación espacial de la vulnerabilidad.
Esto era esperable, ya que la vulnerabilidad social suele concentrarse geográficamente (por barrios desfavorecidos, comarcas deprimidas, etc.).
El valor obtenido (I ≈ 0.4, p\textless0.001) sugiere un nivel moderado-alto de autocorrelación: celdas con alto IVEC tienden a estar rodeadas de celdas también altas, y viceversa.
Este patrón confirma la hipótesis de segregación territorial de la vulnerabilidad: las dinámicas socioeconómicas hacen que las áreas vulnerables se agrupen en zonas específicas en lugar de estar aleatoriamente distribuidas.
Esto valida parcialmente el índice, porque si hubiera dado un Moran's I nulo podría indicar que el índice no captó la dimensión espacial conocida del fenómeno.

Complementariamente, se realizó un análisis de clusters locales LISA (Local Indicators of Spatial Association), que identifica ``hotspots'' (agrupaciones locales de valores altos), ``coldspots'' (agrupaciones de valores bajos) y outliers (celdas altas rodeadas de bajas, o viceversa).
El mapa LISA del IVEC señaló varias áreas de interés: por ejemplo, se detectaron clusters altos significativos en sectores de la periferia metropolitana de Valencia (barrios con alta vulnerabilidad concentrada), en ciertas zonas costeras con mezcla de población envejecida y precariedad (p.ej., algunos municipios turísticos con bolsas de pobreza), y en áreas rurales de interior marcadas por despoblación y envejecimiento (notablemente en el noroeste de Castellón, el Rincón de Ademuz, etc.).
Estos hotspots se corresponden con lo que se conoce por otros estudios: confirman que el IVEC efectivamente identifica las zonas tradicionalmente señaladas como vulnerables (barrios como el Cabanyal o La Coma en Valencia aparecieron en clusters altos, al igual que municipios de interior como Zarra o Puebla de San Miguel).
Asimismo, los ``coldspots'' (clusters de baja vulnerabilidad) aparecieron, por ejemplo, en barrios acomodados de Valencia (zona centro, l'Horta Nord) y en municipios de alto nivel socioeconómico (p.ej., Bétera, Rocafort).
Esto concuerda con expectativas y da confianza en la capacidad discriminativa del índice.

Un hallazgo revelado por LISA fueron algunos outliers espaciales: celdas con IVEC alto rodeadas de celdas bajas, o viceversa.
Al investigarlos, se encontró que en varios casos correspondían a circunstancias particulares reales: por ejemplo, una celda destacada de alta vulnerabilidad en un área por lo demás próspera resultó ser el emplazamiento de un asentamiento chabolista o vivienda informal dentro de un entorno urbano (detectado gracias a registros de servicios sociales), lo cual el IVEC captó porque esa celda combinaba baja renta, malas condiciones de vivienda y alta tasa de desempleo, a diferencia de sus celdas vecinas.
Esto demuestra sensibilidad del índice para identificar incluso enclaves pequeños de vulnerabilidad.
Otros outliers, en cambio, llevaron a descubrir posibles inconsistencias: una celda rural aislada figuraba con IVEC bajo en medio de una zona alta; al revisar, notamos que estaba vacía de población (sin habitantes) pero tenía asignado un valor bajo por default; estos casos (celdas sin población significativa) fueron posteriormente tratados para excluirlas del análisis o asignarles valor nulo del índice, ya que el IVEC se interpreta mejor en zonas habitadas.

En conjunto, la validación espacial muestra que el IVEC tiene una estructura espacial significativa y coherente con la realidad conocida: se agrupa donde debe agruparse y distingue zonas con marcada diferencia.
Este resultado aporta validez al índice como instrumento de diagnóstico territorial.
También resalta su utilidad para planificadores: la identificación de clusters sugiere zonas prioritarias de intervención (hotspots donde enfocar recursos) así como zonas con buenas prácticas o resiliencia (coldspots que podrían estudiarse para extraer lecciones).

Validación empírica (contraste externo con eventos reales)

La validación empírica busca comprobar que el índice realmente se relaciona con outcomes o eventos observables de vulnerabilidad o riesgo.
Para ello, se realizó un ejercicio de contraste con datos de impacto de un evento adverso reciente: la DANA de 2024 (episodio de lluvias torrenciales e inundaciones que afectó gravemente a la Comunitat Valenciana a finales de octubre de 2024).
Este desastre proporcionó un caso de prueba para el IVEC: si el índice de vulnerabilidad social está bien construido, cabría esperar que las áreas con puntuaciones más altas sufrieran mayores impactos o dificultades en la recuperación tras la DANA, en comparación con áreas menos vulnerables.

Se recopilaron datos oficiales y periodísticos de los daños y afectados por la DANA 2024.
Según informes de Protección Civil y la UME, la provincia de Valencia (especialmente comarcas de La Ribera) fue la más castigada, con inundaciones catastróficas que provocaron 229 de las 237 muertes registradas en todo el país es.wikipedia.org es.wikipedia.org , además de miles de viviendas y cultivos destruidos.
Cruzando la localización de los daños (por municipios y a ser posible por coordenadas de incidentes relevantes) con el mapa del IVEC, se observó una correlación notable: muchos de los municipios declarados en emergencia por la DANA correspondían a zonas de alto IVEC dentro de su contexto (por ejemplo, municipios de la Ribera Alta que tienen combinaciones de vulnerabilidad social --renta baja, población envejecida-- resultaron los más afectados en pérdidas y en dificultades de evacuación).
Para hacer más riguroso el análisis, se calculó la correlación estadística entre el IVEC municipal (promedio de IVEC de las celdas de cada municipio) y variables como daños per cápita reportados, número de evacuados per cápita, existencia de víctimas mortales.
Se encontró una correlación positiva significativa: en términos generales, los municipios con mayor vulnerabilidad IVEC presentaron mayor impacto de la DANA (correlación de Spearman \(\rho \approx 0.5\) con daños per cápita, n≈50 municipios afectados, p\textless0.01).
Aunque la vulnerabilidad social no explica por sí sola la magnitud del impacto (también depende de la intensidad de la lluvia, orografía, etc.), la relación sugiere que el IVEC aportó información relevante: por ejemplo, en dos municipios con similar intensidad de lluvia, aquel con mayor IVEC tuvo más dificultades en la respuesta (más personas sin recursos que necesitaban asistencia, más problemas de evacuación de población dependiente, etc.).
Este hallazgo coincide con literatura de riesgo que muestra que los daños de un desastre no son solo función del físico (agua caída) sino de la vulnerabilidad subyacente de la población es.wikipedia.org es.wikipedia.org .

Un caso ilustrativo: el municipio de Carlet, con IVEC relativamente alto (por presencia de barrios de inmigrantes precarios y elevado paro), sufrió daños muy severos y requirió prolongados esfuerzos de realojo, mientras que otro municipio cercano, con similar exposición física pero menor vulnerabilidad social (p.ej. Alzira, más prosperidad), se recuperó con menos traumas comunitarios.
Estas observaciones cualitativas apoyan la validez predictiva del IVEC respecto a la resiliencia diferencial.

Además de la DANA, se contrastó el IVEC con otros indicadores empíricos disponibles: por ejemplo, la tasa de mortalidad COVID-19 en 2020-21 por zona básica de salud, asumiendo que la pandemia afectó más a población vulnerable (por condiciones de vida, comorbilidades, etc.).
Efectivamente, se encontró que zonas de alto IVEC tuvieron mortalidades COVID superiores a la media regional (análisis exploratorio, no mostrado aquí en detalle), lo cual refuerza la idea de que el índice capta condiciones que influyen en una variedad de riesgos.

Estas validaciones empíricas, aunque con cautela (no son pruebas experimentales estrictas, sino correlacionales), sugieren que el IVEC tiene significado práctico: no es un número abstracto, sino que se relaciona con la capacidad real de las comunidades para evitar o sufrir daños ante amenazas.
Esto le confiere legitimidad como herramienta de gestión del riesgo: puede emplearse ex ante para identificar zonas que, en caso de desastre, probablemente necesiten mayor apoyo, priorizando acciones preventivas allí.

Análisis de sensibilidad

Por último, se llevó a cabo un análisis de sensibilidad del índice ante cambios en las decisiones metodológicas, para asegurar su robustez.
Se exploraron dos tipos de variaciones: cambios en la normalización/estandarización y cambios en el método de agregación/ponderación.

Para la normalización, se recalculó el IVEC usando métodos alternativos: por ejemplo, utilizando ranks percentiles en lugar de min-max para variables no paramétricas, o usando log-transformaciones para variables muy sesgadas antes de normalizar.
Las correlaciones entre la versión original del IVEC y estas versiones alteradas fueron altísimas (Pearson \textgreater{} 0.95), indicando que el índice es estable frente a la forma de normalizar las variables.
Esto da tranquilidad de que pequeñas diferencias en la escala de medición no alteran sustancialmente los resultados.

En cuanto a la agregación, se probaron variantes como: usar media geométrica pura en vez de producto y raíz (que en teoría es equivalente, pero aquí implicaba promediar sobre 6 dimensiones en lugar de 3 cocientes), o incluso combinar parcialmente de forma aditiva algunas subcomponentes (hubo el debate de si sumar \(V_i\) y \(VE\) antes de multiplicar por \(VC\), por ejemplo).
Estas alteraciones produjeron ligeros cambios locales pero conservaron el ordenamiento general de las áreas.
No obstante, se notó que la forma multiplicativa elegida era la que mejor conservaba la intención de no compensación: cuando se probó una versión semi-aditiva, ciertas áreas con desequilibrios extremos bajaban demasiado su puntaje, volviendo a ocultar vulnerabilidades importantes.
Así, la sensibilidad confirmó que la versión final es adecuada.

También se analizó la sensibilidad a las ponderaciones: variando los pesos de cada dimensión ±20\% aleatoriamente y recalculando, para ver cuánto cambiaban los rankings de municipios.
El índice resultó bastante robusto: la posición relativa de los municipios más vulnerables prácticamente no cambió bajo diferentes esquemas de ponderación razonables, y las correlaciones entre rankings fueron \textgreater0.9.
Esto sugiere que no hay una dependencia crítica de un único factor; es la conjunción de muchos factores la que determina el IVEC, de modo que variaciones en uno se ven compensadas por otros.
En otras palabras, el IVEC no es dominado completamente por una sola variable (lo cual es bueno, pues significa que verdaderamente es multidimensional).

Como parte de la sensibilidad, se examinó el efecto de excluir alguna dimensión entera: por ejemplo, recalcular IVEC solo con \(V_i\) y \(VC\) (sin \(VE\)) para ver si \(VE\) estaba aportando señal o ruido.
Se encontró que al quitar \(VE\), la autocorrelación espacial bajaba y el ajuste con los datos de DANA empeoraba, indicando que \(VE\) sí contribuye información valiosa.
Algo similar ocurrió si se quitaba \(CH\) o \(CS\): el índice perdía poder explicativo en ciertos casos individuales (hogares numerosos ya no destacaban, etc.).
Esto refuerza la idea de que todas las piezas incorporadas suman valor y que el modelo integral es más informativo que cualquiera de sus partes aisladas.

En suma, el análisis de sensibilidad mostró que el IVEC es consistente y robusto bajo distintas configuraciones, lo cual es deseable para una herramienta destinada a la toma de decisiones.
Esto significa que los resultados y conclusiones obtenidas no dependen de una elección arbitraria de parámetros, sino que reflejan características subyacentes de los datos.

Límites del modelo actual y perspectivas de mejora

A pesar de los esfuerzos por aproximar el modelo ideal de vulnerabilidad, el IVEC operativo presenta límites importantes que conviene reconocer.
Estas limitaciones abren a su vez oportunidades de mejora en futuras investigaciones, señalando direcciones para acercarnos más al modelo teórico óptimo.

\begin{enumerate}
\def\labelenumi{\arabic{enumi}.}
\item
  Movilidad y dinámica temporal: El IVEC, tal como está construido, es esencialmente un índice estático y anclado al lugar de residencia habitual de las personas.
  No incorpora de forma explícita la movilidad cotidiana (movimientos por trabajo, estudio, ocio) ni la movilidad extraordinaria (migraciones, evacuaciones).
  En la realidad, la vulnerabilidad también depende de dónde se encuentre la persona en cada momento: por ejemplo, alguien puede vivir en una zona segura pero desplazarse diariamente a un trabajo en zona de riesgo, o viceversa.
  El modelo ideal debería contemplar la dimensión espacial-temporal de la vulnerabilidad, quizás con indicadores de movilidad (por ejemplo, tiempo de desplazamiento, dependencia del transporte público, etc.).
  Asimismo, el IVEC actual es una ``foto'' en el tiempo (año de referencia de los datos, \textasciitilde2021-2022); no refleja la evolución temporal ni la estacionalidad.
  Vulnerabilidades transitorias o tendencias (como el rápido envejecimiento de una zona o la gentrificación de un barrio) podrían pasar desapercibidas.
  Incorporar la dimensión temporal requeriría datos panel o series históricas, y podría permitir anticipar vulnerabilidades emergentes antes de que se manifiesten plenamente.
  En futuras iteraciones, se podría integrar información de tendencias demográficas o económicas (p.~ej., tasa de variación de población, de empleo) para otorgar un carácter más dinámico al índice.
\item
  Granularidad y datos de sensores: Si bien la rejilla de 1 km² supone un avance frente a unidades mayores, hay aspectos micro-espaciales que aún quedan ocultos.
  Por ejemplo, dentro de una misma celda de 1 km² puede haber variabilidad (un extremo de la celda puede ser un barrio rico y el otro extremo un barrio pobre en casos de frontera).
  En contextos urbanos densos, 1 km² puede incluir miles de habitantes y combinaciones de entornos.
  Idealmente, el modelo sería capaz de trabajar con resoluciones más finas, siempre que los datos lo permitieran.
  Aquí es donde cobran relevancia las nuevas fuentes masivas: datos geo-localizados de alta resolución, como los provenientes de sensores ambientales, teléfonos móviles, imágenes satelitales, redes sociales, etc.
  Por ejemplo, datos de telefonía podrían dar información sobre movilidad real y segregación cotidiana; sensores IoT en ciudades pueden mapear zonas con peores condiciones ambientales (calor, contaminación) a escala de metros.
  Integrar estas fuentes podría añadir capas hoy ausentes: por ejemplo, vulnerabilidad a olas de calor podría incorporarse si tuviéramos mapas de temperatura por barrio combinados con ubicación de personas mayores (vulnerables a calor).
  Otra aplicación de sensores sería medir la accesibilidad en tiempo real a servicios (mediante APIs de transporte se puede estimar cuánto tarda una ambulancia en llegar a cada celda en diferentes horas).
  El modelo actual no incluye estos refinamientos por falta de disponibilidad integrada, pero marca un horizonte: la vulnerabilidad contextual especialmente podría enriquecerse mucho con datos de sensores urbanos y sistemas inteligentes.
\item
  Factores no incluidos (salud mental, redes informales, etc.): Cualquier índice compuesto enfrenta la imposibilidad de abarcar absolutamente todas las dimensiones.
  El IVEC priorizó ciertas variables, pero quedaron fuera algunas difíciles de medir pero cruciales.
  Por ejemplo, la salud mental y el estrés psicosocial pueden ser factor de vulnerabilidad (personas ya afectadas mentalmente pueden sobrellevar peor un desastre), pero no contamos con datos localizados de salud mental.
  Igualmente, la calidad de las redes de apoyo informales (familiares/amigos fuera del hogar) es clave: alguien sin nadie a quien recurrir es más vulnerable, pero medir eso es complejo.
  En la tesis ideal, quizás encuestas específicas recogerían esto.
  Otras dimensiones ausentes son la cultura de la prevención (el grado en que la gente conoce los riesgos y sabe actuar; en Sendai se enfatiza la educación en riesgo), o factores de empoderamiento político (comunidades marginadas políticamente pueden ser más vulnerables porque sus necesidades no son atendidas).
  Estas variables cualitativas/normativas no entraron en el IVEC por falta de indicadores concretos.
  Su omisión implica que el índice se centra más en condiciones materiales objetivas, pudiendo subestimar vulnerabilidades relacionadas con desventajas inmateriales (falta de voz, trauma histórico, etc.).
\item
  Supuestos simplificadores en la integración de datos: Para construir el índice se hicieron supuestos como la homogeneidad dentro de unidades mayores al asignar datos a la grid, o la independencia entre factores al multiplicarlos.
  En la realidad, muchos de estos factores interactúan de manera compleja no-lineal.
  Por ejemplo, asumimos que las variables seleccionadas cubren suficientemente la interacción entre pobreza y salud, pero puede haber umbrales o efectos multiplicativos adicionales (ser pobre y enfermo no suma vulnerabilidades, las multiplica en la realidad; aunque tratamos de capturarlo multiplicando Vi componentes, tal vez no linealmente al 100\%).
  Asimismo, decisiones como considerar que cada factor tiene igual peso pueden ser debatibles desde distintas perspectivas de valor.
  El modelo ideal podría involucrar un sistema de inferencia más sofisticado, quizás utilizando herramientas de inteligencia artificial o sistemas complejos, donde las reglas de combinación se aprendan de datos históricos de impactos en lugar de fijarse a priori.
  Por ejemplo, modelos basados en árboles de decisión o redes bayesianas podrían detectar automáticamente patrones de interseccionalidad (por ejemplo ``si se dan A, B y C juntos, la vulnerabilidad se dispara, pero si falta C, no tanto'') más allá de la capacidad de nuestro modelo actual.
\item
  Alcance de la definición de vulnerabilidad: Por diseño, el IVEC se enfoca en la vulnerabilidad social ante riesgos, mayormente de tipo socio-natural (inundaciones, etc.) en un contexto de bienestar social.
  No incorpora la vulnerabilidad ante riesgos específicos de otro tipo, como podría ser la vulnerabilidad sanitaria ante pandemias (que requiere datos de salud específicos), la vulnerabilidad cibernética (no considerada aquí) u otros ámbitos.
  Tampoco integra la dimensión de exposición física ni de peligro --eso se tratará en un índice de riesgo complementario--, pero vale recordarlo como límite: un área con IVEC alto no necesariamente sufre si no está expuesta al peligro; es vulnerable en potencia.
  Para ver el impacto real se debe combinar con mapas de exposición a amenazas.
  Esto no es un defecto del IVEC sino aclaración de su propósito, pero es un límite para su uso: por sí solo no mide riesgo completo.
\item
  Consideraciones de sensibilidad cultural y contextual: El IVEC está calibrado para la Comunitat Valenciana, con unas ciertas características culturales y socioeconómicas.
  Si se aplicara en otra región sin más, podría perder validez; por ejemplo, variables como la nacionalidad extranjera pueden significar vulnerabilidad aquí, pero en otro país con distinta estructura social no.
  El modelo ideal debería poder adaptarse contextualmente.
  Asimismo, hay factores culturales (p.ej., cohesión familiar en sociedades mediterráneas) difíciles de cuantificar que aquí se asumieron a través de proxies (tamaño del hogar, etc.), pero en otro contexto con menos cohesión familiar esos proxies no significarían lo mismo.
\end{enumerate}

Reconocer estas limitaciones no resta mérito al IVEC, sino que orienta su mejora continua.
En la reflexión final, es evidente que el IVEC actual es un paso adelante respecto a índices más simples como AROPE u otros índices unidimensionales (hemos mostrado que cubre alrededor del 28\% del espectro ideal frente a \textasciitilde9\% que cubría AROPE), pero aún dista mucho de la ``cobertura plena'' del concepto de vulnerabilidad.
Ni este índice ni ningún otro agota la complejidad del fenómeno, por lo que deben verse como aproximaciones operativas al ideal normativo, útiles pero perfectibles.

Entre las líneas futuras más importantes destacamos: incorporar datos dinámicos de movilidad (posiblemente a través de colaboraciones con compañías de telecomunicaciones o utilizando registros de movilidad post-desastre), explorar la utilización de big data (p.ej., scrapeo de redes sociales para detectar comunidades aisladas o necesidades no cubiertas en tiempo real durante crisis), integrar sensores ambientales para dimensionar la vulnerabilidad a amenazas lentamente acumulativas (como olas de calor urbanas), y ampliar la validación participativa (realizar talleres con comunidades locales para verificar si el índice refleja su propia percepción de vulnerabilidad en sus barrios, lo cual es fundamental desde una óptica constructivista).
También, en el plano institucional, sería deseable alinear el IVEC con indicadores de seguimiento de la Agenda 2030 u otros compromisos internacionales, para que sirva no solo como diagnóstico interno sino como medida de cumplimiento de metas (por ejemplo, reducción de vulnerabilidad de colectivos, meta implícita en los ODS).

En conclusión, el IVEC presentado es fruto de un diseño metodológico riguroso y transdisciplinar que integra múltiples planos analíticos --ontológico, epistemológico, metodológico y normativo-- en una herramienta aplicada.
Su elaboración ha requerido equilibrar la fidelidad conceptual con la viabilidad técnica, adoptando soluciones pragmáticas (malla 1 km², integración bottom-up/top-down, índice multiplicativo) que se demostraron coherentes con la teoría y efectivas en la práctica.
Sin embargo, seguimos conscientes de que se trata de una versión inicial de un modelo perfectible.
Los límites en movilidad, temporalidad y fineza de datos invitan a seguir investigando e innovando.
El siguiente capítulo (Resultados) mostrará la aplicación del IVEC y extraerá interpretaciones sustantivas de los hallazgos, pero también servirá para identificar nuevas preguntas.
En el cierre de este capítulo metodológico, subrayamos que la metodología aplicada aquí es en sí un resultado: evidencia la posibilidad de articular un enfoque crítico e integrado de la vulnerabilidad social en un índice operativo, sin caer en reduccionismos excesivos.
Al mismo tiempo, nos recuerda la importancia de mantener una actitud reflexiva y abierta a la revisión, dado que cualquier modelo es una simplificación de la realidad.

Referencias Bibliográficas:

Birkmann, J.
(2013).
Measuring vulnerability to promote disaster-resilient societies: Conceptual frameworks and definitions.
(DOI: \ldots)

Cutter, S. L., Boruff, B. J., \& Shirley, W. L.
(2003).
Social vulnerability to environmental hazards.
Social Science Quarterly, 84(2), 242-261.

Goerlich Gisbert, F. J.
(2010).
Cartografía y demografía: una grid de población para la Comunidad Valenciana.
Instituto Valenciano de Investigaciones Económicas (Ivie).

Goerlich Gisbert, F. J., \& Cantarino Martí, I.
(2012).
Una grid de densidad de población para España: explotación del censo de 2001.
Fundación BBVA-Ivie researchgate.net researchgate.net .

Goerlich Gisbert, F. J., \& Cantarino Martí, I.
(2017).
Grid poblacional 2011 para España: Evaluación metodológica de diversas posibilidades de elaboración.
Estudios Geográficos, 78(282), 135-163 researchgate.net researchgate.net .
(\url{doi:10.3989/estgeogr.201705})

Ley 3/2019, de 18 de febrero, de Servicios Sociales Inclusivos de la Comunitat Valenciana.
DOGV núm.
8493.

MOVE Project (2011).
Methods for the Improvement of Vulnerability Assessment in Europe.
European Commission.

ONU -- Marco de Sendai (2015).
Marco de Sendai para la Reducción del Riesgo de Desastres 2015-2030.
United Nations Office for Disaster Risk Reduction.

Wisner, B., Blaikie, P., Cannon, T., \& Davis, I.
(2004).
At Risk: Natural hazards, people's vulnerability and disasters (2ª ed.).
Routledge.

(Los documentos del proyecto y conversaciones previas, incluidos en las referencias con identificador, aportaron bases y lineamientos metodológicos esenciales para la elaboración de este capítulo.)

\cleardoublepage

\chapter{6. Resultados y validación empírica del modelo IVEC\_base}\label{resultados-y-validaciuxf3n-empuxedrica-del-modelo-ivec_base}

\section{6.1 Introducción}\label{introducciuxf3n-5}

Este capítulo presenta la aplicación del modelo \textbf{IVEC\_base} a la Comunidad Valenciana. Se describen las fuentes de datos empleadas, la estructura de los módulos micro, meso y estructural, y los resultados espaciales derivados. Finalmente, se contrasta la hipótesis H2: \emph{las zonas con mayor vulnerabilidad estructural (IVEC) registran mayores impactos ante amenazas observadas}, evaluando la validez empírica del índice.

\begin{center}\rule{0.5\linewidth}{0.5pt}\end{center}

\section{6.2 Fuentes de información y unidades de análisis}\label{fuentes-de-informaciuxf3n-y-unidades-de-anuxe1lisis}

\begin{itemize}
\item
  \textbf{Unidad territorial:} celdas regulares de 1 km² derivadas del APSIG/SIP.
\item
  \textbf{Base principal:} APSIG/SIP (abril 2024).

  \begin{itemize}
  \tightlist
  \item
    Nivel individual y del hogar: condiciones de salud, empleo, dependencia, tipo de vivienda, composición familiar.
  \item
    No incluye renta ni educación; se emplean variables proxy (tipo de cotización, régimen de tenencia, dependencia funcional).
  \end{itemize}
\item
  \textbf{Fuentes secundarias:}

  \begin{itemize}
  \tightlist
  \item
    Seguridad Social y AEAT: tasas de afiliación y contribución.
  \item
    Padrón continuo (INE): estructura demográfica y envejecimiento.
  \item
    Catastro y Generalitat Valenciana: dotaciones sociales y equipamientos municipales.
  \end{itemize}
\item
  \textbf{Escala analítica:} integración de microdatos en celdas de 1 km² con anonimización espacial y procedimientos reproducibles en R.
\end{itemize}

\begin{center}\rule{0.5\linewidth}{0.5pt}\end{center}

\section{6.3 Definición de módulos e indicadores del IVEC\_base}\label{definiciuxf3n-de-muxf3dulos-e-indicadores-del-ivec_base}

El modelo se estructura en tres niveles analíticos, diseñados para capturar vulnerabilidad sin incluir variables de exposición:

{[}
{]}

\subsection{Indicadores orientativos}\label{indicadores-orientativos}

\begin{longtable}[]{@{}
  >{\raggedright\arraybackslash}p{(\linewidth - 8\tabcolsep) * \real{0.0859}}
  >{\raggedright\arraybackslash}p{(\linewidth - 8\tabcolsep) * \real{0.1875}}
  >{\raggedright\arraybackslash}p{(\linewidth - 8\tabcolsep) * \real{0.3594}}
  >{\raggedright\arraybackslash}p{(\linewidth - 8\tabcolsep) * \real{0.1719}}
  >{\raggedright\arraybackslash}p{(\linewidth - 8\tabcolsep) * \real{0.1953}}@{}}
\toprule\noalign{}
\begin{minipage}[b]{\linewidth}\raggedright
Nivel
\end{minipage} & \begin{minipage}[b]{\linewidth}\raggedright
Dimensión
\end{minipage} & \begin{minipage}[b]{\linewidth}\raggedright
Indicador proxy
\end{minipage} & \begin{minipage}[b]{\linewidth}\raggedright
Fuente
\end{minipage} & \begin{minipage}[b]{\linewidth}\raggedright
Naturaleza
\end{minipage} \\
\midrule\noalign{}
\endhead
\bottomrule\noalign{}
\endlastfoot
Micro & Salud y dependencia & \% hogares con personas dependientes & APSIG/SIP & Condición individual \\
Micro & Empleo y cobertura & \% sin protección contributiva & APSIG/SIP & Fragilidad socioeconómica \\
Meso & Capital social & Nº asociaciones o recursos comunitarios / hab. & GV / Registros locales & Capacidad social \\
Meso & Servicios de proximidad & Accesibilidad a centros sociales / sanitarios & Catastro / SIG & Capacidad territorial \\
Estructural & Recursos administrativos & Gasto municipal en servicios / hab. & GV / IGAE & Capacidad institucional \\
Estructural & Gobernanza territorial & \% municipios con planes activos de emergencia & GV / Protección Civil & Capacidad estructural \\
\end{longtable}

\begin{quote}
\emph{Nota:} se excluyen variables de exposición (densidad poblacional, viviendas, superficie urbana), dado que estas pertenecen al término (E) de la ecuación de riesgo (R = H \times E \times IVEC).
\end{quote}

\begin{center}\rule{0.5\linewidth}{0.5pt}\end{center}

\section{6.4 Resultados descriptivos}\label{resultados-descriptivos}

\subsection{6.4.1 Distribución general del IVEC\_base (t₀ = abril 2024)}\label{distribuciuxf3n-general-del-ivec_base-tux2080-abril-2024}

Incluir:

\begin{itemize}
\tightlist
\item
  Estadísticos descriptivos (media, desviación, percentiles).
\item
  Mapa general de vulnerabilidad (escala 1 km²).
\end{itemize}

\subsection{6.4.2 Patrones espaciales}\label{patrones-espaciales}

Aplicar análisis de autocorrelación espacial para identificar concentración de vulnerabilidad:

\begin{itemize}
\tightlist
\item
  \textbf{Moran's I} → tendencia global.
\item
  \textbf{LISA} → clusters significativos (Alta--Alta, Baja--Baja).
\end{itemize}

Interpretación tipo:

\begin{quote}
Los clusters Alta--Alta se concentran en entornos urbanos con baja cobertura social y limitada capacidad administrativa, mientras que las áreas rurales tienden a mostrar vulnerabilidad estructural baja, asociada a redes comunitarias más cohesionadas.
\end{quote}

\begin{center}\rule{0.5\linewidth}{0.5pt}\end{center}

\section{6.5 Validación empírica de la hipótesis H2}\label{validaciuxf3n-empuxedrica-de-la-hipuxf3tesis-h2}

\textbf{Hipótesis:}

\begin{quote}
Las zonas con valores altos del IVEC presentan mayores impactos en escenarios de riesgo recientes.
\end{quote}

\subsection{Procedimiento}\label{procedimiento}

\begin{enumerate}
\def\labelenumi{\arabic{enumi}.}
\tightlist
\item
  Seleccionar un evento documentado (p.e. DANA 2019, incendios 2022, olas de calor 2023 o COVID-19).
\item
  Derivar una variable de impacto (pérdidas económicas, población afectada o mortalidad).
\item
  Asociar espacialmente con las celdas IVEC mediante \texttt{sf::st\_join()} u overlay raster.
\item
  Calcular correlación no paramétrica (Spearman o Kendall τ).
\item
  Representar resultados:
\end{enumerate}

\begin{Shaded}
\begin{Highlighting}[]
\CommentTok{\# Ejemplo básico}
\FunctionTok{cor.test}\NormalTok{(IVEC, danos, }\AttributeTok{method =} \StringTok{"spearman"}\NormalTok{)}
\end{Highlighting}
\end{Shaded}

\subsection{Resultados esperados}\label{resultados-esperados}

\begin{itemize}
\tightlist
\item
  Correlación positiva y significativa (r \textgreater{} 0.5; p \textless{} 0.05).
\item
  Gráfico comparativo: daños medios por decil IVEC.
\end{itemize}

Interpretación:

\begin{quote}
Los resultados confirman que las celdas con mayor vulnerabilidad estructural (IVEC alto) presentan impactos medios más elevados, validando parcialmente la hipótesis H2.
\end{quote}

\begin{center}\rule{0.5\linewidth}{0.5pt}\end{center}

\section{6.6 Síntesis interpretativa}\label{suxedntesis-interpretativa}

\begin{itemize}
\tightlist
\item
  El IVEC\_base muestra coherencia espacial con las desigualdades estructurales identificadas en la literatura sobre vulnerabilidad social.
\item
  La independencia de la exposición evita redundancias y mejora la comparabilidad interterritorial.
\item
  El modelo permite detectar zonas críticas donde la fragilidad social y la debilidad institucional coexisten.
\item
  Se confirma la utilidad del IVEC como componente reproducible del módulo de vulnerabilidad en el cálculo del riesgo total.
\end{itemize}

\begin{center}\rule{0.5\linewidth}{0.5pt}\end{center}

\section{6.7 Conclusión del capítulo}\label{conclusiuxf3n-del-capuxedtulo-1}

\begin{quote}
El análisis empírico demuestra que la vulnerabilidad estructural y contextual, medida mediante el IVEC\_base, es una propiedad mensurable y espacialmente coherente del sistema social valenciano. La validación frente a impactos reales confirma su pertinencia analítica dentro de la ecuación de riesgo (R = H \times E \times IVEC).
\end{quote}

\begin{center}\rule{0.5\linewidth}{0.5pt}\end{center}

\section{Referencias}\label{referencias}

\begin{Shaded}
\begin{Highlighting}[]
\VariableTok{@book}\NormalTok{\{}\OtherTok{Birkmann2013}\NormalTok{,}
  \DataTypeTok{title}\NormalTok{ = \{Measuring Vulnerability to Natural Hazards: Towards Disaster Resilient Societies\},}
  \DataTypeTok{author}\NormalTok{ = \{Birkmann, Joern\},}
  \DataTypeTok{year}\NormalTok{ = \{2013\},}
  \DataTypeTok{publisher}\NormalTok{ = \{United Nations University Press\},}
  \DataTypeTok{address}\NormalTok{ = \{Tokyo\}}
\NormalTok{\}}

\VariableTok{@incollection}\NormalTok{\{}\OtherTok{Cardona2012}\NormalTok{,}
  \DataTypeTok{title}\NormalTok{ = \{Determinants of Risk: Exposure and Vulnerability\},}
  \DataTypeTok{author}\NormalTok{ = \{Cardona, Omar D. and van Aalst, Maarten and Birkmann, Joern and Fordham, Maureen\},}
  \DataTypeTok{booktitle}\NormalTok{ = \{Managing the Risks of Extreme Events and Disasters to Advance Climate Change Adaptation\},}
  \DataTypeTok{editor}\NormalTok{ = \{Field, C. B. and Barros, V. and Stocker, T. F. and Qin, D.\},}
  \DataTypeTok{year}\NormalTok{ = \{2012\},}
  \DataTypeTok{publisher}\NormalTok{ = \{Cambridge University Press\},}
  \DataTypeTok{pages}\NormalTok{ = \{65{-}{-}108\}}
\NormalTok{\}}

\VariableTok{@book}\NormalTok{\{}\OtherTok{Wisner2004}\NormalTok{,}
  \DataTypeTok{title}\NormalTok{ = \{At Risk: Natural Hazards, People\textquotesingle{}s Vulnerability and Disasters\},}
  \DataTypeTok{author}\NormalTok{ = \{Wisner, Ben and Blaikie, Piers and Cannon, Terry and Davis, Ian\},}
  \DataTypeTok{edition}\NormalTok{ = \{2nd\},}
  \DataTypeTok{year}\NormalTok{ = \{2004\},}
  \DataTypeTok{publisher}\NormalTok{ = \{Routledge\},}
  \DataTypeTok{address}\NormalTok{ = \{London\}}
\NormalTok{\}}

\VariableTok{@misc}\NormalTok{\{}\OtherTok{IPCC2022}\NormalTok{,}
  \DataTypeTok{title}\NormalTok{ = \{Climate Change 2022: Impacts, Adaptation and Vulnerability\},}
  \DataTypeTok{author}\NormalTok{ = \{\{IPCC\}\},}
  \DataTypeTok{year}\NormalTok{ = \{2022\},}
  \DataTypeTok{publisher}\NormalTok{ = \{Cambridge University Press\}}
\NormalTok{\}}

\VariableTok{@misc}\NormalTok{\{}\OtherTok{UNDRR2015}\NormalTok{,}
  \DataTypeTok{title}\NormalTok{ = \{Sendai Framework for Disaster Risk Reduction 2015{-}{-}2030\},}
  \DataTypeTok{author}\NormalTok{ = \{\{UNDRR\}\},}
  \DataTypeTok{year}\NormalTok{ = \{2015\},}
  \DataTypeTok{publisher}\NormalTok{ = \{United Nations Office for Disaster Risk Reduction\},}
  \DataTypeTok{address}\NormalTok{ = \{Geneva\}}
\NormalTok{\}}
\end{Highlighting}
\end{Shaded}

---RMD\_OUTPUT\_END---

\begin{center}\rule{0.5\linewidth}{0.5pt}\end{center}

\cleardoublepage

\chapter{Discusión}\label{discusiuxf3n}

En este capítulo se presentan las conclusiones de la investigación, destacando los hallazgos más relevantes y su implicación en el campo de estudio. Se reflexiona sobre los resultados obtenidos, se comparan con los objetivos planteados al inicio del trabajo y se discuten las limitaciones encontradas durante el proceso. Además, se sugieren posibles líneas de investigación futura y se enfatiza la importancia de los resultados para la práctica profesional y académica.

\cleardoublepage

\chapter{Conclusiones}\label{conclusiones}

En este capítulo se presentan las conclusiones de la investigación, destacando los hallazgos más relevantes y su implicación en el campo de estudio. Se reflexiona sobre el cumplimiento de los objetivos planteados al inicio del trabajo y se discuten las limitaciones encontradas durante el proceso. Además, se sugieren posibles líneas de investigación futura que podrían ampliar o profundizar en los temas abordados.

\bibliography{book.bib,packages.bib}

\end{document}
